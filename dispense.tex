\documentclass[10pt,a4paper]{amsbook}

\usepackage{ucs}
\usepackage[utf8x]{inputenc}
\usepackage{amsmath}
\usepackage{amsfonts}
\usepackage{amssymb}
\usepackage{amsthm}
\usepackage[italian]{babel}
\usepackage{fontenc}
\usepackage[all]{xy}
\usepackage{graphicx}
\usepackage{prooftree}

\usepackage{hyperref}
\hypersetup{%
    colorlinks=true, linktocpage=true, pdfstartpage=3, pdfstartview=FitV,%
    breaklinks=true, pdfpagemode=UseNone, pageanchor=true,
    pdfpagemode=UseOutlines, plainpages=false, bookmarksnumbered,
    bookmarksopen=true, bookmarksopenlevel=1, hypertexnames=true,
    pdfhighlight=/O,%hyperfootnotes=true,%nesting=true,%frenchlinks,%
    %urlcolor=webbrown, linkcolor=RoyalBlue, citecolor=webgreen,
    %pagecolor=RoyalBlue,%
    % uncomment the following line if you want to have black links (e.g., for
    % printing)
    %urlcolor=Black, linkcolor=Black, citecolor=Black, pagecolor=Black,%
    pdftitle={Dispense di Logica 2},%
    pdfauthor={},%
    pdfsubject={},%
    pdfkeywords={},%
    pdfcreator={pdfLaTeX},%
    pdfproducer={LaTeX with hyperref and classicthesis}%
}

\usepackage{tikz}
\usetikzlibrary{shapes,arrows}
% Define block styles
\tikzstyle{decision} = [diamond, draw, fill=white, 
    text width=3.5em, text badly centered, node distance=2cm, inner sep=0pt]
\tikzstyle{block} = [rectangle, draw, fill=white, 
    text width=5em, text centered, minimum height=2em]
\tikzstyle{line} = [draw, -latex']
\tikzstyle{noblock}=[rectangle,
    text width=5em, text centered, minimum height=2em]

\newcommand{\roundentry}[1]{*++[o][F-] {#1}} 
\newtheorem{esempio}{Esempio}[chapter]
\newtheorem{oss}{Osservazione}[chapter]
\newtheorem{prop}{Proposizione}[chapter]
\newtheorem{defi}{Definizione}[chapter]
\newtheorem{lem}{Lemma}[chapter]
\newtheorem{thm}{Teorema}[chapter]
\newtheorem{extra}{Esercizio}
\newtheorem{nota}{Nota}
\newtheorem{programmi}{Definizione}[section]
\newtheorem{progabaco}[programmi]{Teorema}
\newtheorem{corol}{Corollario}[chapter]


\newenvironment{mylisting}
{\begin{list}{}{\setlength{\leftmargin}{1em}}\item\bfseries}
{\end{list}}

\newenvironment{mytinylisting}
{\begin{list}{}{\setlength{\leftmargin}{1em}}\item\tiny\bfseries}
{\end{list}}

\title{Dispense di Logica 2}
\author{}
\date{}

\begin{document}

\maketitle

\tableofcontents

%\newcommand{\bew}{\mathsf{Bew}}
% per scrivere eqz più velocemente
\newcommand{\eq}[1]{\begin{equation} #1 \end{equation}}
% per fare le parentesi squadrate
\newcommand{\gdnum}[1]{\ulcorner #1 \urcorner}
% per scrivere pi velocemente un'eqz e darle un label
\newcommand{\eqconnome}[2]{\begin{equation} \label{eq:#2} #1 \end{equation}}

\chapter{Macchine di Turing}

\section{Introduzione}
Intuitivamente diciamo che una funzione \`e calcolabile se esiste un insieme
finito di istruzioni che ci dica esattamente cosa dobbiamo fare, senza lasciare
nulla di sottointeso ed ambiguo. Possiamo parlare quindi di una procedura
effettiva (meccanica) proprio perch\`e la sua esecuzione deve dipendere
unicamente dagli input e dalle regole che applico (fattori come il tempo, lo
spazio, il singolo esecutore non devono influire sul risultato) e quindi
possiamo immaginare tale procedura eseguibile da una macchina ipotetica.

Pensiamo ad una macchina come ad un sistema che possiede certe abilit\`a, \`e
quindi in grado di fare certe cose. Ci concentreremo sulle procedure effettive
che potremo scrivere con una certa macchina e che tipo di funzioni potremo
rappresentare.

\section{Definizione di Macchina di Turing}
Il logico inglese Alan Turing nel suo articolo del 1936 era interessato alla
questione di cosa volesse dire calcolabile e descrisse quindi dei dispositivi
astratti conosciuti come Macchine di Turing. Prima di tutto vediamo pi\`u in
dettaglio come Turing descrive una macchina.

Turing stesso paragon\`o una macchina al comportamento di un essere umano che
sta eseguendo un calcolo.  Immaginiamo quindi un uomo seduto ad una scrivania
munito di carta e penna: immaginiamo quindi quest'uomo come una testina di
controllo che possa scrivere simboli di un certo alfabeto su di un ipotetico
foglio di carta che noi chiameremo nastro.

Poich\'e non \`e importante il numero di fogli a disposizione dell'uomo,
supporremo la carta come una risorsa infinita: questo si traduce nel fatto che
il nastro su cui la testina di controllo scrive sia infinito in entrambe le
direzioni.

In secondo luogo, l'uomo pu\`o ricordare ed ha memoria finita, quindi dotiamo la
testina di un numero finito di stati di memoria. Infine, la scrittura su di un
pezzo di carta comporta l'avanzamento del processo di scrittura; parallelamente
la scrittura su nastro di un simbolo comporter\`a l'avanzamento verso destra o
verso sinistra della testina medesima.

Dopo queste analogie possiamo riassumere la definizione di Macchina di Turing in
questo modo.

Una macchina di Turing (MdT) \`e definita da un \textsl{insieme di
  regole} che definiscono il comportamento della macchina su un nastro
di input-output (lettura e scrittura). Il nastro pu\`o essere
immaginato come un nastro di lunghezza infinita diviso in quadratini
dette \textsl{celle}. Ogni cella contiene un simbolo $S_i$ di un certo
alfabeto oppure \`e vuota e questo viene rappresentato dal carattere
Blank ($B$). La macchina ha una testina che si sposta lungo il nastro
iniziando dalla cella che contiene il simbolo pi\`u a sinistra e
analizzando una cella alla volta: i simboli devono essere posizionati
in celle vicine e si possono concatenare anche pi\`u input dividendoli
al pi\`u da un carattere B di Blank. I caratteri di Blank si
susseguono anche al di fuori della sequenza di simboli dati in input,
occupando tutta la lunghezza infinita del nastro.  La testina \`e
dotata di un numero finito di stati di memoria $q_i$. Lo stato ha la
funzione di un'etichetta: ad uno stato $q_i$ corrisponde un'operazione
se viene letto un simbolo $S_j$ e un'altra operazione se viene letto
il simbolo $B$. Inoltre a seconda dello stato in cui la testina si
trova, sa che simboli ha letto prima.

\begin{figure}[htbp!]
\centering
\includegraphics[scale=0.8]{img/Nastro.jpg}
\end{figure}

La macchina legge, scrive oppure cancella i simboli nelle celle del nastro
seguendo l'insieme di regole o quintuple,  che ne definiscono il comportamento.
Tali regole sono di questo tipo:

\vspace{0.3cm}
(\textsl{stato-interno-corrente}, \textsl{simbolo-letto},
\textsl{simbolo-scritto}, \textsl{prossimo-stato-interno}, \textsl{direzione}).
\vspace{0.3cm}

Quindi ad ogni passo, la macchina legge un simbolo sul nastro e in accordo al
suo \textsl{stato interno corrente}:

\begin{enumerate}
\item scrive un simbolo sul nastro
\item decide il suo prossimo stato interno, e decide se spostare la testina a
sinistra (L) o a destra (R) di una posizione.
\end{enumerate}
\vspace{0.5cm}

Diamo di seguito un esempio per queste istruzioni:

\begin{itemize}
\item La quintupla $<q_{i}S_{j}S_{k}Rq_{l}>$ indica che se la macchina
  si trova nello stato interno $q_{i}$ e legge sul nastro il simbolo
  $S_{j}$, allora lo sostituisce con $S_{k}$ e sposta la testina di
  lettura di una cella verso destra passando allo stato $q_{l}$
\item La quintupla $<q_{i}S_{j}S_{k}Lq_{l}>$ indica che se la macchina
  si trova nello stato $q_{i}$ e legge sul nastro il simbolo $S_{j}$,
  allora lo sostituisce con $S_{k}$ e sposta la testina di lettura di
  una cella verso sinistra lungo il nastro passando nello stato
  $q_{l}$.\\ Notiamo che i simboli $S_{j}$ e $S_{k}$ possono
  coincidere (cio\`e non sovrascrive nulla $<q_{i}S_{j}S_{j}Rq_{l}>$),
  cos\`i come $q_{i}$ pu\`o essere uguale a $q_{l}$ (ossia rimane
  nello stesso stato $<q_{i}S_{j}S_{j}Rq_{i}>$).
\item La scrittura del simbolo $B$ di Blank corrisponde a cancellare
  il contenuto di una cella. Ad esempio la quintupla
  $<q_{i}S_{j}BRq_{l}>$ indica che se la macchina si trova nello stato
  $q_{i}$ e legge $S_{j}$, allora cancella il simbolo mettendo il
  blank e sposta la testina di lettura di una cella verso destra
  passando allo stato $q_{l}$.
\end{itemize}

Noi tratteremo, in accordo col fatto che il nostro ``uomo anonim\`o''
non pu\`o scegliere cosa fare ma solo eseguire regole date. Diamo
quindi la seguente definizione:

\begin{defi}
Una Macchina di Turing si dice \underline{deterministica} se dato uno
stato $q_{i}$ e un simbolo $S_{j}$ c'\`e \textsl{al pi\`u} una
quintupla che inizi per $<q_{i},S_{j}...>$ cio\`e la testina
trovandosi in $q_{i}$ e leggendo $S_{j}$ non pu\`o scegliere cosa fare
ma solo eseguire l'unica istruzione corrispondente, \textsl{se c'\`e},
altrimenti significa che la macchina si arresta.
\end{defi}

Inoltre \`e possibile descrivere il nastro in ogni fase della
computazione che fa la macchina. Introduciamo quindi la seguente
definizione:

\begin{defi}
Una \underline{descrizione istantanea} $\alpha$ di una macchina di
Turing $\tau$ \`e un'espressione contenente esattamente un $q_{i}$ (lo
stato), i simboli sul nastro, e la cella in cui si trova la testina al
momento.
\end{defi}

Quindi una descrizione istantanea (ID) $\alpha$ \`e una stringa della
forma $$S_1S_2S_3.....S_{i-1}qS_{i}S_{i+1}......S_{n}$$ dove

\begin{enumerate}
\item q \`e lo stato della macchina di Turing;
\item $S_1S_2S_3.........S_{n}$ \`e la porzione non-blank del nastro;
\item La testina \`e sopra il simbolo i-esimo.
\end{enumerate}
\vspace{0.5cm}

Vediamo ora come una Macchina di Turing effettua i suoi
calcoli. Inizialmente il nastro contiene una sequenza finita di
simboli, detta \textsl{sequenza di ingresso}. Assumiamo che all'inizio
la testina della macchina $\tau$ sia posizionata sulla prima cella di
input, cio\`e sulla prima cella in cui c'\`e un simbolo e diciamo che
$\tau$ si trova nello stato $q_0$. La testina vede un simbolo $S_i$ e
allora parte: cerca tra le quintuple che forniscono le varie
istruzioni quella che comincia per $q_0$ e ha come secondo elemento il
simbolo letto, del tipo $<q_0,S_i..>$. A questo punto vede quale
istruzione deve eseguire (se sovrascrivere un altro simbolo o se
cancellare), in quale direzione andare e in quale stato portarsi, e
cos\`i via. Se non c'\`e una tale quintupla, la macchina si arresta.

Osserviamo che Turing propose l'idea di tale macchina che fosse capace
di eseguire ogni tipo di calcolo su numeri e simboli. Noi per
semplicit\`a ci riconduciamo a calcoli su numeri naturali e vedremo
che ci possiamo ridurre a considerare un alfabeto costituito
unicamente dai simboli 1 e B: $$\Sigma=\left\{1,B\right\}$$ e quindi
con tale alfabeto riusciremo a rappresentare tutti i numeri
naturali. Notiamo infatti che \`e del tutto equivalente considerare un
alfabeto con un simbolo o con n simboli ed i calcolatori, essendo
rappresentazioni di Macchine di Turing ed usando l'alfabeto binario lo
confermano.

Introduciamo ora una notazione che ci permetter\`a di rappresentare i
numeri naturali e quindi gli input da dare alle macchine (compreso lo
0). Per come abbiamo definito la macchina di Turing, risulta che
abbiamo a disposizione l'unico simbolo di input $1$ per rappresentare
ogni naturale o n-upla di naturali che siano nel dominio di una
funzione.

\begin{itemize}
\item preso un numero $n\in {N}$, ad esso associamo un'espressione
$$\overline{n}=1^{n+1}=\underbrace{11...1}_{n+1\ volte}\quad\mbox{.}$$
  Ad esempio $\overline{3}=1111$.
\item in maniera del tutto analoga, a una k-tupla $(n_{1}...n_{k})$ di
  interi non negativi associamo l'espressione
  $\overline{(n_{1}...n_{k})}=\overline{n_{1}}B...B\overline{n_{k}}$. Per
  esempio $$\overline{(1,4,0)}=\overline{1}B\overline{4}B\overline{0}=
  11B11111B1\quad\mbox {.}$$ In questo caso si vede come si procede
  per separare gli input, cio\`e inserendo un $B$.
\end{itemize}

Se invece con questa notazione abbiamo rappresentato nel nastro un
determinato numero indichiamo con $<\overline{n}>$ il numero naturale
da esso rappresentato, ossia $<\overline{n}>=n$.\\ Quindi gli input
che compariranno nella sequenza d'ingresso sono della forma
$$...BB\overline{n_{1}}B\overline{n_{2}}B...B\overline{n_{k}}BB...$$
dove $\overline{n_1},...,\overline{n_k}$ sono i numeri naturali nella
notazione data sopra. Converremo che ogni blocco rappresentante un
argomento \`e separato dagli altri da un solo $B$; in questo modo la
macchina sa dove inizia e termina l'input (se si incontrano due blank
consecutivi, la testina non \`e pi\`u collocata su una delle celle di
input).  La macchina di Turing su un dato input produce un output. Noi
accetteremo output della forma $$...BB\overline{n_r}BB...$$ ove
$\overline{n_r}$ \`e il risultato della computazione. Dunque in output
avremo il nastro contenente un blocco unico contenente tanti $1$
quanti sono necessari per rappresentare il risultato, in modo tale che
anche in questo caso la macchina sa dove comincia e finisce l'output,
e quindi si noti che, una volta calcolata la soluzione, sar\`a
necessario anche cancellare gli input.

\section{Varianti della Macchina di Turing}
Ci sono molteplici varianti della macchina di Turing e seppur non
entrando qui nel dettaglio si pu\`o mostrare che queste hanno la
stessa portata computazionale della macchina di Turing che abbiamo
definito all'inizio.
\begin{itemize}
\item Macchina di Turing Multinastro: invece di esservi un singolo
  nastro, vi sono k nastri indipendenti e k testine di
  lettura-scrittura, una per ogni nastro.
\item Macchina di Turing con movimento arbitrario della testina di
  lettura: si pu\`o modificare la definizione di una macchina di
  Turing in modo che la testina di lettura si possa spostare ogni
  volta di un numero arbitrario di celle
\item Macchina di Turing con nastro bidimensionale: la macchina opera
  su un nastro bidimensionale, e sono permessi spostamenti verso
  sinistra, destra, l'alto, il basso.
\item Macchina di Turing non deterministica: si distingue da quella
  deterministica definita in precedenza per il fatto che, in presenza
  di un determinato stato e di un determinato carattere letto, essa
  permette la presenza di pi\`u quintuple che iniziano con quello
  stato e quel carattere letto.
\end{itemize}


\chapter{Macchine di Turing e Funzioni Calcolabili}

\section{Cosa pu\`o Essere Calcolato}

Le macchine di Turing sono strumenti molto potenti. Per una grande
quantit\`a di problemi di calcolabilit\`a, \`e possibile costruire una
macchina di Turing che sia in grado di eseguire il calcolo
desiderato. Per esempio, abbiamo visto che \`e possibile ideare una
macchina di Turing per eseguire le operazioni aritmetiche di base sui
numeri naturali.

\subsection{Numeri Calcolabili}

Il lavoro originale di Turing trattava i numeri calcolabili. Un numero
si dice Turing-calcolabile se esiste una macchina di Turing che inizia
la computazione su un nastro vuoto e calcola un numero arbitrario, con
una certa approssimazione. Le macchine di Turing possono calcolare
tutti i numeri algebrici (radici di equazioni con coefficienti
algebrici) e molte costanti matematiche, come ad esempio $\textit{e}$
(costante di Nepero) e $\pi$.

\subsection{Funzioni Calcolabili}

Come abbiamo visto in precedenza, le macchine di Turing possono fare
molto pi\`u che calcolare semplici numeri. Tra le altre cose possono
calcolare funzioni numeriche del tipo $f:N^{n}->N$, come ad esempio si
possono scrivere macchine per calcolare la somma, la moltiplicazione,
la sottrazione tra naturali, l'esponenziale, il fattoriale e molte
altre.

Usando una convenzione per rappresentare i simboli di verit\`a TRUE e
FALSE, rappresentando, per esempio, il valore TRUE come una sequenza
di due ``1'' e il valore FALSE come un singolo ``1'', si possono
trattare anche funzioni binarie, ovvero del tipo
$f:N^{n}->\{0,1\}$. Con tali funzioni \`e possibile calcolare la
funzione caratteristica di un predicato. E', inoltre, possibile
combinare i risultati di tali funzioni con gli usuali operatori
booleani: AND, NOT, OR e in aggiunta IF-THEN-ELSE.

Difatti, le funzioni Turing-calcolabili coincidono con le funzioni
ricorsive, di cui tratteremo pi\`u avanti.

\subsection{Macchina di Turing Universale}

Il pi\`u importante e suggestivo risultato del lavoro di Turing
riguarda il fatto che, per come sono state definite le macchine di
Turing, \`e possibile definire la \textit{Macchina di Turing
  Universale} (UTM).

In teoria della computazione si dice macchina di Turing Universale
$T_U$ una macchina di Turing capace di simulare le evoluzioni di ogni
macchina di Turing. Tale macchina ha permesso di dare una risposta
negativa al problema della decidibilit\`a posto da David Hilbert nel
1928. Brevemente, il problema posto da Hilbert riguarda la
possibilit\`a o meno di poter definire un metodo matematico o un
processo per decidere tutti i possibili problemi matematici.

Essenzialmente la UTM \`e in grado di simulare il comportamento di
qualunque macchina di Turing (compresa se stessa!). Un modo per
pensare alla UTM potrebbe essere quello di immaginarla come un
interprete di un linguaggio di programmazione. Quando all'interprete
(concretamente \`e esso stesso un programma) viene dato in input un
programma, questo comincia ad eseguire le istruzioni del programma
dato e, quindi, si comporta proprio come se fosse lo stesso programma,
dato in input, ad essere eseguito.

La macchina di Turing Universale richiede in input una coppia di
numeri naturali (n,m), dove il primo \`e un indice che identifica una
specifica macchina di Turing $T_n$ (in grado di computare una data
funzione f). Mentre, il secondo indice rappresenta l'input sul quale
eseguire la macchina $T_n$.

Per poter definire una macchina del tipo descritto, \`e prima
necessario ideare un modo per rappresentare una macchina di Turing sul
nastro della UTM. E' possibile associare in modo effettivo ad ogni
macchina di Turing un numero naturale che la identifichi
univocamente. L'idea \`e quella di avere una funzione, preferibilmente
biunivoca, che mappi l'insieme delle macchine di Turing, diciamo T,
nell'insieme dei numeri naturali $f:T->N$. A tale scopo, tenendo
presente che le macchine di Turnig sono definite da una quintupla, si
pu\`o prima definire una codifica per i singoli elementi di una
quintupla e in seguito per ogni possibile quintupla. Esistono molti
modi possibili per ottenere questa codifica effettiva. Il paragrafo
\ref{codeffettiva} che segue mostra a titolo di esempio una di queste
possibili codifiche.

L'esecuzione della macchina universale $T_U(n,m)$ corrisponde quindi
ad eseguire la macchina $T_n(m)$.

\begin{center}
$T_U(n,m) = T_n(m)$
\end{center}

\vspace{10pt}

La macchina $T_U$ opera con un nastro che inizialmente in una sua
prima sezione contiene la codifica delle istruzioni della macchina
$T_n$ seguita, dopo un demarcatore particolare bene individuabile,
dallo stesso input m.

E' interesante osservare che la macchina di Turing Universale \`e in
grado di simulare anche la propria evoluzione mentre procede a
simulare una qualsiasi macchina $T$.

I computer programmabili sono effettivamente delle macchine di Turing
universali, poich\`e sono in grado di simulare l'esecuzione di ogni
macchina di Turing (se vengono intese come algoritmi che calcolano
funzioni) tramite l'esecuzione di un programma, che pu\`o essere visto
come un processo di calcolo da eseguire.

\subsection{Codifica Effettiva delle Macchine di Turing}\label{codeffettiva}

Di seguito viene riportato un'esempio di codifica effettiva per le
macchine di Turing. Con tale codifica \`e possibile rappresentare una
qualunque macchina di Turing sul nastro della macchina di Turing
universale.\\

Ogni quintupla nella definizione di una macchina sar\`a codificata
come una sequenza di cinque blocchi di ``1'', separati da un singolo
``0''.
\begin{enumerate}
  \item Il primo blocco di ``1'' sar\`a la codifica del numero dello
    stato corrente, usando la convezione sopra definita per la
    rappresentazione dei numeri naturali sul nastro (n+1 ``1''
    rappresentano il numero n).
  \item Il secondo blocco di ``1'' sar\`a la codifica del simbolo
    corrente (simbolo che si trova sotto la testina). La codifica del
    simbolo corrente dipende dall'alfabeto e dalla notazione per i
    caratteri usata.``1''
  \item Il terzo blocco di ``1'' sar\`a la codifica dell'eventuale
    simbolo da scrivere in posizione sul nastro sottostante la
    testina.
  \item Il quarto elemento rappresenter\`a l'azione che deve essere
    eseguita dalla testina, e ci sono tre possibilit\`a: con un blocco
    di un solo ``1'' si rappresanta lo spostamento nullo, con un
    blocco di due ``1'' lo spostamento a sinistra (L) e con tre ``1''
    lo spostamento a destra (R).
  \item Infine, il quinto elemento della tuple rappresenter\`a il
    numero del nuovo stato in cui la macchina si trover\`a dopo
    l'esecuzione dell'azione richiesta, rappresentato anche esso
    mediante la notazione unaria.
\end{enumerate}

Per codicifare, in fine, l'intera macchina, basta semplicemente
scrivere sul nastro tutte le tuple che caratterizzano la macchina,
separate tra loro da un carattere di \textit{Blank}.

Se si utilizza il carattere ``0'' al posto del carattere
\textit{Blank}, si ottiene una sequenza di ``0'' e di ``1''. Tale
sequenza pu\`o essere interpretata come un numero naturale espresso in
rappresetntazione binaria. Se tale numero viene espresso con la
notazione $\overline{n}=1^{n+1}$, risulta effettivamente possibile
dare in input alla macchina di Turing universale una coppia di numeri
naturali espressi secondo questa notazione.


\section{Cosa non pu\`o Essere Calcolato}
Un altro sorprendente risultato presentato dall'articolo di Turing del
1936 riguarda l'esistenza di problemi ben definiti che non possono
essere risolti da nessuna procedura computazionale. Se questi problemi
sono formulati come formule, possiamo riferirci a queste funzioni come
funzioni non calcolabili; se questi problemi vengono espressi come
predicati, si dice problemi indecidibili. Usando il concetto della
macchina astratta di turing, possiamo dire che una funzione non \`e
calcolabile se non esistono macchine di Turing che sono in grado di
calcolare tale funzione.

Per mostrare l'esitenza di funzioni non calcolabili dalle macchine di
Turing bisogna ragionare sulla cardinalit\`a dell'insieme delle
macchine di Turing e confrontarla con la cardinalit\`a dell'insieme
delle funzioni del tipo $f:N->N$ (con uno sforzo minimo ci si pu\`o
portare al caso pi\`u generale di funzioni $f:N^n->N$).

\begin{center}
$F = \{ f | f \textrm{ \`e una funzione del tipo } f:N->N \}.$
\end{center}

\vspace{10pt}

\begin{center}
$T = \{ t | t \textrm{ \`e una macchina di Turing} \}.$
\end{center}

\vspace{10pt}

Poich\`e \`e possibile, come mostrato in precedenza, costruire una
codifica effettiva delle macchine di Turing, cio\`e costruire una
funzione che associ ad ogni TM un numero naturale unico, l'insieme
delle TM \`e enumerabile, ovvero la sua cardinalit\`a \`e pari a $|N|$
(cardinalit\`a dei numeri naturali). Tuttavia, l'insieme di tutte le
funzioni $f:N->N$ \`e dimostrato non essere enumerabile. Per questo
motivo, possiamo concludere che tutte le macchine di Turing sono in
grado di calcolare solo un sottoinsieme di tutte le funzioni
definibili.

Ci sono molti problemi legati ai programmi che si possono dimostrare
non decidibili. Poich\`e le macchine di Turing sono molto semplici se
comparate con i computer come uso pratico, risulta concettualmente
pi\`u facile provare l'impossobilit\`a di alcuni risultati per le
macchine di Turing. Dando un definizione formale di non banale per una
propriet\`a, H.G. Rice riusc\`i, addirittura, a dimostrare il seguente
teorema:

\begin{thm}
Ogni propriet\`a non banale dei programmi \`e indecidibile.
\end{thm}

E' da notare che dimostrando che un problema \`e non decidibile, non
significa che non pu\`o essere deciso solo in alcuni casi, bens\`i che
\`e impossibile costruire una soluzione \textit{per ogni} possibile
input.



%%%%%%%%%%%%%%%%%%%%%%%%%%%%%%%%%%%%%%%%%%%%%%%%%ESEMPI%%%%%%%%%%%%%%%%%%%%%%%%%%%%%%%%%

\chapter{Macchine di Turing: Esempi}


\section{Esempi di Macchine di Turing}
Diamo ora qualche esempio di semplici macchine di Turing.\\

\begin{itemize}
\item Sia $\Sigma=\left\{0,1\right\}$ l'alfabeto di simboli in
  input. \\\underline{Scopo della macchina:} data una stringa di 0 e 1
  scambia le occorrenze di 1 con 0 e viceversa, riposizionandosi sulla
  cella precedente a quella iniziale.\\ \underline{Passi:} leggere i
  simboli sul nastro e ogni volta che si trova 1 si scrive 0 e
  viceversa. Quando la sequenza di 0 e 1 è finita la testina torna
  sulla cella precedente alla prima significativa.\\

Nastro in input:$$...BB100101BB....$$\\

Sequenza di quintuple:
\begin{eqnarray*}
&<q_0,0,1,R,q_0>&\\
&<q_0,1,0,R,q_0>&\mbox{(scambia 0 e 1)}\\
&<q_0,B,B,L,q_1>&\\
&<q_1,0,0,L,q_1>&\\
&<q_1,1,1,L,q_1>&\mbox{(si riposiziona sulla cella precedente alla prima significativa)}\\
\end{eqnarray*}
Quindi la macchina di Turing per la sequenza di ingresso $...BB100101BB....$ esegue la seguente computazione:
\begin{eqnarray*}
&...BBq_0100101BB...\\
&...BB0q_000101BB...\\
&...BB01q_00101BB...\\
&...BB011q_0101BB...\\
&...BB0110q_001BB...\\
&...BB01101q_01BB...\\
&...BB011010q_0BB...\\
&...BB01101q_10BB...\\
&...\\
&...Bq_1B011010BB...\\
\end{eqnarray*}
Notiamo che l'ultima descrizione istantanea \`e tale da essere
terminale. Infatti arrivata a questo punto la macchina si arresta in
quanto nell'insieme di quintuple che regolano il suo comportamento non
c'\`e una quintupla del tipo $<q_1,B....>$.\\
\vspace{0.3 cm} Nastro in output:$$...BB011010BB....$$ Quindi dato
l'output vediamo che tale macchina sostituisce, nella sequenza data in
input, le occorrenze di 1 con 0 e viceversa.\\
\vspace{0.2cm}
Diagramma:
\begin{figure}[hbtp!]
\hspace{3 cm} \entrymodifiers={++[o][F-]} \xymatrix@=40pt{q_{0}
  \ar@(dr,dl)[]^{1|0 \ R} \ar@(ul,ur)[]^{1|0 \ R} \ar@<2pt>[r]^{B|B
    \ L} & q_{1}\ar@(dr,dl) []^{0|0 \ L} \ar@(ul,ur)[]^{1|1 \ L}\\}
\end{figure}

\item Sia $\Sigma=\left\{1,B\right\}$ l'alfabeto di simboli in
  input. \\\underline{Scopo della macchina:} calcolare il successore
  di un numero naturale dato in input ovvero data una sequenza di
  $n+1$ uni, che rappresenta il numero naturale n, aggiunge alla fine
  della sequenza un altro uno($f(x)=x+1$).\\\underline{Passi:} Essendo
  la testina posizionata nella cella con il primo 1 della sequenza si
  sposta fino al primo B a destra della sequenza. Scrive 1 e torna
  alla posizione di partenza.\\ Nastro in input: $x=2$ rappresentato
  da 3 uni $$BBB111BBB$$ Sequenza di quintuple:
\begin{eqnarray*}
&<q_0,1,1,R,q_0>&\mbox{(scorre tutti gli uni della sequenza)}\\
&<q_0,B,1,L,q_1>&\mbox{(sostituisce il primo B trovato a destra della sequenza con un 1)}\\
&<q_1,1,1,L,q_1>&\\
&<q_1,B,B,R,q_{2}>&\mbox{(si riposiziona sulla prima cella significativa del nastro)}\\
\end{eqnarray*}
Quindi la macchina di Turing per la sequenza di ingresso
$...BB111BB....$ esegue la seguente computazione:
\begin{eqnarray*}
&...BBq_0111BB...\\
&...BB1q_011BB...\\
&...BB11q_01BB...\\
&...BB111q_0BB...\\
&...BB11q_011BB...\\
&...\\
&...Bq_0B1111BB\\
\end{eqnarray*}
Nastro in output: $f(2)=3$ rappresentato da 4 uni $$...BB1111BB....$$
Diagramma:
\begin{figure}[hbtp!]
\hspace{3 cm} \entrymodifiers={++[o][F-]} \xymatrix@=40pt{q_{0}
  \ar@(dr,dl)[]^{1|1 \ R} \ar@<2pt>[r]^{B|1 \ L} & q_{1}\ar@(dr,dl)
      []^{1|1 \ L} \ar@<2pt>[r]^{B|B \ R} & q_{2}}\\
\end{figure}
\begin{oss}
Per calcolare il successore, invece di aggiungere un uno alla fine
della sequenza di uni data in input, si poteva aggiungerlo nella cella
precedente alla prima cella significativa del nastro. In tal caso le
quintuple si sarebbero ridotte a due e precisamente:
\begin{eqnarray*}
&<q_0,1,1,L,q_1>&\mbox{(scorre tutti gli uni della sequenza)}\\
&<q_1,B,1,R,q_2>&\mbox{(sostituisce il primo B trovato a destra della sequenza con un 1)}\\
\end{eqnarray*}
\end{oss}
\item Sia $\Sigma=\left\{1,B\right\}$ l'alfabeto di simboli in
  input.\\\underline{Scopo della macchina:} data in input una sequenza
  (anche di pi\`u input basta siano separati al pi\`u da un B) sposta
  tutta la parte significativa del nastro di una cella verso
  destra.\\ \underline{Passi:} cancellare il primo 1 che si trova, poi
  scorrere la parte significativa del nastro fino al primo B e
  sostituirlo con un 1. Se c'\`e un altro input procede
  analogamente.\\ Nastro in input:$$...BB11B1BB....$$ Sequenza di
  quintuple:
\begin{eqnarray*}
&<q_0,1,B,R,q_1>&\mbox{(cancella il primo 1)}\\
&<q_1,1,1,R,q_1>&\mbox{(scorre la sequenza di 1)}\\
&<q_1,B,1,R,q_2>&\mbox{(sostituisce il primo B con un 1, cos\`i sposta il primo blocco)}\\
&<q_2,1,B,R,q_1>&\mbox{(cancella il primo 1 del secondo input e procede come prima)}\\
&<q_2,B,B,L,q_3>&\\
&<q_3,1,1,L,q_3>&\\
&<q_3,B,B,L,q_4>&\\
&<q_4,1,1,L,q_3>&\\
&<q_4,B,B,R,q_5>&\\
&<q_5,B,B,R,q_6>&\mbox{(si riposiziona sulla prima cella significativa del nastro)}\\
\end{eqnarray*}
Quindi la macchina di Turing per la sequenza di ingresso $...BB11B1BB....$ esegue la seguente computazione:
\begin{eqnarray*}
&...BBq_011B1BB...\\
&...BBBq_11B1BB...\\
&...BBB1q_1B1BB...\\
&...BBB11q_21BB...\\
&...BBB11Bq_1BB...\\
&...BBB11B1q_2BB...\\
&...BBB11Bq_31BB...\\
&...BB011q_3B1BB...\\
&...BBB1q_41B1BB...\\
&...BBBq_311B1BB...\\
&...BBq_3B11B1BB...\\
&...Bq_4BB11B1BB...\\
&...BBq_5B11B1BB...\\
&...BBBq_611B1BB...\\
\end{eqnarray*}
Nastro in output: tutta la sequenza \`e stata spostata di una cella
verso destra $$...BBB11B1BB....$$ Diagramma:
\begin{figure}[hbtp!]
\hspace{3 cm} \entrymodifiers={++[o][F-]} \xymatrix@=40pt{q_{0}
  \ar[d]^{1|B \ R}\\ q_{1}\ar@(dr,dl) []^{1|1 \ R} \ar@<2pt>[r]^{B|1
    \ R} & q_{2}\ar@(d,dr)[l]^{1|B \ R} \ar@<2pt>[r]^{B|B \ L} &
  q_{3}\ar@(dr,dl) []^{1|1 \ L}\ar@<2pt>[r]^{B|B \ L } &
  q_{4}\ar@(d,dr)[l]^{1|1 \ L} \ar@<2pt>[r]^{B|B \ R} & q_{5}
  \ar@<2pt>[r]^{B|B \ R} & q_{6}}\\
\end{figure}

\end{itemize}

\section{Alcune operazioni aritmetiche}
Mostriamo ora alcune macchine di Turing pi\`u complesse, in grado di
calcolare alcune operzioni aritmetiche.\\

\begin{esempio}[Sottrazione]
Vogliamo calcolare la differenza di due numeri naturali non
negativi: $$g(x,\ y)=x-y.$$ L'idea \`e quella di cancellare di volta
in volta un 1 all'inizio e un 1 alla fine della sequenza in input
finch\`e uno dei due blocchi non si svuota e aggiunge un 1 per avere
l'output nella forma standard.\newline Notare che nei naturali $x-y$,
con $x<y$, non \`e definita e per questo la nostra macchina di Turing
per la sottrazione potrebbe andare in loop, ossia non fermarsi
mai. Questo \`e esattamente un esempio di funzione semicalcolabile.\\

Definizione della macchina:
\begin{eqnarray*}
&<q_{0},1,B,R,q_{1}>&\mbox{(cancella l'1 pi\`u a sinistra)}\\
&<q_{1},1,1,R,q_{1}>&\\
&<q_{1},B,B,R,q_{2}>&\mbox{(trova il blank che separa i blocchi)}\\
&<q_{2},1,1,R,q_{2}>&\\
&<q_{2},B,B,L,q_{3}>&\mbox{(trova l'estremit\`a destra)}\\
&<q_{3},1,B,L,q_{4}>&\mbox{(cancella l'1 pi\`u a destra)}\\
&<q_{4},1,1,L,q_{5}>&\\
&<q_{4},B,1,L,q_{9}>&\mbox{(se $y$ \`e stato consumato torno sulla prima cella altrimenti continua)}\\
&<q_{5},1,1,L,q_{5}>&\\
&<q_{5},B,B,L,q_{6}>&\mbox{(trova il blank che separa i blocchi)}\\
&<q_{6},1,1,L,q_{7}>&\\
&<q_{6},B,B,R,q_{8}>&\mbox{(se $x$ \`e stato consumato torna all'inizio senn\`o continua)}\\
&<q_{7},1,1,L,q_{7}>&\\
&<q_{7},B,B,R,q_{0}>&\mbox{(trova l'estremit\`a sinistra e torna in $q_{0}$)}\\
&<q_{8},1,1,L,q_{8}>&\\
&<q_{8},B,B,R,q_{8}>&\mbox{(computa un ciclo infinito se $x< y$)}\\
&<q_{9},1,1,L,q_{9}>&\\
&<q_{9},B,B,R,q_{10}>&\mbox{(torno sulla prima cella significativa del nastro)}\\
\end{eqnarray*}

Grafico:
\begin{figure}[hbtp]
\resizebox{\columnwidth}{!}{%
\hspace{0cm} \xymatrix@=40pt{\roundentry{q_{0}} \ar[r]^{1|BR} &
  \roundentry{q_{1}} \ar@(dr,dl)[]^{1|1R} \ar[r]^{B|BR} &
  \roundentry{q_{2}} \ar@(dr,dl)[]^{1|1R} \ar[r]^{B|BL} &
  \roundentry{q_{3}} \ar@(dr,dl)[]^{1|BL} \ar[r]^{1|BL} &
  \roundentry{q_{4}} \ar[r]^{B|1L} \ar[d]^{1|1L} & \roundentry{q_{9}}
  \ar@(dr,dl)[]^{1|1L} \ar[r]^{B|BR} & \roundentry{q_{10}} \\ &&&&
  \roundentry{q_{5}} \ar@(dr,dl)[]^{1|1L} \ar[r]^{B|BL} &
  \roundentry{q_{6}} \ar[r]^{B|BR} \ar[d]^{1|1L} & \roundentry{q_{8}}
  \ar@(dr,dl)[]^{1|1L} \ar@(ru,rd)[]^{B|BR} \\ &&&&&
  \roundentry{q_{7}} \ar@(dr,dl)[]^{1|1L} \ar `l[lllll] ^{B|BR}
             [llllluu] }}
\end{figure}

Diamo ora qualche esempio di computazione e calcoliamo
quindi $$\overline{m_{1}}-\overline{m_{2}}\mbox{,}\quad
\overline{m_{1}}\mbox{,} \overline{m_{2}}\in \mbox{\upshape N.}$$ La
configurazione iniziale con cui la macchina parte \`e:
$$...BBq_{0}1^{m_{1}+1}B1^{m_{2}+1}BB...$$

Ora la testina si trova in $q_{0}$ e legge un 1, quindi deve cancellarlo. Dunque abbiamo:
$m_{1}\geq m_{2}$. Chiamiamo $k=m_{1} - m_{2}$.\\
\begin{eqnarray*}
&...BBBq_{1}1^{m_{1}}B1^{m_{2}+1}BB...&\mbox{(ho cancellato il primo 1)}\\
&...BBB1q_{1}1^{m_{1}-1}B1^{m_{2}+1}BB...&\\
&.........&\\
&...BBB1^{m_{1}}q_{1}B1^{m_{2}+1}BB...&\\
&...BBB1^{m_{1}}Bq_{2}1^{m_{2}+1}BB...&\mbox{(ho trovato la B che separa i blocchi)}\\
&...BBB1^{m_{1}}B1q_{2}1^{m_{2}}BB...&\\
&.........&\\
&...BBB1^{m_{1}}B1^{m_{2}+1}q_{2}BB...&\\
\end{eqnarray*}
\begin{eqnarray*}
&...BBB1^{m_{1}}B1^{m_{2}}q_{3}1BB...&\mbox{(ho trovato l'ultimo 1)}\\
&...BBB1^{m_{1}}B1^{m_{2}}q_{4}BBB...&\\
&...BBB1^{m_{1}}B1^{m_{2}-1}q_{4}1BBB...&\mbox{(ho cancellato l'ultimo 1)}\\
&...BBB1^{m_{1}}B1^{m_{2}-2}q_{5}11BBB...&\\
&...BBB1^{m_{1}}B1^{m_{2}-3}q_{5}111BBB...&\\
&.........&\\
&...BBB1^{m_{1}}q_{5}B1^{m_{2}}BBB...&\\
&...BBB1^{m_{1}-1}q_{6}1B1^{m_{2}}BBB...&\mbox{(ho trovato la B tra i due blocchi)}\\
&...BBB1^{m_{1}-2}q_{7}11B1^{m_{2}}BBB...&\\
&...BBB1^{m_{1}-3}q_{7}111B1^{m_{2}}BBB...&\\
&.........&\\
&...BBq_{7}B1^{m_{1}}B1^{m_{2}}BBB...&\\
&...BBBq_{0}1^{m_{1}}B1^{m_{2}}BBB...&\mbox{(sono tornato all'inizio e ricomincio)}\\
&.........\qquad\qquad\qquad&[*]\\
&...BBB^{m_{2}}q_{0}11^{k}B1B^{m_{2}}BB...&\\
&...BBB^{m_{2}}q_{0}B1^{k}B1B^{m_{2}}BB...&\\
&...BBB^{m_{2}+1}q_{1}1^{k}B1B^{m_{2}}BB...&\\
&...BBB^{m_{2}+1}1q_{1}1^{k-1}B1B^{m_{2}}BB...&\\
&.........&\\
&...BBB^{m_{2}+1}1^{k}q_{1}B1B^{m_{2}}BB...&\\
&...BBB^{m_{2}+1}1^{k}Bq_{2}1B^{m_{2}}BB...&\\
&...BBB^{m_{2}+1}1^{k}B1q_{2}B^{m_{2}}BB...&\\
&...BBB^{m_{2}+1}1^{k}Bq_{3}1B^{m_{2}}BB...&\\
&...BBB^{m_{2}+1}1^{k}Bq_{3}BB^{m_{2}}BB...&\mbox{($m_{2}$ \`e stato consumato)}\\
&...BBB^{m_{2}+1}1^{k}q_{4}BBB^{m_{2}}BB...&\\
&...BBB^{m_{2}+1}1^{k}q_{9}1BB^{m_{2}}BB...&\\
&...BBB^{m_{2}+1}1^{k-1}q_{9}11BB^{m_{2}}BB...&\\
&.........&\\
&...BBB^{m_{2}}q_{9}B1^{k+1}BB^{m_{2}}BB...&\\
&...BBB^{m_{2}+1}q_{10}1^{k+1}BB^{m_{2}}BB...&\mbox{(sono tornato all'inizio)}\\
\end{eqnarray*}

Il risultato della computazione \`e esattamente $k$.\\

($m_{1}<m_{2}$). Chiamiamo $k=m_{2}-m_{1}$. Partendo dalla stessa
configurazione iniziale del caso precedente, ottengo una serie di
descrizioni istantanee come sopra, fino a [*]. Riprendiamo la nostra
computazione proprio da questo punto. Arriveremo a:\\
\begin{eqnarray*}
&...BBB^{m_{1}}q_{0}1B1^{k}1B^{m_{1}}BB...&\\
&...BBB^{m_{1}+1}q_{1}B1^{k}1B^{m_{1}}BB...&\\
&...BBB^{m_{1}+2}q_{2}1^{k}1B^{m_{1}}BB...&\\
&...BBB^{m_{1}+2}1q_{2}1^{k-1}1B^{m_{1}}BB...&\\
&.........&\\
&...BBB^{m_{1}+2}1^{k}1q_{2}B^{m_{1}}BB...&\\
&...BBB^{m_{1}+2}1^{k}q_{3}1B^{m_{1}}BB...&\\
&...BBB^{m_{1}+2}1^{k-1}q_{4}1B^{m_{1}+1}BB...&\\
&...BBB^{m_{1}+2}1^{k-2}q_{5}11B^{m_{1}+1}BB...&\\
&...BBB^{m_{1}+2}1^{k-3}q_{5}111B^{m_{1}+1}BB...&\\
&.........&\\
&...BBB^{m_{1}+1}q_{5}B1^{k}B^{m_{1}+1}BB...&\mbox{($m_{1}$ \`e stato svuotato!)}\\
&...BBB^{m_{1}}q_{6}BB1^{k}B^{m_{1}+1}BB...&\\
&...BBB^{m_{1}+1}q_{8}B1^{k}B^{m_{1}+1}BB...& \qquad[\dag]\\
&...BBB^{m_{1}+2}q_{8}1^{k}B^{m_{1}+1}BB...&\\
&...BBB^{m_{1}+1}q_{8}B1^{k}B^{m_{1}+1}BB...& \qquad[\dag]\\
\end{eqnarray*}

Notare che le due descrizione istantanee contrassegnate con $[\dag]$
sono le stesse, questo significa che siamo entrati in un ciclo
infinito. Questo dipende dal fatto che nei naturali, la sottrazione
tra due numeri $x$ e $y$, con $x<y$, non \`e definita.\\
\end{esempio}


\begin{esempio}[Sottrazione Adeguata]
Possiamo evitare il problema del ciclo infinito nell'esempio
precedente definendo una \textsl{sottrazione adeguata}:\\
\begin{center}
$g(x,y)=x\dot{-} y=
\left\{ \begin{array}{ll}
x-y &  x \geq y\\
0 & \textrm{altrimenti}
\end{array} \right.$
\end{center}
Una macchina di Turing che computi tale funzione \`e molto simile a
quella dell'esempio precedente: la definizione della macchina a
partire da tutti gli stati diversi da $q_{8}$ \`e la stessa e abbiamo
in aggiunta le seguenti mosse e stati:
\begin{eqnarray*}
&<q_{8},1,1,L,q_{8}>&\\
&<q_{8},B,B,R,q_{11}>&\\
&<q_{11},1,B,R,q_{8}>&\\
&<q_{11},B,1,R,q_{12}>&\\
\end{eqnarray*}
Il grafico \`e come quello per lo stato $q_{8}$:
\begin{figure}[hbtp]
\hspace{0cm} \entrymodifiers={++[o][F-]} \xymatrix@=40pt{ q_{6}
  \ar[r]^{B|BR} & q_{8} \ar@(dr,dl)[]^{1|1L} \ar@<2pt>[r]^{B|BR} &
  q_{11} \ar@(d,dr)[l]^{B|BR} \ar[r]^{B|1-} & q_{12}}\end{figure}\par
In questa maniera, se $x\geq y$ la macchina funziona come nell'esempio
precedente, altrimenti cancella tutto tranne un $1$ (ricordare che
$<1>=<\overline{0}>=0$).\\

E' da notare che non \`e possibile, in generale, modificare ogni
funzione parziale in modo tale da permetterle di esecuire gli stessi
calcoli e allo stesso tempo renderla totale, cos\`i da evitare
potenziali esecuzioni infinite. In questo caso, infatti, abbiamo
potuto modificare la funzione in modo tale da farle ritornare un
valore particolare nel caso in cui la funzione originale non sia
definita sull'input dato. In questo modo la macchina di Turing che la
calcola termina sempre su qualunque input, ma questo tipo di modifica
in generale non \`e sempre possibile.\\
\end{esempio}

\begin{esempio}[Prodotto]
Calcoliamo ora il prodotto di due numeri\\ \underline{naturali non
  negativi}: $g(m,n) = m * n$\\

L'idea in questo caso \`e quella di sommare $m$ a s\`e stesso $n$
volte; per fare questo necessario usare una cella del nastro come
segnaposto per capire a che punto della moltiplicazione si \`e
arrivati. Ovviamente ci sono dei casi particolari in cui uno dei due
numeri o eventualmente entrambi sono nulli ed essi vanno analizzati
singolarmente.\\

In questo esempio una volta arrivati alla conclusione della
computazione, invece di tornare all'inizio del nastro, si fa andare la
macchina allo stato $q_{42}$, per cui si suppone che non ci siano
istruzioni.\\

Definizione della macchina:\\
\begin{eqnarray*}
&<q_{0},1,B,R,q_{1}>&\mbox{(cancello il primo ``1'')}\\
&<q_{1},B,B,R,q_{2}>&\mbox{(caso in cui ho uno zero nel primo input)}\\
&<q_{2},1,B,R,q_{2}>&\\
&<q_{2},B,1,L,q_{42}>&\mbox{(il secondo input viene cancellato e quando si arriva alla fine }\\
&&\mbox{si trasporma il primo ``B'' che si trova in ``1'' per indicare lo ``0'')}\\
&<q_{1},1,B,R,q_{3}>&\mbox{(cancello il secondo ``1'' del primo input)}\\
&<q_{3},1,1,R,q_{3}>&\\
&<q_{3},B,B,R,q_{4}>&\mbox{(scorro tutto il primo input)}\\
&<q_{4},1,1,R,q_{5}>&\mbox{(sono arrivato all'inizio del secondo input quindi}\\
&&\mbox{ trovo sicuramente un ``1'')}\\
&<q_{5},B,B,L,q_{6}>&\mbox{(caso in cui il secondo input \`e zero)}\\
&<q_{6},1,B,L,q_{6}>&\\
&<q_{6},B,B,L,q_{7}>&\\
&<q_{7},1,B,L,q_{7}>&\\
&<q_{7},B,1,L,q_{42}>&\mbox{(anche qui l'``1'' che si aggiunge alla fine \`e per indicare lo zero)}\\
&<q_{5},1,B,R,q_{8}>&\mbox{(caso in cui entrambi gli input sono diversi da zero; l'``1'' che}\\
&&\mbox{ si cancella in questo momento \`e usato come segnaposto)}\\
&<q_{8},1,1,R,q_{8}>&\\
\end{eqnarray*}
\begin{eqnarray*}
&<q_{8},B,B,R,q_{9}>&\mbox{(scorro il secondo input)}\\
&<q_{9},1,1,R,q_{9}>&\mbox{(scorro il risultato)}\\
&<q_{9},B,1,L,q_{10}>&\\
&<q_{10},1,1,L,q_{10}>&\\
&<q_{10},B,B,L,q_{11}>&\mbox{(aggiungo un ``1'' alla fine e poi torno indietro fino a quando arrivo }\\
&&\mbox{al ``B'' che separa il secondo input dal risultato)}\\
&<q_{11},B,1,L,q_{12}>&\mbox{(caso in cui sono gi\`a arrivato alla fine del secondo input)}\\
&<q_{12},1,1,L,q_{12}>&\\
&<q_{12},B,B,L,q_{13}>&\\
&<q_{13},1,1,L,q_{14}>&\mbox{(se sono arrivato alla fine del secondo input tolgo il segnaposto e}\\
&&\mbox{ torno all'inizio del primo input)}\\
&<q_{13},B,B,R,q_{1}>&\mbox{(caso in cui sono ho gi\`a finito)}\\
&<q_{14},1,1,L,q_{14}>&\\
&<q_{14},B,B,R,q_{1}>&\\
&<q_{11},1,1,L,q_{15}>&\\
&<q_{15},1,1,L,q_{15}>&\\
&<q_{15},B,1,R,q_{16}>&\mbox{(ho trovato il segnaposto)}\\
&<q_{16},1,B,R,q_{8}>&\mbox{(aggiorno il segnaposto)}\\
&<q_{16},B,B,R,q_{9}>&\mbox{(quando arrivo alla fine del secondo input) }\\
\end{eqnarray*}
\end{esempio}


\chapter{Macchine a Registri}


\section{Premessa}
Lo studio delle macchine di Turing ci ha fatto capire che il metodo di
dare istruzioni per definirle risulta piuttosto complesso da
implementare e lontano dal nostro modo di ragionare (si veda per
esempio la macchina di Turing che calcola il prodotto).

Per questo introduciamo le macchine a registri, ossia delle
meta-macchine che lavorano con un linguaggio di secondo livello, che
\`e pi\`u complesso ma che permette di strutturare le istruzioni da
dare alla macchina di Turing in modo pi\`u facile e lineare. \`E
simile alla situazione di un linguaggio di programmazione di vario
livello: il linguaggio di programmazione che una macchina pu\`o
eseguire si chiama Assembler ed \`e poco pi\`u che una stringa di 0 e
1. Inizialmente era questo il linguaggio usato per la
programmazione. Successivamente sono stati inventati dei linguaggi di
pi\`u alto livello: si tratta di linguaggi pi\`u comprensibili che non
vengono eseguiti direttamente, ma vengono trasformati da un
metaprogramma in istruzioni del linguaggio Assembler.  Allo stesso
modo qui abbandoniamo il linguaggio macchina, che \`e quello delle
macchine di Turing (elementare, ma niente affatto banale, perch\'e
questo resta quello che sapremo fare), e adesso diamo delle istruzioni
di pi\`u alto livello che si possono leggere come abbreviazioni di
istruzioni di linguaggio. Possiamo quindi dire che:
\begin{thm}
Ogni funzione calcolabile a registri \`e Turing-cal\-co\-la\-bi\-le.
\end{thm}
\textsc{Dimostrazione}: \`e ovvio perch\'e, per le motivazioni scritte
sopra, ciascuna operazione eseguibile con una macchina a registri non
\`e altro che un insieme strutturato di istruzioni che definiscono una
macchina di Turing che calcola la medesima operazione.\\

Per proseguire con la definizione di macchina a registri, dobbiamo
introdurre dei registri di memoria dove inserire dei valori numerici
che si possano estrarre quando servono durante il calcolo. Inoltre
vogliamo che tali registri di memoria siano accessibili in modo
diretto.

Il fatto che una macchina di Turing lavori come un nastro permette di
registrare dei dati, ma non di accedervi in modo diretto: per esempio
se abbiamo \( k \) ingressi possiamo usare i blocchi dal \( k + 1
\)-esimo in poi come registri di memoria, ma in questo modo per andare
a vedere cosa c'\`e nel \( k + 1 \)-esimo posto dobbiamo spostarci sul
nastro, memorizzando dove eravamo e tornare indietro con inserita,
dentro lo stato interno, l'informazione; tutto ci\`o risulta molto
complicato.  Ora invece vediamo che \`e possibile definire un
linguaggio che faccia queste cose in modo automatico.

I registri contengono sempre un numero intero positivo. Un aspetto
importante \`e che non ci sono limitazioni n\'e alla grandezza dei
registri, cio\'e possiamo mettere in un registro numeri grandi a
piacere, n\'e al numero di registri che vogliamo usare. Potremo quindi
avere \( k \) registri che contengono gli input, un registro per l'
output e tutti i registri ausiliari che ci occorrono durante l'
operazione per inserire i risultati intermedi. Le operazioni permesse
sono:
\begin{itemize}
\item sommare 1 al contenuto del registro \( R_i \);
\item sottrarre 1 al contenuto del registro \( R_i \); vogliamo poter
  applicare tale operazione in ogni caso, quindi se c'\`e qualcosa nel
  registro si toglie 1 altrimenti il registro resta vuoto;
\item avere accesso diretto al registro \( R_i \).
\end{itemize}

Indicando con un \( n \) cerchiato il registro \( R_n \), scriviamo:
\begin{itemize}
\item \( n+ \) cerchiato per indicare l'operazione ``aggiungi 1 al registro \( R_n \)''
\item  \( n- \) cerchiato, per indicare l'operazione ``se il registro \( R_n \) non \`e vuoto
sottrai 1''.
\end{itemize}

Utiliz\-zia\-mo poi delle frecce che ci dicono come proseguire dopo
aver eseguito un' operazione:
\begin{itemize}
\item dal nodo \( n+ \) abbiamo una sola freccia che si legge
  ``aggiungi 1 al registro \( R_n \) e prosegui''
\item dal nodo \( n- \) partono due frecce e si legge ``se il registro
  \( R_n \) \`e vuoto prosegui a destra, se non \`e vuoto sottrai 1 e
  prosegui sotto''. Si noti che questo corrisponde all'aver introdotto
  l'operatore condizionale ``\emph{if-then-else}''.
\end{itemize}

\section{Esempi di Macchine a Registri}
Presentiamo ora una serie di esempi che aiuteranno a familiarizzare
con le macchine a registri.

\begin{esempio}[Proiezione \( p_n^i(x_1, \ldots, x_n)=x_i \).]

    Abbiamo supposto che, dati \( n \) registri, abbiamo accesso
    diretto \mbox{all' \( i \)-esimo} registro. Vogliamo l' output nel
    registro specifico da noi scelto (\textbf{ad esempio}) \( R_{3}
    \). Basta quindi accedere all' \( i \)-esimo registro e svuotarlo
    in \( R_{3} \).

    \begin{figure}[hbtp]
    \hspace{0cm}
    \xymatrix{
    \roundentry{i-} \ar@(ul,ul)[] \ar[d] \ar[r] & stop \\
    \roundentry{3+} \ar@(d,dl)[u]}
    \caption{Macchina a registri che calcola la proiezione sull' \( i \)-esimo registro}
    \end{figure}

\end{esempio}

\begin{esempio}[Successore]

    Partiamo con l' input in \( R_1 \) e vogliamo l' output in \(
    R_{3} \). Iniziamo aggiungendo 1 in \( R_{3} \) e poi svuotiamo \(
    R_1 \) in \( R_{3} \). (Si noti che non \`e detto che la macchina
    debba partire da \( R_1 \).)

    \begin{figure}[hbtp]
    \hspace{0cm}
    \xymatrix{
    \roundentry{3+} \ar@(ul,ul)[] \ar[d] \\
    \roundentry{1-} \ar@(d,dl)[u] \ar[r] & stop}
    \caption{Macchina a registri che calcola il successore di un numero dato in input}
    \end{figure}

\end{esempio}

\begin{esempio}[Differenza] \ \\
    \begin{center}
    $n \dot{-} m=
    \left\{ \begin{array}{ll}
    n-m & \textrm{se } m \leq n\\
    0 & \textrm{altrimenti}
    \end{array} \right.$
    \end{center}


    \ \\ I registri \( R_1 \) e \( R_2 \) contengono rispettivamente
    \( n \) e \( m \) e vogliamo l' output in \( R_{3} \) (che
    supponiamo inizialmente vuoto). Iniziamo togliendo 1 da \( R_2 \),
    poi sottraiamo 1 da \( R_1 \); continuiamo cos\`i fino a che \(
    R_2 \) \`e vuoto. A questo punto il contenuto di \( R_1 \) \`e $n
    - m$ e lo svuotiamo in \( R_{3} \). Se invece \( R_1 \) resta
    vuoto prima che si sia svuotato \( R_2 \) (cio\`e $n < m$) ci
    fermiamo (\( R_{3} \) \`e rimasto vuoto).

    \begin{figure}[hbtp]
    \hspace{0cm}
    \xymatrix{
      \roundentry{1-} \ar[r] \ar@(dr,d)[d] & stop \\
     \roundentry{2-} \ar@(ul,ul)[] \ar[u] \ar[r] & \roundentry{1-} \ar[d] \ar[r] & stop \\
     & \roundentry{3+} \ar@(d,dl)[u]
    }

   %  \includegraphics[scale=0.5]{diffeLabel.png}
    \caption{Macchina a registri che calcola la differenza $n \dot{-} m$}
    \end{figure}

\end{esempio}

\begin{esempio}[Somma \( m \) con \( n \), senza dimenticare \( m \).]
    Prendiamo tre registri:
    \begin{itemize}
    \item un registro \( R_1 \) dentro cui ci sar\`a \( m \);
    \item un registro \( R_2 \) dentro cui ci sar\`a \( n \);
    \item un registro ausiliario \( R_3 \), che inizialmente sar\`a vuoto;
    \end{itemize}
    Vogliamo ottenere l' output in \( R_2 \) e che \( m \) rimanga in
    \( R_1 \). L' operazione da eseguire \`e la seguente:

    Partiamo dal registro \( R_1 \): togliamo 1 dal contenuto di \(
    R_1 \), aggiungiamo 1 a quello di \( R_2 \) e poi aggiungiamo 1 in
    \( R_3 \). Ripetiamo queste operazioni fino a che \( R_1 \) rimane
    vuoto; ci\`o significa che abbiamo ``svuotato'' \( R_1 \) in \(
    R_2 \) e quindi in \( R_2 \) alla fine c' \`e \( m+n \), mentre in
    \( R_3 \) c' \`e \( m \), perch\'e ogni volta che passiamo in \(
    R_2 \) (\( m \) volte), aggiungiamo 1. Poi svuotiamo \( R_3 \) in
    \( R_1 \), cio\`e togliamo 1 da \( R_3 \) e aggiungiamo 1 in \(
    R_1 \) fino a quando \( R_3 \) rimane vuoto. A questo punto
    abbiamo finito.

    \begin{figure}[hbtp]
    \hspace{0cm}
    \xymatrix{
    \roundentry{1-} \ar@(ul,ul)[] \ar[r] \ar[d] & \roundentry{3-} \ar[d] \ar[r] & stop \\
    \roundentry{2+} \ar[d] & \roundentry{1+} \ar@(d,dl)[u] \\
    \roundentry{3+} \ar@(d,dl)[uu]
    }
    \caption{Macchina a registri che somma \( m \) con \( n \), senza dimenticare \( m \)}
    \end{figure}

\end{esempio}

\`E evidente come questo sistema sia molto pi\`u comprensibile rispetto a quello usato nelle macchine di Turing.

\begin{esempio}[Moltiplica \( m \) per \( n \)]
    Partiamo mettendo \( m \) in \( R_1 \) ed \( n \) in \( R_2 \);
    vogliamo il risultato in \( R_3 \). Usiamo \( R_4 \) come registro
    ausiliario.

    L'idea \`e sommare \( n \) con s\`e stesso \( m \) volte, quindi
    ogni volta che togliamo 1 da \( R_1 \) dobbiamo aggiungere al
    risultato \( n \). Per fare ci\`o cominciamo col sottrarre 1 dal
    registro \( R_1 \), svuotiamo il registro \( R_2 \), che
    all'inizio contiene \( n \), nel registro \( R_3 \) e allo stesso
    tempo mettiamo la stessa quantit\`a \( n \) nel registro \( R_4
    \); quando il registro \( R_2 \) \`e svuotato, rimettiamo il
    contenuto del registro \( R_4 \) nel registro \( R_2 \). A questo
    punto nel registro \( R_2 \) c'\`e \( n \), in \( R_4 \) c'\`e 0
    ed in \( R_3 \) c'\`e \( n \). Quindi ritorniamo al registro \(
    R_1 \) e sottraiamo un altro 1; ripetiamo l'operazione, e cio\'e
    la quantit\`a \( n \) del registro \( R_2 \) va svuotata nel
    registro \( R_3 \), quindi \( R_3 \) avr\`a \( n \) + \( n \) e \(
    R_4 \) avr\`a \( n \). Quando \( R_2 \) \`e vuoto risvuotiamo \(
    R_4 \) in \( R_2 \) e ripetiamo questo ciclo finch\'e \( R_1 \)
    non \`e vuoto. Quando quest'ultimo \`e vuoto ci fermiamo e
    leggiamo il risultato in \( R_3 \).

    \begin{figure}[hbtp]
    \hspace{0cm}
    \xymatrix{
    \roundentry{1-} \ar@(ul,ul)[] \ar[d] \ar[r] & stop \\
    \roundentry{2-} \ar[r] \ar[d] & \roundentry{4-} \ar[d] \ar@(r,r)[ul] \\
    \roundentry{3+} \ar[d] & \roundentry{2+} \ar@(d,dl)[u] \\
    \roundentry{4+} \ar@(d,dl)[uu]}
    \caption{Macchina a registri che moltiplica \( m \) per \( n \)}
    \end{figure}

\end{esempio}

\begin{esempio}[Composizione]

    Supponiamo che $ f $ e $ g $ siano due funzioni calcolabili a
    registri e vogliamo dimostrare che $ g\circ f $ \`e calcolabile a
    registri. Esistono quindi due macchine a registri \( M_f \) e \(
    M_g \) che calcolano rispettivamente \( f \) e \( g \) partendo
    dal registro \( R_1 \) e restituendo l' output in \( R_42
    \). Af\mbox{}f\mbox{}inch\'e non ci siano ambiguit\`a bisogna
    cambiare i nomi ai registri di \( M_g \), e quindi costruire una
    macchina a registri \( M_{g'} \) i cui re\-gi\-stri siano \(
    R_{n+1} \), \ldots, \( R_{n+42} \), con \( n \) maggiore del
    numero di registri di \( M_f \). A questo punto facciamo partire
    \( M_f \), questa restituisce l' output in \( R_{42} \), che
    svuotiamo in \( R_{n+1} \). Basta poi far partire \( M_{g'} \) e
    scaricare l' output di \( M_{g'} \) in \( R_{42} \).


    \begin{figure}[hbpt]
    \hspace{0cm} \xymatrix{ \framebox[1cm]{\( M_f \)} \ar@(ul,ul)[]
      \ar@(d,l)[r] & \roundentry{42-} \ar[d] \ar[r] &
      \framebox[1cm]{\( M_{g'} \)} \ar@(d,l)[r] &
      \roundentry{\ (n+42)\ -} \ar[d] \ar[r] & stop \\ &
      \roundentry{\ (n+1)\ +} \ar@(d,dl)[u] & & \roundentry{42+}
      \ar@(d,dl)[u] }
     \caption{Macchina a registri che calcola la composizione di due funzioni}
    \end{figure}
\end{esempio}

Ora che abbiamo visto come funzionano le macchine a registri, diamo
una definizione formale di "funzione calcolabile a re\-gi\-stri".

\begin{thm}
Una funzione \( f \) (di \( n \) variabili) \`e calcolabile a registri
se esiste una macchina a registri che, mettendo gli input \( x_1,
\ldots,\\ x_n \) nei registri pref\mbox{}issati \( R_1, \ldots, R_n
\), d\`a output \( f(x_1, \ldots, x_n) \) per ogni \( x_1, \ldots, x_n
\).
\end{thm}

\chapter{Funzioni Ricorsive}
\label{lemma diagonalizzazione}

Lo scopo di questo capitolo \`e quello di dimostrare l'equivalenza tra
i concetti di calcolabilit\`a con registri e calcolabilit\`a con
programmi.  Iniziamo dunque a definire i programmi.

\section{I programmi}
Nella sezione precedente abbiamo visto come il contenuto dei registri,
nelle macchine a registri, possa essere modificato attraverso
determinate \emph{istruzioni} che la macchina \`e in grado di
riconoscere. Una sequenza di queste semplici istruzioni costituisce un
\emph{programma}.


\begin{programmi}
Chiamiamo programma P una sequenza finita non nulla di istruzioni etichettate
\end{programmi}

Chiameremo $x_i$, $x_j$ e $x_k$ variabili dove $i,j,k \in \mathbb{N}$. Assumeremo come convenzione,
che l'output di $P$ deve essere messo (dal programmatore) nella variabile $x_0$.
Inoltre l'ennupla $a_1, a_2, ..., a_n \in \mathbb{N}^k $ fornita come input al nostro programma verrà memorizzato nelle prime n variabili $x_1, ..., x_n$.
Assumiamo anche che tutte le variabili utilizzare nel nostro programma siano automaticamente inizializzate a zero e scriveremo $P(a_1, a_2, ..., a_n)$ per dire che eseguiamo $P$ con l'input $a_1, a_2, ..., a_n$.

Le istruzioni del programma possono essere:

\begin{enumerate}
\item $x_j \leftarrow 0$ assegna ad $x_j$ il valore zero. (azzeramento)
\item $x_j \leftarrow x_i $ assegna il contenuto di $x_i$ a $x_j$ (trasferimento)
\item $x_j \leftarrow x_j+1$ aumenta il contenuto di $x_j$ di uno e 
  riassegna il risultato come valore a $x_j$ (successsore)
\item $\textrm{if }(condizione) $ then goto $\alpha$ else goto $\beta$ con
  $\alpha$ e $\beta$ etichette di istruzioni del programma. (salto)
\end{enumerate}
	
Esempi di condizioni possono essere il confronto fra due variabili, il confronto di
una variabile con un numero naturale ecc..

\begin{nota}
Nella (\emph{4}) come condizione \`e possibile utilizzare 
una qualunque formula atomica di un linguaggio che usi le relazioni
$=,\geq,<$ e le operazioni $+,-,successore,\ldots $ e le loro
composizioni.
\end{nota}



\begin{extra}
Si scriva un programma in cui sia presente un'istruzione del tipo
(\emph{4}) avente per condizione $x_i\neq x_j$. (\emph{Suggerimento}:
si pensi ad una funzione in $(x_i,\ldots ,x_j)$ il cui valore sia
diverso da $0$ quando $x_i\neq x_j$)
\end{extra}	

\begin{extra}
Mostrare, con qualche esempio, che \`e possibile sostituire alla
condizione della (\emph{4}) una qualsiasi espressione di
calcolo \newline booleano classico.
\end{extra}	

Per eseguire una \emph{computazione} la macchina deve essere
fornita di un \emph{programma} $P$ e di una \emph{configurazione
  iniziale} (input), che pu\`o essere, ad esempio, una sequenza $a_1, a_2,
...$ di numeri naturali.
Supponiamo che $P$ sia composto da $s$ istruzioni $I_1, I_2,...,
I_s$.

Allora la macchina inizia la computazione eseguendo $I_1$, poi $I_2$,
e cos\`i via, a meno che non si incontri un'istruzione di \emph{goto}
che fa saltare la computazione in una delle $s$ istruzioni. La
computazione si ferma quando, e solo quando non c'\`e una prossima
istruzione oppure incontriamo la parola chiave \textbf{stop}.
Possiamo illustrare questo attraverso un esempio.
	
\begin{esempio} Consideriamo il seguente programma con $s=6$ istruzioni:
\begin{mylisting}
$I_1$: $\textrm{if }x_1 > x_2$ then goto $I_2$ else goto $I_5$
  \\ $I_2$: $x_2 \leftarrow x_2+1$\\ $I_3$: $x_3 \leftarrow
  x_3+1$\\ $I_4$: $\textrm{if }x_1 \neq x_2$ then goto $I_2$ else goto
  $I_5$\\ $I_5$: $x_0 \leftarrow x_3$\\
\end{mylisting}
E con la seguente configurazione iniziale:
\begin{mylisting}
$x_0 = 0, x_1 = 9, x_2 = 7, x_3 = 0, ...$
\end{mylisting}
Durante il calcolo le variabili vengono modificate come segue:
\begin{mylisting}	
$x_0 = 0, x_1 = 9, x_2 = 7, x_3 = 0, ...$ ($I_1: x_1 > x_2$)\\ $x_0 =
  0, x_1 = 9, x_2 = 8, x_3 = 0, ...$ ($I_2$)\\ $x_0 = 0, x_1 = 9, x_2
  = 8, x_3 = 1, ...$ ($I_3$)\\ $x_0 = 0, x_1 = 9, x_2 = 8, x_3 = 1,
  ...$ ($I_4: x_1 \neq x_2$)\\ $x_0 = 0, x_1 = 9, x_2 = 9, x_3 = 1,
  ...$ ($I_2$)\\ $x_0 = 0, x_1 = 9, x_2 = 9, x_3 = 2, ...$
  ($I_3$)\\ $x_0 = 0, x_1 = 9, x_2 = 9, x_3 = 2, ...$ ($I_4: x_1 =
  x_2$)\\ $x_0 = 2, x_1 = 9, x_2 = 9, x_3 = 2, ...$ ($I_5$)\\
\end{mylisting}	
\end{esempio}

\begin{nota}		
Al momento non poniamo la nostra concentrazione su quale funzione
calcola questo programma, ma ci limitiamo ad illustrare in quale modo
opera un programma in senso puramente meccanico.
\end{nota}

\begin{extra}
Dare un programma che memorizza in $x_0$ 1 se $x_1 > x_2$ e 0
altrimenti.
\end{extra}	

Per comprendere il significato del programma e l'andamento della
computazione \`e spesso conveniente desciverlo in modo informale
attraveso un \emph{Flow Chart}, per esempio il flow chart
rappresentante il programma dell'Esempio 1.1 \`e dato in Figura
1. Notare che i test contenuti nei rombi rappresentano le istruzioni
\emph{if then else} (4) che hanno quindi due prosecuzioni in base al
risultato del test; mentre i rettangoli sono le istruzioni di
successore o assegnazione che continuano sempre con la prossima
istruzione.
		
\begin{figure}[h] 
\begin{tikzpicture}[node distance = 1.6 cm, auto]
 % Place nodes
   \node[noblock] (start) {START};
    \node [decision, below of=start] (I1) {$x_1 > x_2$};
    \node [block, below of=I1] (I2) {$x_2 \leftarrow x_2 + 1$};
    \node [block, below of=I2] (I3) {$x_3 \leftarrow x_3 + 1$};
    \node [decision, below of=I3] (I4) {$x_1 \neq x_2$};
    \node [block, below of=I4] (I5) {$x_0 \leftarrow x_3$};
   \node[noblock,below of=I5] (stop) {STOP};

    
    % Draw edges 
   \path [line] (start) -- (I1); \path [line] (I1) --
   node{yes}(I2); \path [line] (I1.east) -- ++(1.0,0) -- ++(0,-1) |-
   node [near start] {no}(I5.east); \path [line] (I2) -- (I3); \path
   [line] (I3) -- (I4); \path [line] (I4.west) -- ++(-.8,0) --
   ++(-.8,0) |- node [near start] {yes}(I2.west); \path [line] (I4)
   --node{no} (I5); \path [line] (I5) -- (stop);
\end{tikzpicture}
\caption{Flow Chart}   
	
\end{figure}
	
Dall'esempio 1 notiamo inoltre come il comando \emph{goto} nella
definizione dell'\emph{if} renda possibile l'esecuzione di cicli,
infatti l'istruzione $I_4$ riporta l'esecuzione all'istruzione $I_2$
tante volte fino a quando la condizione $x_1 \neq x_2$ non viene
soddisfatta.
	
Pi\`u in generale possiamo dire che il comando \emph{goto} pu\`o
essere utilizzato per simulare quello che nei linguaggi ad alto livello conosciamo come 
ciclo \emph{for}. 
Infatti con $n$ fissato il ciclo:\\
$$ \left[
\begin{array}{l}
for \; $i = 1$ \; to \; $n$\\ \ \ \ \ \left[P\;
programma]\right. \\ next \; $i$\\
\end{array} \right.
\
$$ 
viene implementato dal programma nell'Esempio 1.2 dove $n$ è un opportuna variabile
che riceve in input il valore di $n$.

\newpage

\begin{esempio}

\begin{minipage}[c]{.50\textwidth}
\begin{mylisting}
 $I_1: x_i \leftarrow 0 $\\ 
 $I_2: x_i \leftarrow x_i + 1 $ \\
 $I_3: $ if $ n<x_i $ then goto stop else goto $I_4$ \\
 $I_4:  \left[ P\; programma \right ] $ \\
 $I_5: x_i \leftarrow x_i+1 $ \\
 $I_6 $ if $n < x_i $ then \\ \hspace{\stretch{1}} goto stop  else goto $I_4$\\

\end{mylisting}
\end{minipage}
\hspace{5mm}%
\begin{minipage}[c]{.50\textwidth}
\begin{tikzpicture}[node distance = 1.6 cm, auto]
    % Place nodes 
\node[noblock] (start) {START}; 
\node [block, below of=start] (I1) {$x_i \leftarrow 0$};
\node [block, below of=I1] (I2) {$x_i \leftarrow x_i+ 1$};
\node [decision, below of=I2] (I3) {$n < x_i$};
\node [block, below of=I3] (I4) {$ \left [ P \right] $};
\node [block, below of=I4] (I5) {$x_i \leftarrow x_i+ 1$};
\node [decision, below of=I5] (I6) {$n < x_i$};
\node[noblock,below of=I6] (stop) {STOP};

    % Draw edges
\path [line] (start) -- (I1);
\path [line] (I1) -- (I2);
\path [line] (I2) -- (I3);
\path [line] (I3.west) -- ++(-.8,0) -- ++(-.8,0) |- node [near start] {yes}(stop.west);
\path [line] (I3) -- node{no} (I4);
\path [line] (I4) -- (I5);
\path [line] (I5) -- (I6);
\path [line] (I6.east) -- ++(+.10,0) -- ++(+.8,0) |- node [near start] {no}(I4.east);
\path [line] (I6) --node{yes} (stop);

\end{tikzpicture}
\end{minipage}

\end{esempio}

\vspace{5mm}% 
	
\newpage

Ci chiediamo adesso, le quattro istruzioni che abbiamo introdotto sono tutte necessarie? 
E' possibile toglierne qualcuna e continuare a calcolare le stesse funzioni?
E immediato verificare che l'istruzione di trasferimento $x_j \leftarrow x_i$ può
essere scritta in funzione delle altre. Vediamo l'esempio:

\begin{esempio} utilizziamo le altre tre istruzioni per simulare il comportamento dell' istruzione di trasferimento:
\begin{mylisting}
$I_1$: $\textrm{if }x_j = x_i$ then goto stop else goto $I_2$\\
$I_2$: $x_j \leftarrow 0$\\
$I_3$: $x_j \leftarrow x_j +1$\\ 
$I_4$: $\textrm{if }x_j = x_i$ then goto stop else goto $I_2$\\
\end{mylisting}
\end{esempio}

Abbiamo quindi introdotto l'istruzione di trasferimento al solo fine di semplificare 
la scrittura dei nostri programmi.

Un altra domanda che sorge è: "senza l'istruzione di salto
riusciamo a calcolare le stesse funzioni di quante ne calcoleremo con tutte e quattro? Ebbene si può dimostrare che 
le uniche funzioni che possiamo calcolare senza l'istruzione di salto sono del tipo

\begin{enumerate}
\item $f(x) = m \, \, m \in  \mathbb{N} $ 
\item $f(x) = x+ m \, \,  \forall x ,  m \in  \mathbb{N} $ 
\end{enumerate}

\begin{extra}
Dimostrare che senza l'istruzione di if .. goto calcoliamo solo funzioni del tipo (1) e (2).
\end{extra}

\begin{extra}
Dimostrare che senza l'istruzione di if .. goto calcoliamo solo funzioni che sono totali.
\end{extra}

Questo fatto ci mostra che con tutta la sua semplicità l'istruzione di salto 
è importantantissima per calcolare quanto c'e di calcolabile e ciò si traduce nel fatto che
se il nostro linguaggio di programmazione preferito non disponesse dell' istruzione di salto 
esso servirebbe a ben poco.

Notiamo anche il fatto che non tutti i moderni linguaggi di programmazione 
dispongono dell'istruzione goto dovuto al fatto che l'uso questa istruzione 
è generalmente considerato indice di cattiva programmazione
ma ciò è stato rimpiazzato dall'uso delle cosiddette strutture di controllo iterative (for e while)
che implicitamente permettono di saltare in maniera controllata da un punto 
all'altro del programma.


\subsection{Funzioni calcolabili con un programma}

Abbiamo visto dei capitoli precedenti come in maniera naturale "leghiamo" il
concetto di macchina al concetto di funzione, così come in questo 
capitolo abbiamo legato il concetto di programma a quello di funzione.
Cioè è naturale pensare il calcolo del valore di una qualsiasi funzione $f(a_1, a_2, ... , a_n)$ 
per mezzo di un programma $P$ con configurazione iniziale $a_1, a_2, ... , a_n$.

In generale, la funzione associata ad un programma è parziale, anziché totale, in quanto tra le caratteristiche dei nostri programmi, rientra la
posibilità che, per certi input non venga prodotto alcun output.

Si consideri ad esempio il seguente programma:
	
\begin{mylisting}
$I_1$: if $x_i = x_i $ then goto $I_1$ else goto stop
\end{mylisting}	

che non termina mai. La condizione è sempre vera è ciò causa l'esecuzione
continua della stessa istruzione. La funzione calcolata
da questo programma la chiameremo funzione sempre indefinita.

Diamo adesso una definizione di funzione calcolabile con un programma.

\begin{programmi}
Una funzione $f$ si dice \emph{calcolabile da un programma} se,
$\exists$ un programma $P$ tale che per ogni $(a_1,...,a_n) \in \mathbb{N}^k$ 
$P(a_1,...,a_n)$ termina sse $(a_1,...,a_n) \in dom(f) $ e  $x_0=a$ se 
$f(a_1,...,a_n) = a$
\end{programmi}

\subsection{Equivalenza tra programmi e macchine a registri}

E' facile verificare che con una macchina a registri si possono
eseguire le stesse operazioni viste all'inizio della sezione (e
viceversa), e in effetti vale il seguente teorema.
\begin{progabaco}
Ogni funzione \'e computabile con un programma se e solo se \'e
computabile con una macchina a registri.
\end{progabaco}
\textsc{Dimostrazione} $(\Rightarrow)$ Vediamo come le macchine a
registri sono in grado di calcolare le quattro istruzioni che
costituiscono un programma:\\

\begin{enumerate}
\item 
 $x_j \leftarrow 0$ \hspace{10mm} \xymatrix{ \roundentry{j-}
  \ar@(ul,ul)[] \ar@(ul,dl)[] \ar[r] & stop }
 \vspace{10mm}

\item $x_j \leftarrow x_i $ \hspace{10mm} \xymatrix{ \roundentry{j-}  
  \ar@(ul,ul)[] \ar@(ul,dl)[] \ar[r] & \roundentry{i-} \ar[d]  \ar[r] &  
  \roundentry{42-} \ar@(dr,d)[d] \ar[r] & stop \\ & \roundentry{j+} \ar[d] & \roundentry{i+} \ar[u]
  \\ & \roundentry{42+} \ar@(dl,dl)[uu]}
\vspace{10mm}

\item $x_j \leftarrow x_j+1$ \hspace{10mm} \xymatrix{ \roundentry{j+}
  \ar@(ul,ul)[] \ar[d] \\ ...  }
\vspace{10mm}

\item $\textrm{if }x_i \neq0$ then goto $\alpha$ else goto
  $\beta$ \hspace{10mm} \xymatrix{ \roundentry{i-} \ar@(ul,ul)[]
  \ar[d] \ar[r] & \beta\\ \roundentry{i+} \ar[d] \\ \alpha }

\end{enumerate}
	
$(\Leftarrow)$ Si tratta di simulare con (1)$\rightarrow$(4) le
operazioni di una macchina a registri:
\begin{enumerate}
\item esegue l'incremento di uno, la prima operazione elementare di
  una macchina a registri;\\ \\
\hspace{10mm} \xymatrix{\roundentry{j+} \ar@(ul,ul)[] \ar[d] \\ ...}
\hspace{10mm} $x_j \leftarrow x_j+1$

\item la seconda operazione elementare sulle macchine a registri si
  ottiene usando (4) e ponendo in $\alpha$ un'istruzione che sottragga
  1 al registro se non \'e vuoto, altrimenti il programma esegue
  l'istruzione $else$ $\beta$, che corrisponde all'operazione di
  procedere a destra.\\ \\ \xymatrix{ \roundentry{i-} \ar@(ul,ul)[]
    \ar[d] \ar[r] & \beta\\ \alpha }
 \hspace{10mm} $\textrm{if }x_i \neq 0$ then goto $\alpha$ else goto
 $\beta$ \\ con $\alpha = x \stackrel{\centerdot}{-} 1$ e $\beta=$
 'procedi a destra'.
\end{enumerate}

\hspace{\stretch{1}} $\Box$\\
\begin{esempio}[Differenza]
\ Come esempio riportiamo il calcolo della funzione differenza
definita come segue:\\
\begin{center}
$n \stackrel{\centerdot}{-} m= \left\{ \begin{array}{ll} n-m &
    \textrm{se } m \leq n\\ 0 & \textrm{altrimenti}
\end{array} \right.$
\end{center} 

\begin{figure}[h]
\hspace{0cm} \xymatrix{ \roundentry{1-} \ar[r] \ar@(dr,d)[d] & stop
  \\ \roundentry{2-} \ar@(ul,ul)[] \ar[u] \ar[r] & \roundentry{1-}
  \ar[d] \ar[r] & stop \\ & \roundentry{42+} \ar@(d,dl)[u] }
\caption{Macchina a registri che calcola la differenza definita sui
  naturali $n \stackrel{\centerdot}{-} m$}
\end{figure}
	
\ \\ Tale funzione viene calcolata dalla macchina a registri in Figura
2 e dal programma o che abbiamo esaminato nell'Esempio 1.1.
\end{esempio}
	
	
Il teorema appena enunciato stabilisce la completa equivalenza tra i
concetti di calcolabilità con registri e calcolabilità con
programmi. Si noti come attraverso queste equivalenze si giunga ad un
livello di astrazione sempre maggiore, e come proprio questa
progressione ci fornisca strumenti via via pi\'u versatili attraverso
i passaggi per la dimostrazione dei teoremi di incompletezza.
				

\section{Funzioni primitive ricorsive}
Vogliamo definire ora una classe che comprenda tutte e sole le
funzioni calcolabili. Seguiremo a tale scopo Post, il quale si serve
di una definizione ricorsiva. Una definizione \`e detta
\emph{ricorsiva} quando ci\`o che \`e da definirsi viene definito
facendo ricorso a istanze pi\`u elementari dello stesso problema.
Tale metodo consiste nel:
\begin{itemize}
 \item fissare un insieme di funzioni iniziali immediatamente
   calcolabili quale base della procedura di definizione
 \item indicare alcune regole per derivare altre funzioni ricorsive da
   quelle date in partenza (regole che ovviamente preservino la
   calcolabilità delle funzioni derivate)
 \item escludere dalla classe delle funzioni ricorsive quelle funzioni
   che non siano le iniziali o da queste derivabili.
\end{itemize}
Diamo dunque la seguente definizione:

\begin{programmi}
Si dice \emph{funzione primitiva ricorsiva} un elemento della
classe definita induttivamente a partire dalle seguenti funzioni:
\begin{itemize}
\item [a.] la funzione nulla $z$ tale che $z(x)=0$;
\item [b.] la funzione proiezione $p^n_i$ tale che $p_i^n(x_1,\ldots
  ,x_n)=x_i$ $\forall i\leq n$
\item [c.]la funzione successore $s$ tale che $s(x) = x+1$;
\end{itemize}
\end{programmi}
	
Le regole per produrre nuove funzioni sono:
\begin{enumerate}
\item le operazioni di \emph{\emph{composizione}}, che date le
  funzioni primitive ricorsive $f: \mathbb{N}^k \rightarrow
  \mathbb{N}$ e $g_i : \mathbb{N}^n \rightarrow \mathbb{N}$ per $i= 1,
  ... , k$ permette di ottenere una funzione $h : \mathbb{N}^n
  \rightarrow \mathbb{N}$ per cui:\\ $
  h(x_1,.....,x_n)=f(g1(x_1,.....,x_n),....., g_k (x_1,.....,x_n))$
  \\ anch'essa primitiva ricorsiva.
\item lo schema di \emph{\emph{ricorsione primitiva}}, che date
  le funzioni primitive ricorsive $f: \mathbb{N}^k \rightarrow
  \mathbb{N}$ e $g : \mathbb{N}^{k+2} \rightarrow \mathbb{N}$, allora
  \`e primitiva ricorsiva anche la funzione $h : \mathbb{N}^{k+1}
  \rightarrow \mathbb{N}$ che soddisfi il sistema di equazioni:
\begin{center}
$\left\{ \begin{array}{ll} h(x_1,.....,x_n,0)= f(x_1,.....,x_n)
    \\ h(x_1,.....,x_n,s(y))=g(x_1,.....,x_n, y, h(x_1,.....,x_n,y))
				\end{array} \right.$
\end{center}
\end{enumerate}
\vspace{5mm}%
	
La classe delle funzioni primitive ricorsive \`e dunque \emph{chiusa
  rispetto alle operazioni di composizione e ricorsione
  primitiva}. Ora vedremo che questa classe di funzioni, che abbiamo
appena definito in termini matematici, \`e calcolabile dai programmi.


\begin{thm}
\emph{Ogni funzione primitiva ricorsiva è eseguibile da un programma
(e quindi calcolabile).}\end{thm}
\begin{proof}
Dobbiamo fare un programma per le tre funzioni di base, per la
composizione generalizzata e la ricorsione. Come sappiamo l' output è
per convenzione in $x_{0}$. Assumiamo anche in questa dimostrazione che 
se un programma P utilizza una variabile per la prima volta prima di usarla
azzeri il suo contenuto. Questa assunzione deve essere fatta perché
quando parleremo di ricorsione e composizione delle funzioni 
i programmi verranno eseguiti uno dopo l'altro e tra le loro possibilità rientra quella 
di utilizzare le stesse variabili usate dai programmi precedenti. Ciò assicura
che azzerando la variabile prima dell'uso non si trovino valori `sporchi`.

\begin{itemize}
\item[a.] Facciamo un programma per la funzione zero:
\begin{mylisting}
$I_1$: $x_{0}\leftarrow0$
\end{mylisting}

\item[b.] Il programma per il successore:\begin{mylisting}$I_1$:
  $x_{1}\leftarrow x_{1}+1$ \\$I_2:x_{0}\leftarrow x_{1} $\end{mylisting}

\item[c.] La proiezione:\begin{mylisting}$I_1$: $x_{0}\longleftarrow
  x_{i}$\end{mylisting}
\end{itemize}

\begin{enumerate}
\item La composizione: siano $F, G_1, G_2, \dots, G_k$ programmi che calcolano
  le funzioni $f,g_{1},...g_{k}$ rispettivamente. Dobbiamo far vedere che
  riusciamo ad esibire un programma $H$ che calcola la funzione $h$.
  L'output di ogni
  funzione viene salvato in $x_{0}$; quindi, poich\'e vogliamo salvare
  tutti i risultati intermedi del calcolo di $g_{1},\dots,g_{k}$
  dobbiamo copiare di volta in volta $x_0$ in una variabile che siamo
  sicuri che non sia usata da uno dei programmi. Inoltre dobbiamo
  riservarci dello spazio per le variabili di input poich\'e qualche
  programma potrebbe modificarle durante la sua esecuzione.  Per
  sapere dove possiamo salvare tutti questi valori dobbiamo sapere
  quali variabili vengono utilizzate dai programmi.
  Sia quindi
  $$\rho(Q)=max\left\{ i|x_{i}\leftarrow...\in Q\right\}$$
  il massimo indice di una variabile assegnata in $Q$ e
  $$\varepsilon=max\left\{
  \rho(G_{1}),....,\rho(G_{k}),\rho(F),k,n\right\}$$
  il massimo indice utilizzato nei programmi $F, G_1, \dots, G_k$.

Costruiamo il programma $H$ che calcola la composizione come segue:
salviamo l'input, che si trova in $x_{1},\dots,x_{n}$ in
$x_{\varepsilon+1},\dots,x_{\varepsilon+n}$, eseguiamo il programma
$G_1$ e copiamo l'output in $x_{\varepsilon+n+1}$:

\begin{mylisting}
$x_{\varepsilon+1}\leftarrow x_{1}$\\
$\vdots$\\
$x_{\varepsilon+n}\leftarrow x_{n}$\\
$[G_{1}]$\\
$x_{\varepsilon+n+1}\leftarrow x_{0}$
\end{mylisting}

Copiamo l'input:
\begin{mylisting}
$x_{1}\leftarrow x_{\varepsilon+1}$\\
$\vdots$\\
$x_{n}\leftarrow x_{\varepsilon+n}$
\end{mylisting}

Eseguiamo $G_{2}$ e copiamo l'output in $x_{\varepsilon+n+2}$;
ripetiamo questa operazione per tutti gli altri programmi fino ad
applicare il programma $G_{k}$ e copiare il suo output in
$x_{\varepsilon+n+k}$.

A questo punto tutti gli input per F si trovano in
$x_{\varepsilon+n+1},\dots,x_{\varepsilon+n+k}$.  Copiamo questi
valori in $x_{1},\dots,x_{k}$ e applichiamo il programma F:

\begin{mylisting}
$x_{1}\leftarrow x_{\varepsilon+n+1}$\\
$\vdots$\\
$x_{k}\leftarrow x_{\varepsilon+n+k}$\\
$[F]$
\end{mylisting}

Dopo questa sequenza di operazioni il risultato della composizione si
trova in $x_{0}$.

\item La ricorsione: sia $F$ un programma che calcola $f$ e $G$ un
  programma per $g$. Per realizzare un programma $H$ che calcola $h$
  dobbiamo innanzitutto salvare le variabili di input e i risultati
  parziali di $f$ e $g$ in variabili non utilizzate dai due programmi
  $F$ e $G$.

  Sia $\rho(Q)=max\left\{ i|x_{i}\leftarrow\dots \in Q\right\}$ il
  massimo indice utilizzato nel programma Q; definiamo
  $\varepsilon=max\left\{ \rho(F),\rho(G),k+2\right\}$ il massimo
  indice utilizzato per entrambi i programmi.

Allora, prima copiamo l'input e mettiamo 0 in $x_{\varepsilon+k+2}$
per cominziare la ricorsione (per calcolare $h(x,0)$). Adesso possiamo
applicare il programma $F$.

Poi dobbiamo guardare se l'$y$ che ci è stato dato all'inizio che copiamo
in $x_{\varepsilon+k+3}$ è uguale
a zero; se è cosi, il programma deve terminare. Se non è cosi, il
programma deve calcolare $h(x,s(y))$ (la prima volta $s(y)$ sarà
1). Il programma G utilizza come input $x_{1},\dots,x_{k}$ per
$x_{1},\dots,x_{k}$, $x_{k+1}$ per $y$ e $x_{k+2}$ per il risultato
parziale $h(x,y)$. Quindi dobbiamo mettere l'input iniziale in
$x_{1},\dots,x_{k}$, il risultato al passo precedente che si trova in
$x_{\varepsilon+k+2}$ dobbiamo copiarlo in $x_{k+1}$ e, dopo aver
eseguito $G$, dobbiamo copiare il suo output da $x_{0}$ in $x_{k+2}$.
Infine calcoliamo il successore di $y$ e controlliamo se equivale
all'$y$ iniziale: in caso affermativo dobbiamo terminare, altrimenti
torniamo ad eseguire il ciclo. Allora, il programa sarà:

\begin{mylisting}
$x_{\varepsilon+1}\leftarrow x_{1}$\\
$\vdots$\\
$x_{\varepsilon+k}\leftarrow x_{k}$\\
$x_{\varepsilon+k+3}\leftarrow x_{k+1}$ \\
$[F]$\\
$x_{\varepsilon+k+2}\leftarrow x_{0}$\\
$x_{\varepsilon+k+1}\leftarrow 0$\\
$I_t:\; $if$\; x_{\varepsilon+k+3}=x_{\varepsilon+k+1}\; $then goto$\;stop$ else goto $I_{t+1}$\\
$I_{t+1}:x_{1}\leftarrow x_{\varepsilon+1}$\\
$\vdots$\\
$x_{k}\leftarrow x_{\varepsilon+k}$\\
$x_{k+1}\leftarrow x_{\varepsilon+k+1}$\\
$x_{k+2} \leftarrow x_{\varepsilon+k+2}$\\
$[G]$\\
$x_{\varepsilon+k+2} \leftarrow x_0$\\
$x_{\varepsilon+k+1}\leftarrow x_{\varepsilon+k+1}+1$\\
$if \; x_0=x_0\; $ then goto $I_t$ else ...
\end{mylisting}
\end{enumerate}
\end{proof}

% Controllare
\subsection{Esempi di funzioni primitive ricorsive}

\begin{esempio}[somma]
$h:\mathbb{N}^2 \to \mathbb{N}$, $h(x,y)=x+y$.
Possiamo usare lo schema semplificato
$\left\{ \begin{array}{ll}
	x+0=x\\
	x+s(y)=s(x+y)
\end{array}\right.$.\\
Nel nostro schema di ricorsione la stessa funzione si ottiene con $h$
tale:\newline
$$\begin{array}{ll}
	h(x,0)=f(x)=p_1^1(x)=x\\
	h(x,s(y))=g(x,y,h(x,y))=s(p_3^3(x,y,h(x,y)))=s(h(x,y)).
\end{array}$$\newline
\end{esempio}
%
\begin{esempio}[prodotto]

 $h:\mathbb{N}^2 \to \mathbb{N}$, $h(x,y)=x \cdot y$ che possiamo scrivere
come
$\left\{ \begin{array}{ll}
	x \cdot 0=0\\
	x \cdot s(y)= x \cdot y + x
\end{array}\right.$ e quindi: \newline
$$\begin{array}{ll}
	h(x,0)=f(x)=z(x)\\
	h(x,s(y))=g(x,y,h(x,y))=p_3^1(x,y,h(x,y))+p_3^3(x,y,h(x,y)).
\end{array}$$\newline
\end{esempio}
%
\begin{esempio}[fattoriale] Per adattare la definizione della funzione
fattoriale
$h:\mathbb{N} \to \mathbb{N}$, $h(x)=x!$ allo schema generale si considera
$\left\{ \begin{array}{ll}
	0!=1\\
	s(y)! = y! \cdot s(y)
\end{array}\right.$ e quindi si pu\`o prendere $h$ come segue: \newline
$$\begin{array}{ll}
	h(x,0)=f(x)=s(z(x))=1\\
	h(x,s(y))=g(x,y,h(x,y))= s(p_3^2(x,y,h(x,y)))) \cdot p_3^3(x,y,h(x,y))=
\\
	\hspace{1.8cm} = s(y)\cdot h(x,y).
	\end{array}$$\newline
\end{esempio}
%
%
\begin{esempio}[elevamento a potenza] $h:\mathbb{N}^2 \to \mathbb{N}$,
$h(x,y)=x^y$:
$\left\{ \begin{array}{ll}
	x^0=1\\
	x^{s(y)} = x^y \cdot x
\end{array}\right.$ e quindi: \newline
$$\begin{array}{ll}
	h(x,0)=f(x)=s(z(x))=1\\
	h(x,s(y))=g(x,y,h(x,y))= p_3^1(x,y,h(x,y)) \cdot p_3^3(x,y,h(x,y)) =
x\cdot h(x,y).
	\end{array}$$\newline
\end{esempio}
%
%
\begin{esempio}[predecessore] $p:\mathbb{N} \to \mathbb{N}$ tale che
$\left\{ \begin{array}{ll}
	p(0)=0\\
	p(s(y)) = y
\end{array}\right.$ e quindi: \newline
$$\begin{array}{ll}
	p(0)=f=0\\
	p(s(y))=g(y,p(y))= p_2^1(y,p(y)).
	\end{array}$$\newline
\end{esempio}
%
%
\begin{esempio}[sottrazione] $h:\mathbb{N}^2 \to \mathbb{N}$,
$ h(x,y) = \left\{ \begin{array}{ll}
	x \stackrel{\centerdot}{-} y \ se \ y \leq x\\
	0 \ altrimenti
\end{array}\right.$ dunque: \newline
$$\begin{array}{ll}
	h(x,0)=f(x)=x\\
	h(x,s(y))=p(h(x,y)).
	\end{array}$$\newline
\end{esempio}
%
%
\begin{esempio}[segno] $sgn:\mathbb{N} \to \left\{0,1\right\}$,
$ sgn(y) = \left\{ \begin{array}{ll}
	0 \ se \ y = 0\\
	1 \ se \ y > 0
\end{array}\right.$ che si pu\`o scrivere come:
 \[ sgn(y) = y \stackrel{\centerdot}{-} p(y) \]
che \`e primitiva ricorsiva per quanto visto negli esempi precedenti.
\end{esempio}
%
%
\begin{esempio}[controsegno] $\overline{sgn}:\mathbb{N} \to
\left\{0,1\right\}$,
$ \overline{sgn}(y) = \left\{ \begin{array}{ll}
	1 \ se \ y = 0\\
	0 \ se \ y > 0
\end{array}\right.$ che si pu\`o pensare come:
 \[ \overline{sgn}(y) = 1 \stackrel{\centerdot}{-} sgn(y) \]
dunque \`e primitiva ricorsiva.
\end{esempio}
%
%
\begin{esempio}[] $f(\overrightarrow{x},y)= \sum_{i=0}^{y} g(\overrightarrow{x},
i)$ con $g$ primitiva ricorsiva, allora:\\
 $ f(\overrightarrow{x}, 0) = g(\overrightarrow{x}, 0)  $ \\
 $ f(\overrightarrow{x}, s(y)) = f(\overrightarrow{x}, y) +
g(\overrightarrow{x}, s(y)). $
\end{esempio}
%
%
\begin{esempio}[] $f(\overrightarrow{x},y)= \prod_{i=0}^{y}
g(\overrightarrow{x}, i)$ con $g$ primitiva ricorsiva, allora:\\
$	f(\overrightarrow{x}, 0) = g(\overrightarrow{x}, 0) $ \\
$ 	f(\overrightarrow{x}, s(y)) = f(\overrightarrow{x}, y) \cdot
g(\overrightarrow{x}, s(y)). $
\end{esempio}
%
%
\begin{esempio}[] $ \chi_{\geq}(x,y) = \left\{ \begin{array}{ll}
	1 \ se \ x \geq y\\
	0 \ altrimenti
\end{array}\right.$
\[ \chi_{\geq}(x,y) = sgn (s(x) \stackrel{\centerdot}{-} y) . \]
\end{esempio}
%
%
\begin{esempio}[valore assoluto] Basta scriverlo come
\[ \left|x - y \right| = (x \stackrel{\centerdot}{-} y) + ( y
\stackrel{\centerdot}{-} x) \]
oppure
\[ \left|x - y \right| = (x \stackrel{\centerdot}{-} y) \chi_{\geq}(x,y)  +
( y \stackrel{\centerdot}{-} x) (1 - \chi_{\geq}(x,y)) .\]
\end{esempio}
%
%
\begin{esempio}[] $f(\overrightarrow{x}) = \left\{ \begin{array}{ll}
	g_1(\overrightarrow{x})\ se\ vale\  R(\overrightarrow{x})\\
	g_2(\overrightarrow{x})\ altrimenti
\end{array}\right.$ con \\$\chi_R(\overrightarrow{x}) = \left\{
\begin{array}{ll}
	1\ se\ R(\overrightarrow{x})\ vale\\
	0\ altrimenti
\end{array}\right.$ con $g_1(\overrightarrow{x})$, $g_2(\overrightarrow{x})$
e $\chi_R(\overrightarrow{x})$ primitive ricorsive.\\
Allora
\[ f(\overrightarrow{x}) = g_1(\overrightarrow{x}) \chi_R(\overrightarrow{x}) +
 g_2(\overrightarrow{x}) (1 - \chi_R(\overrightarrow{x})). \]
\end{esempio}
%

% Alessandro Onnivello
\section{Dalle funzioni primitive ricorsive alle funzioni ricorsive}

Nel paragrafo precedente abbiamo definito l'insieme delle funzioni
primitive ricorsive; possiamo dire che queste siano sufficienti per
rappresentare tutte le funzioni calcolabili da una macchina di Turing?
La risposta è negativa, come andremo ora a dimostrare.

\begin{prop}\label{PRsonoTotali}
Le funzioni primitive ricorsive sono totali.
\end{prop}
\begin{proof}
Si dimostra per induzione sulla struttura delle funzioni primitive
ricorsive.  Poichè le funzioni di base sono funzioni totali dobbiamo
verificare che la composizione e la ricorsione primitiva preservano la
totalità delle funzioni composte.

Per la composizione dobbiamo verificare che la funzione composta
$f(g_1, \dots, g_k)$ è totale. Per ipotesi induttiva
(strutturale) $f,g_1,\dots,g_k$ sono totali. Per la totalità di
$g_1,\dots,g_k$ tutti gli argomenti di $f$ sono definiti e, poichè $f$
è totale, anche la composizione è definita.

Per la ricorsione primitiva dobbiamo dimostrare che
$h:\mathbb{N}^{k+1} \to \mathbb{N}$ è definita $\forall y \in
\mathbb{N}$ e $\forall \vec{x} \in \mathbb{N}^k$.

$$h(\vec{x},y)=\left\{
\begin{array}{ll} h(\vec{x}, 0) = f(\vec{x})\\
                  h(\vec{x}, s(y)) = g(\vec{x},y,h(\vec{x},y))
\end{array} \right.$$

Si dimostra per induzione su $y$.

$y = 0$: $h(\vec{x}, 0) = f(\vec{x})$\\ $f$ è totale per ipotesi
induttiva quindi $\forall \vec{x} \in \mathbb{N}^k (\vec{x}, 0) \in
dom(h)$ come atteso

$y + 1$: $h(\vec{x}, y + 1) = g(\vec{x}, y, h(\vec{x}, y))$\\
$h(\vec{x}, y)$ è totale per ipotesi induttiva su $y$ quindi
$h(\vec{x}, y)$ è definita. Poichè $g$ totale per ipotesi induttiva e
$h(\vec{x}, y)$ è definita allora $\forall \vec{x} \in \mathbb{N}^k \;
(\vec{x}, y + 1) \in dom(h)$.
\end{proof}

Ora, per quanto detto nella precedente proposizione, l'insieme delle funzioni
primitive ricorsive permette di rappresentare solo funzioni calcolabili totali
ma, come abbiamo visto nel primo capitolo, le macchine di Turing possono
calcolare anche funzioni parziali. Quindi possiamo affermare che le
funzioni primitive ricorsive non rappresentano tutte le funzioni Turing
calcolabili.

Ma limitandoci al caso delle funzioni totali, possiamo dire che queste
siano tutte primitive ricorsive? Anche in questo caso la risposta è
negativa come vedremo nel prossimo paragrafo.

\section{Metodo diagonale di Cantor}
Per poter dimostrare che l'insieme delle funzioni primitive ricorsive
non contiene tutte le funzioni totali calcolabili andremo ora a
introdurre il metodo diagonale (o diagonalizzazione) di Cantor.  Il
metodo di diagonalizzazione è una tecnica dimostrativa, ideata da
Georg Cantor per provare la non numerabilità dei numeri reali, che
risulta molto utile anche nell'ambito della logica matematica e della
teoria della computabilità come vedremo in seguito.

Il metodo diagonale di Cantor consiste nel costruire una funzione $\chi$
che differisce da un'insieme infinito di funzioni $\chi_0,\chi_1,\dots$
effettivamente enumerabile. Si costruisce quindi la seguente tabella
dove ogni colonna contiene una funzione unaria e le righe contengono
la sequenza dei naturali, ovvero tutti i possibili argomenti.
\begin{table}[!h]
\begin{center}
\begin{tabular}{|c|c|c|c|c}

\hline
 & $\chi_0$ & $\chi_1$ & $\chi_2$ & $\ldots$\\
\hline
0 & $\chi_0(0)$ & $\chi_1(0)$  & $\chi_2(0)$  & $\ldots$\\
\hline
1 & $\chi_0(1)$ & $\chi_1(1)$  & $\chi_2(1)$  & $\ldots$\\
\hline
2 &  $\chi_0(2)$ & $\chi_1(2)$  & $\chi_2(2)$  & $\ldots$\\
\hline
$\vdots$ & $\vdots$ &       $\vdots$  & $\vdots$  & $\ddots$\\

\end{tabular}
\end{center}
\caption{Metodo diagonale di Cantor}
\end{table}

A questo punto possiamo costruire una funzione $\chi$ in modo tale che
differisca da ogni funzione enumerata nella tabella; vogliamo quindi
costruire una funzione tale che $\forall i \; \chi(i) \neq
\chi_{i}(i)$, ovvero che differisce da ogni altra funzione elencata
nella tabella almeno sulla diagonale. Questa nuova funzione, creata
tramite una procedura effettiva, non può appartenere all'insieme di
partenza, poichè altrimenti differirebbe da sè stessa sulla diagonale.

\begin{prop}\label{TotCalcNonNum}
Data una qualunque lista \emph{effettiva} di tutte le funzioni totali
calcolabili, unarie per semplicit\`a, $f_{0},\; f_{1},\; f_{2},
\cdots$ esiste sempre una funzione $g$ totale calcolabile che non
compare nella lista.
\end{prop}
\begin{proof}
Sia $f_{0},\; f_{1},\; f_{2}, \cdots$ una lista effettiva di funzioni
totali calcolabili, allora costruiamo la funzione $g$ nel seguente
modo:

\begin{center}
$g: \mathbb{N} \to \mathbb{N}$\\
$g(x) = f_{x}(x) + 1$
\end{center}
\begin{table}[!h]
\begin{center}
\begin{tabular}{|c|c|c|c|c}

\hline
 & $f_0$ & $f_1$ & $f_2$ & $\ldots$\\
\hline
0 & $f_0(0) \mathbf{+1}$ & $f_1(0)$  & $f_2(0)$  & $\ldots$\\
\hline
1 & $f_0(1)$ & $f_1(1) \mathbf{+1}$  & $f_2(1)$  & $\ldots$\\
\hline
2 &  $f_0(2)$ & $f_1(2)$  & $f_2(2) \mathbf{+1}$  & $\ldots$\\
\hline
$\vdots$ & $\vdots$ &       $\vdots$  & $\vdots$  & $\ddots$\\
\end{tabular}
\end{center}
\caption{Funzione di diagonalizzazione $\mathbf{g}$}
\label{diagG}
\end{table}

$g$ è sicuramente totale calcolabile poichè, essendo la lista di
funzioni effettiva è quindi possibile, dato un qualsiasi $n \in
\mathbb{N}$, calcolare l'ennesima funzione, valutarla in $n$ ed
aggiungerci 1. È chiaro che $g$ non può appartenere alla lista, poichè
presa una qualsiasi colonna $n$ della tabella~\ref{diagG} che
rappresenta l'immagine dell'ennesima funzione, alla riga ennesima
$g(n) \neq f_{n}(n)$ poichè per definizione $g(n)=f_{n}(n)+1$. Perciò
non esiste alcun $n$ tale che $f_{n}(x)=g(x)$ per ogni $x \in
\mathbb{N}$ e quindi $g$ non è contenuta nella nostra lista.
\end{proof}

Fatto questo ragionamento, ne consegue che non è possibile enumerare
in modo effettivo tutte le funzioni totali calcolabili.

Possiamo usare ora la proposizone~\ref{TotCalcNonNum} per dimostrare il seguente

\begin{thm}\label{diagRic} Esiste una funzione totale calcolabile che non
è primitiva ricorsiva.
\end{thm}

\begin{proof}
Dobbiamo innanzitutto costruire una lista \emph{effettiva} di tutte le
funzioni primitive ricorsive (per semplicit\`a, supponiamole ad un
argomento).

Cerchiamo di scrivere informalmente un algoritmo che ci permetta di
listare tutte queste funzioni (in seguito vedremo come si pu\`o fare
in modo rigoroso).  Una tale enumerazione si pu\`o ottenere nello
stesso modo in cui si ottiene una lista di teoremi a partire dagli
assiomi di una teoria. Si prende prima una funzione base, poi si
``contano'' tutte le funzioni ottenute da questa mediante una sola
applicazione di composizione e/o ricorsione. Poi si prende la seconda
funzione e tutte le funzioni derivate mediante una applicazione delle
due regole da quest'ultima e/o dalle funzioni gi\`a ottenute al passo
precedente, e cos\`i via. In questa maniera non si tralascia alcuna
funzione e ad ogni passo la lista \`e finita. Abbiamo cos\`i ottenuto
una enumerazione effettiva per tutte le funzioni dell'insieme
$\mathcal{PR}$.\\ Ma per quanto detto nella
proposizone~\ref{TotCalcNonNum} sappiamo che le funzioni totali non
sono enumerabili effettivamente quindi questo implica che esiste una
funzione totale calcolabile che non appartiene l'enumerazione che
abbiamo appena definito e che quindi non appartiene all'insieme
$\mathcal{PR}$.
\end{proof}

A questo punto ci chiediamo: cosa ci manca per avere tutte le funzioni
calcolabili?
Ripensando a quanto visto fino ad ora, il problema pu\`o stare o nel pretendere
di avere una lista effettiva delle funzioni calcolabili (cosa che per\`o \`e
ragionevole a farsi: lo abbiamo visto poco fa), o nell'imporre la propriet\`a di
totalit\`a alle funzioni. Dunque vediamo se lasciando cadere questa assunzione
riusciamo a classificare tutte le funzioni calcolabili.


\section{Le funzioni ricorsive}
Avendo osservato che le funzioni primitive ricorsive non
esauriscono tutte le funzioni calcolabili, il passo successivo \`e  quello di
estenderle ulteriormente. In particolare ci sar\`a bisogno di trovare
anche funzioni parziali nella nostra definizione di ``ricorsivit\`a''. Perci\'o
introduciamo
la seguente funzione: \\

f. \textbf{minimizzazione}: sia $f: \mathbb{N}^{d+1} \to \mathbb{N}$ (anche
parziale)\\
definiamo la funzione di minimizzazione $h: \mathbb{N}^{d} \to \mathbb{N}$ come
$$ h(\vec{x}):= \mu_{y}(f(\vec{x},y)) $$
dove

\[\mu_{y}(f(\vec{x},y)) = 
\begin{cases}
il \; minimo \; y \; tale \; che: \\
\qquad (i) \; f(\vec{x},z)\downarrow \; \forall z \leq y \; e \\
\qquad (ii) \; f(\vec{x},y)=0 & \text{se y esiste,} \\
non \; definita \; & \text{se $\exists z < y \; f(\vec{x},z) \uparrow$}\\
 & \text{o se $f(\vec{x},y)\neq 0 \; \forall y \in \mathbb{N} $}
\end{cases} \]

con $\mu$ che prende il nome di operatore di minimizzazione.
\begin{thm} Le funzioni ricorsive sono b-programma-calcolabili.
\end{thm}
\begin{proof} Poich\`e abbiamo gi\`a dimostrato che le tre funzioni di base
(funzione zero, successore e proiezioni), la composizione generalizzata e la
ricorsione primitiva sono calcolabili da un b-programma ci resta solo da far
vedere che questo vale anche per la minimizzazione. Infatti se
$f:\mathbb{N}^{d+1} \to \mathbb{N}$ \`e computabile con un b-programma $\alpha$
allora possiamo eseguire la seguente assegnazione:
$$x_i \leftarrow f(\vec{x_{j}}, x_k)$$
(dove con $\vec{x_{j}}$ vogliamo rappresentare in modo compatto una lista di d
variabili di input)che significa "`dai come input $\vec{x_{j}}$ e $x_k$ al
b-programma $\alpha$ che calcola $f(\vec{x_{j}}, x_k)$ e dai output risultante
come valore a $x_i$"'. Dunque il b-programma che calcola la funzione di
minimizzazione su $\vec{x}$ \`e il seguente:\\
   \begin{mylisting}
       $x_0 \leftarrow 0$ \\
       $loop$: $\vec{x_{1}} \leftarrow \vec{x}$\\
       $x_{d+2} \leftarrow f(\vec{x_{1}}, x_0)$\\
       if $x_{d+2} = 0$ then goto $stop$\\
       $x_0 \leftarrow x_0 + 1$\\
       goto $loop$
  \end{mylisting}
\end{proof}

Una volta dimostrato che la minimizzazione è calcolabile dai
b-programmi possiamo dare la seguente
\begin{defi} Una funzione $f:\mathbb{N}^{d} \to \mathbb{N}$ (totale o
parziale) si dice \emph{ricorsiva} se si ottiene dalle funzioni
iniziali (a.--c.) applicando la composizione, la ricorsione e la
minimizzazione (d.--f.).
\end{defi}

Una cosa importante da notare \`e che l'utilizzo dell'operatore $\mu$ consente
di ottenere funzioni parziali anche a partire funzioni totali, come vedremo nei
seguenti esempi.
\begin{esempio} data $f:\mathbb{N} \to \mathbb{N}$ definita nel seguente modo:
$$f(x)= \left\{ \begin{array}{ll}
il \; min \; y \; t. \, c. \; x+y=0 & se \; \exists \; y\\
non \; definita & se \; x+y \neq 0 \; \forall y \in \mathbb{N}\\
\end{array} \right.$$
utilizzando la minimizzazione pu\`o essere scritta nel seguente modo:
$$f(x)= \mu_{y}(x+y)$$

Si noti che, essendo $x \in \mathbb{N}$, se $x \neq 0 \; f$ non \`e definita
anche se la funzione somma su cui viene applicato l'operatore di minimizzazione
\`e totale.
\end{esempio}

\begin{esempio} data $f:\mathbb{N}^{2} \to \mathbb{N}$ definita nel seguente
modo:
$$f(x,y)= \left\{ \begin{array}{ll}
x/y & se \; y \vert x\\
non \; definita & se \; y \nmid x\\
\end{array} \right.$$
pu\`o essere scritta nel seguente modo:
$$f(x,y) = \mu_{z}(\vert (z*y) - x \vert) $$
anche in questo caso la funzione $g(x,y,z)=\vert (z*y) - x \vert$ \`e totale, ma
combinata con l'operatore di minimizzazione consente di ottenere la funzione
parziale $f$.
\end{esempio}

\section{Relazioni ricorsive}
\begin{defi} $R \subseteq \mathbb{N}^{n}$ è una \emph{relazione
ricorsiva} se esiste una funzione ricorsiva $\chi_R$ che assume solo valori 0 e
1 e che soddisfa
$$\chi_R(x_1, \ldots, x_n):= \left \{ \begin{array}{ll}
                                      1 & (x_{1}, x_{2}, \cdots, x_{n}) \in R\\
                                      0 & (x_{1}, x_{2}, \cdots, x_{n}) \not \in
 R\\
                                      \end{array} \right. $$
\end{defi}
\begin{prop} Se $R, S \subseteq \mathbb{N}^{n}$ sono relazioni
ricorsive allora anche $R\land S$, $R \vee S$, $\neg R$ sono relazioni
ricorsive. Inoltre, se $w \in \mathbb{N}$, anche $\forall y \leq
w. R(x_1, \ldots, x_{n-1}, y)$ e $\exists y \leq w. R(x_1, \ldots,
x_{n-1}, y)$ sono relazioni ricorsive.
\end{prop}
\begin{proof} Si pone
\begin{itemize}
\item $\chi_{R\land S} = \chi_R \chi_S$;
\item $\chi_{R\vee S} = \chi_R + \chi_S - \chi_R  \chi_S$;
\item $\chi_{\neg R} = 1 - \chi_R$;
\item $\chi_{\forall y
\leq w. R(x_1, \ldots, x_{n-1}, y)} = \prod_{i=0}^{w} \chi_R(x_1, \ldots,
x_{n-1},i)$;
\item $\chi_{\exists y
\leq w. R(x_1, \ldots, x_{n-1}, y)} = sgn(\sum_{i=0}^{w} \chi_R(x_1, \ldots,
x_{n-1},i))$
\end{itemize}
\end{proof}

Grazie alla definizione data sopra, ci \`e possibile giustificare la
\emph{definizione per casi} delle funzioni ricorsive.

Supponiamo di avere una funzione $f: \mathbb{N}^{n} \to n$ t. c.
$$f(\vec{x}):= \left \{ \begin{array}{ll}
                    g_1(\vec{x}) & $se $ \vec{x} \in R_{1}\\
                    \vdots \\
                   g_k(\vec{x}) & $se $ \vec{x} \in R_{k}\\
                    \end{array} \right.$$
con $g_{1}, g_{2}, \cdots, g_{k}:\mathbb{N}^{n} \to \mathbb{N}$ funzioni
ricorsive e dove $R_{1}, R_{2}, \cdots, R_{k} \subseteq \mathbb{N}^{n}$ sono
relazioni ricorsive mutualmente esclusive ed esaustive, ossia abbiamo che:
\begin{itemize}
\item $\sum_{i=1}^{k} \chi_{R_{i}}(\vec{x})=1$ e che
\item se $R_{i}(\vec{x})$ vale allora $\forall j \neq i \; \chi_{R_{i}}(\vec{x})
\cdot \chi_{R_{j}}(\vec{x})=0$.
\end{itemize} 
Possiamo quindi prendere le funzioni caratteristiche $\chi_{R_{1}},
\chi_{R_{2}}, \cdots, \chi_{R_{k}}$ e definire la funzione $f$ come:
$$f(\vec{x})= \sum_{i=1}^{k} g_i(\vec{x})\chi_{R_{i}}(\vec{x})$$ e
quindi essendo $f$ ottenuta per composizione di funzioni ricorsive \`e
anch'essa ricorsiva.  Quindi se vogliamo descrivere una funzione che
assume valori diversi in casi diversi lo possiamo fare mantenendoci
all'interno delle funzioni primitive ricorsive. Questo può permetterci
di definire funzioni come quella del prossimo esempio.

\begin{esempio} Data la seguente funzione:
\[ f: \mathbb{N}^{2} \to \mathbb{N} \]
\[ f(x,y)= \begin{cases}
x-y	& \text{se $x \geq y$}\\
x+y & \text{se $x<y$}
\end{cases}\]
la sua definizione per casi \`e la seguente:\\
$f(x,y) = \sum_{i=1}^{2} g_{i} \cdot \chi_{R_{\geq}}(x,y) = (x-y)\cdot
\chi_{R_{\geq}}(x,y) + (x+y)\cdot \neg \chi_{R_{\geq}}(x,y) $
Si lascia come esercizio al lettore il trovare $\chi_{R_{\geq}}$.
\end{esempio}

% Alessandro Bruni

\section{Equivalenze}

Per riassumere il percorso seguito finora presentiamo il seguente schema.
Ad ogni passaggio abbiamo introdotto un linguaggio pi\`u
raffinato senza per\`o aggiungere nuove funzioni:
 ogni volta abbiamo aggiunto qualcosa che fosse definibile in
termini del precedente e quindi in particolare di macchine di Turing.
\begin{align*}
{\rm Macchine\ }&{\rm di\ Turing\ }\\
\bigcup &|\\
{\rm Macchine\ }&{\rm a\ registri\ }\\
\bigcup &|\\
{\rm Progra}&{\rm mmi}\\
\bigcup &|\\
{\rm Funzioni\ }&{\rm ricorsive\ }\\
\bigcup &| \; {\rm?}\\
{\rm Macchine\ }&{\rm di\ Turing\ }\\
\end{align*}

È valida valida l'ultima inclusione? in caso affermativo tutti i
concetti di funzioni calcolabili descritti fino ad ora sono tra loro
equivalenti. Il seguente teorema dimostra che quest'inclusione è
effettivamente valida.

\begin{thm}[Le funzioni Turing-calcolabili sono ricorsive]
Ovvero, per ogni MdT possiamo costruire una funzione ricorsiva che la calcola.
\end{thm}

\begin{proof}
Dobbiamo innanzitutto trovare un modo per rappresentare il nastro (ovvero la
memoria) della MdT. Per semplicità di esposizione poniamoci nel caso $\sigma =
\{ B, 1 \}$, poichè ogni altro alfabeto finito può avere una rappresentazione
simile nei naturali, scegliendo una rappresentazione in base $|\sigma|$
dell'input.

Una MdT computa una funzione $f: \mathbb{N}^k \rightarrow \mathbb{N}$,
possibilmente parziale, quindi lo stato del nastro all'inizio della
computazione è una sequenza di caratteri che contiene le variabili in
input:
$$\dots BB\overline{x_1+1}B\overline{x_2+1}B...B\overline{x_k+1}BB\dots $$
mentre alla fine del calcolo la macchina dovrebbe lasciare nel nastro
il risultato della funzione che calcola:
$$\dots BB\overline{f(x_1,\dots,x_k)+1}BB\dots $$

Diamo ora le seguenti definizioni, per meglio comprendere la nostra codifica
del nastro in $\mathbb{N}$.

\begin{defi}[Numerale sinistro]
Si definisce numerale sinistro la sequenza di caratteri, letti da sinistra a
destra a partire dal primo 1, che si trovano strettamente a sinistra della
testina della MdT, dove 1 rappresenta la cifra 1 e $B$ rappresenta la cifra 0.
\end{defi}
\begin{defi}[Numero sinistro]
Si dice numero sinistro l'interpretazione nei naturali del numerale sinistro,
ovvero quel numero $n \in \mathbb{N}$ la cui rappresentazione in base 2 è
esattamente il numerale sinistro.
\end{defi}
\begin{defi}[Numerale destro]
È la sequenza di caratteri a destra della testina (compreso quello sopra la
testina stessa) lette da destra verso sinistra, a partire dal primo 1 che
compare sul nastro.
\end{defi}
\begin{defi}[Numero destro]
È l'interpretazione nei naturali del numerale destro.
\end{defi}

È facile comprendere da queste definizioni che il nastro e la testina della
macchina di Turing possono essere rappresentati da una coppia di interi, ovvero
il numero sinistro e il numero destro. Inoltre il carattere esattamente sotto
la testina è 1 se il numero destro è dispari, 0 se pari.

\begin{esempio}
Per questa macchina di Turing
\begin{align*}
\dots BB11B&111B1111BB \dots\\
&\,\uparrow
\end{align*}
il numerale sinistro è 1101, il numero sinistro è 13; il numerale destro è
1111011, il numero destro è 123.
\end{esempio}

Ora dobbiamo poter simulare su questa coppia di numeri le operazioni che la
macchina di Turing svolge sul nastro. Queste sono:
\begin{enumerate}
 \item scrivere un 1 nella casella sulla quale sta la testina;
 \item scrivere uno 0 sulla casella;
 \item spostare la testina a destra (R);
 \item spostare la testina a sinistra (L);
\end{enumerate}

Per le prime due è facile trovare una funzione ricorsiva che modifica il
numero destro in modo da rappresentare la scrittura di un 1 o di uno 0: nel
primo caso basta sommare 1 se il numero è pari (ovvero nella posizione
sottostante alla testina della MdT simulata vi è uno 0), nel secondo caso è
sufficente sottrarre 1 se il numero è dispari.

Definiamo quindi le funzioni che svolgono queste due operazioni:
$$w_1(x) = \left\{
\begin{array}{ll}
x+1 & \text{se } 2 \mid x\\
x & \text{se } 2 \nmid x
\end{array}
\right.$$
che simula l'operazione di scrivere un 1, e
$$w_B(x) = \left\{
\begin{array}{ll}
x & \text{se } 2 \mid x\\
x-1 & \text{se } 2 \nmid x
\end{array}
\right.$$
che simula l'operazione di scrivere uno 0.

Per simulare lo spostamento della testina sul nastro bisogna osservare come
cambiano i numerali quando la testina si muove a destra o a sinistra. Siano $l$
e $r$ rispettivamente il numero sinistro e il numero destro prima dello
spostamento della testina, $l'$ ed $r'$ il numero sinistro ed il numero destro
dopo lo spostamento della testina.
\begin{enumerate}
 \item Se la testina si muove verso sinistra ed $l$ è pari allora vuol
   dire che a sinistra della testina c'è uno 0, che passa da cifra
   meno significativa del numerale sinistro ad essere la cifra meno
   significativa del numerale destro.  Quindi $l' = l/2$ ed $r' = r
   \cdot 2$.
 \item Se la testina si muove verso sinistra ed $l$ è pari allora vuol
   dire che a sinistra della testina c'è un 1, che passa dal numerale
   sinistro al numerale destro; quindi $l' = (l-1)/2$ ed $r' = r \cdot
   2 + 1$.
 \item Se la testina si muove verso destra e la cifra meno
   significativa del numerale destro è un 1, questa passa al numerale
   sinistro per cui: $l' = l \cdot 2 + 1$, $r' = (r-1)/2$.
 \item Se la testina si muove verso destra e la cifra meno
   significativa del numerale destro è uno 0, $l' = l \cdot 2$ e $r' =
   r/2$.
\end{enumerate}

Vediamo ora come codificare le quintuple $\langle q_i, S_i, S_j,
\{L,R\}, q_j \rangle$ che descrivono una macchina di Turing.

Queste definiscono una funzione di transizione $\tau : \mathbb{N}^2
\rightarrow \mathbb{N}^3$ e, poichè le nostre funzioni ricorsive sono
tutte della forma $f:\mathbb{N}^k \rightarrow \mathbb{N}$, dobbiamo
trovare una funzione di codifica delle ennuple $\nu: \mathbb{N}^n
\rightarrow \mathbb{N}$ ed una classe di funzioni di proiezione
$\nu_i: \mathbb{N} \rightarrow \mathbb{N}$ che estrapolano la
proiezione dell'i-esima componente dall'ennupla codificata nei
naturali.

Delle funzioni adatte a questo scopo sono le seguenti:
$$\nu(\vec{x}) = \prod_{i=1}^n p_i^{x_i}$$ per la codifica
e $$\nu_i(w) = \max_x.p_i^x \mid w$$ per la decodifica dell'i-esima
componente, dove $p_i$ indica l'i-esimo numero primo. Queste funzioni
sfruttano l'unicità della fattorizzazione in numeri primi dei
naturali.

La funzione di codifica è chiaramente primitiva ricorsiva, in quanto
composizione di funzioni primitive ricorsive, ovvero il prodotto e
l'elevamento a potenza.

Per verificare che anche la funzione di proiezione è primitiva
ricorsiva dobbiamo introdurre gli operatori di minimizzazione e
massimizzazione limitata, rispettivamente $\mu_{w<n}[f](\vec{x})$ e
$\max_{w<n}[f](\vec{x})$, che calcolano il minimo e il massimo $w \in
[0,n]$ tale per cui $f(\vec{x}, w) = 0$.

$$\mu_{w<n}[f](\vec{x}) = \sum_{i=0}^n sgn\left(\prod_{j=0}^i
f(\vec{x},w)\right)$$

Sia $h(\vec{x},y) = f(\vec{x},n-y)$, definiamo la funzione di
massimizzazione in questo modo:

$$\max_{w<n}[f](\vec{x}) = n-\mu_{w<n}[h](\vec{x})$$

Chiaramente sia la minimizzazione che la massimizzazione limitate sono
funzioni primitive ricorsive, in quanto definite in termini di altre
funzioni primitive ricorsive.

Ora definiamo la funzione caratteristica della divisione:

\begin{align*}
div(x,y) =& \left\{
\begin{array}{ll}
1 & \text{se } x \mid y\\
0 & \text{se } x \nmid y
\end{array}
\right. \\
=& \, sgn\left( \left| \mu_{w<y+1}.\left|x \cdot w - y \right| - (y+1)\right|
\right)
\end{align*}

Infine possiamo definire la funzione di proiezione nel seguente modo:
$$\nu_i(w) = \max_{x<w}.\left| div(p_i^x, w) - 1\right|$$ e quindi
abbiamo dimostrato che anche la massimizzazione è primitiva rcorsiva.

Con gli strumenti che ci siamo appena procurati possiamo ora costruire
la codifica della funzione di transizione. Intanto ci serve una mappa
dei simboli che usiamo per descrivere la macchina di Turing nei
naturali:
\begin{align*}
q_i &\rightsquigarrow i+1\\
B &\rightsquigarrow 0\\
1 &\rightsquigarrow 1\\
L &\rightsquigarrow 1\\
R &\rightsquigarrow 2\\
\end{align*}

La funzione di transizione, dato un insieme $S$ di regole per la
macchina di Turing, è la seguente:
$$\tau(x,y) = \left\{
\begin{array}{ll}
\nu(u,v,w) & \text{se $\langle q_i, S_i, S_j, D, q_j \rangle \in
  S$}\\ & \text{e $q_i \rightsquigarrow x$, $S_i \rightsquigarrow y$,
  $S_j \rightsquigarrow u$, $D \rightsquigarrow v$, $q_j
  \rightsquigarrow w$}\\ \nu(0,0,0) & \text{se non si applica nessuna
  regola}
\end{array}
\right.$$ è una funzione definita per casi su un insieme finito di
regole $S$, per cui, come dimostrato nei paragrafi precedenti, è
primitiva ricorsiva.

Codifichiamo ora la funzione che esegue sul nastro la tripla ottenuta
dalla funzione di transizione. Siano $left(l,r)$ e $right(l,r)$ le
funzioni che simulano lo spostamento della testina rispettivamente a
sinistra ed a destra secondo le regole date in precedenza. Un passo
della MdT è codificato dalla seguente funzione:

$$step(l,r,x) = \left\{
\begin{array}{ll}
left(l, w_0(r)) \cdot p_3^{\nu_3(x)} & \text{se }\nu_1(x) = 0, \nu_2(x) = 1\\
left(l, w_1(r)) \cdot p_3^{\nu_3(x)} & \text{se }\nu_1(x) = 1, \nu_2(x) = 1\\
right(l, w_0(r)) \cdot p_3^{\nu_3(x)} & \text{se }\nu_1(x) = 0, \nu_2(x) = 2\\
right(l, w_1(r)) \cdot p_3^{\nu_3(x)} & \text{se }\nu_1(x) = 1, \nu_2(x) = 2\\
\end{array}
\right.$$

Questa funzione, definita per ricorsione primitiva, ci da lo stato
della macchina dopo $t$ passi:
\begin{align*}
f(l,r,0) &= \nu(l,r,1)\\
f(l,r,t+1) &= step(\nu_1(n), \nu_2(n), \tau(\nu_3(n), \nu_2(n) \mod 2) )
\end{align*}
con $n = f(l,r,t)$.

Vediamo come codificare una k-tupla di una funzione ricorsiva nel valore $r$.
Intanto codifichiamo un singolo elemento:
\begin{align*}
g(r,0) &= 2\cdot r + 1\\ 
g(r,x+1) &= 2\cdot f(r, x) + 1\\
\end{align*}
usando questa funzione codifichiamo la tupla:
$$cod(x_1, \dots, x_n) = g(\dots 2\cdot g(0, x_n), x_1)$$
e decodifichiamo il risultato:
$$dec(r) = \mu_x.r+1\dot{-}2^x$$

Ora non ci resta che costruire la funzione che esegue la macchina di
Turing sull'input e che ritorna la codifica del nastro alla
terminazione: $$MdT(\vec{x}) = dec(\nu_2(f(0,cod(\vec{x}),\mu_t .
|\nu_3(f(0,cod(\vec{x}),t)) = 0|)))$$

In questo modo abbiamo definito una funzione ricorsiva che svolge
l'intera computazione di una macchina di Turing, se questa termina, ed
è indefinita altrimenti. In effetti si può dire qualcosa in più: fino
alla fine della nostra costruzione abbiamo usato solo funzioni
primitive ricorsive, quindi totali per la
Proposizione~\ref{PRsonoTotali}. Ovvero è sempre possibile sapere qual
è lo stato della computazione della macchina di Turing (rappresentato
dai numerali sinistro e destro) ad un dato tempo $t$, la vera
incognita sta nella terminazione della macchina, com'era giusto
aspettarsi.

È interessante inoltre notare che la costruzione di questa funzione è
quasi del tutto indipendente dalla macchina di Turing che andiamo a
simulare con la nostra funzione ricorsiva, se non per la funzione di
transizione $\tau$ che non è altro che la rappresentazione
dell'insieme di regole date in pasto alla macchina. Potremmo dunque
benissimo pensare sostituire $\tau$ con un'altra funzione, più
sofisticata, che prende in input le quintuple codificate, lo stato
iniziale e il carattere letto e ritorna la tripla che codifica il
carattere da scrivere, la direzione in cui spostare la testina e lo
stato finale; questa funzione è in grado di prendere una qualsiasi
macchina di Turing codificata ed eseguirla su un qualsiasi
input. Inserendo questa particolare funzione di transizione e
sostituendola nel nostra prova otteniamo la macchina di Turing
universale, in grado cioè di eseguire una qualunque macchina di
Turing.

Abbiamo quindi dimostrato che per ogni macchina di Turing esiste una
funzione ricorsiva che la computa ed abbiamo visto come costruirla.
\end{proof}

\subsection{Tesi di Church-Turing}
Poichè tutti questi (ed altri) tentativi di definire tutto ciò che è
effettivamente calcolabile portano alla stessa classe di funzioni, si
può pensare che la nostra nozione intuitiva di calcolabilità coincida
esattamente con ognuna di queste definizioni.

%\include{03_insiemi_re}
%\chapter{Sistema assiomatico HA}
\section{Introduzione}

Quotidianamente l'uomo al fine di argomentare, strutturare discorsi e giungere a conclusioni fa uso della logica e di un linguaggio formale ben preciso che gli permette di evitare ambiguit\`a e ridondanze durante un colloquio.\\
Di fondamentale importanza sono le \emph{deduzioni formali} in quanto forzano la verit\`a delle nostre argomentazioni perch\`e ci permettono di definire in modo chiaro le premesse da cui partiamo e le varie regole di inferenza che utilizziamo per arrivare alla conclusione.\\
Quello che a noi interessa quindi sono le \emph{prove} che portiamo per mostrare se determinate proposizioni sono vere e perch\`e lo sono.\\
L'esempio pi\`u antico di teoria assiomatica in questo senso \`e dato dagli "Elementi" di Euclide e si pu\`o dire che, ad oggi, quasi tutta la matematica \`e sviluppata in modo assiomatico, anche se spesso non vengono specificati a quale insieme di assiomi si faccia riferimento.

Quello di cui noi abbiamo bisogno \`e una \emph{teoria assiomatica formale}, cio\`e  una teoria formale in cui \`e possibile esprimere enunciati e dedurre le loro conseguenze logiche in modo del tutto formale e meccanico.\\

Una \emph{teoria assiomatica} $T$ nel linguaggio $\mathcal{L}$ \`e determinata da un calcolo logico $L$ e da un insieme $\Sigma$ (anche infinito) di formule di $\mathcal{L}$ chiamate assiomi di $T$. Si dice che $\varphi$ \`e conseguenza logica di $\Gamma$\footnote{Con $\Gamma$ indichiamo l'insieme di tutte le assunzioni (premesse) di $T$} in $T$ se esiste una deduzione in $L$ con conclusione $\varphi$ e assunzioni contenute in $\Gamma \cup \Sigma$. In particolare, si dice che $\varphi$ \`e un teorema di $T$ se esiste una deduzione di $\varphi$ con assunzioni tutte contenute in $\Sigma$


Noi vogliamo inoltre che una teoria assiomatica soddisfi le seguenti propriet\`a:
\begin{description}
\item[Coerenza]
La coerenza, o non contraddittoriet\`a \`e la caratteristica di una teoria assiomatica di non permettere di dedurre una qualche proposizione e, contemporaneamente, la sua negazione. 
Ovviamente se la teoria \`e in grado di dimostrare una proposizione e la sua negazione la teoria non ha alcun senso, perché da quella possiamo (\textit{Ex falso quodlibet}) dedurre qualsiasi proposizione, quindi la teoria perde di significato.
\item[Decidibilit\`a degli assiomi] Significa che deve esistere una procedura effettiva per poter stabilire se una qualche proposizione \`e o meno un assioma della teoria.
\item[Indipendenza] Questa propriet\`a non \`e in realta, indispensabile, quanto le altre, per\`o torna utile.
In sostanza una teoria si dice indipendente se ogni suo assioma \`e indipendente dai restanti assiomi. 
\end{description}

\vspace{0.4 cm}
Vediamo allora gli elementi fondamentali di una teoria assiomatica:
\begin{enumerate}
	\item un \textsl{\textbf{linguaggio formalizzato}} $\mathcal{L}$
\item un insieme di enunciati nel linguaggio che trattiamo come \textsl{\textbf{assiomi}} della teoria
\item un \textsl{\textbf{sistema deduttivo}} per costruire prove, ossia per derivare teoremi da assiomi

\end{enumerate}
Un linguaggio formalizzato $\mathcal{L}$ consiste di:
\begin{itemize}
\item un \underline{alfabeto} $S$ di simboli che possono essere le variabili, i connettivi e i quantificatori e i simboli relazionali
\item \underline{regole di costruzione sintattica} che permettono di costruire espressioni accettate dal sistema.
\end{itemize}
\vspace{0.5 cm}

Dal momento che le prove sono composte da un numero finito di passaggi logici, allora per ogni prova esister\`a solo un numero finito di assiomi coinvolti. \\
Specifichiamo infine che fornire una prova non significa decidere in anticipo se questa prova esiste o meno. Noi siamo interessati non ad argomentazioni semantiche ma a derivazioni nel sistema formale HA, che \`e l'equivalente \textsl{costruttivo} della teoria assiomatica di Peano, di cui parleremo tra breve.

\vspace{0.5 cm}

\underline{Perch\'e ci serve tutto questo?}\\
\vspace{0.2 cm}

Il primo teorema di incompletezza di G\"odel pone dei limiti alle teorie assiomatiche dell'aritmetica, o meglio alle teorie formali assiomatiche dell'aritmetica, il cui linguaggio di base consiste nella definizione della struttura dei numeri naturali, della somma e della moltiplicazione in $\mathbb{N}$ e in un apparato logico del primo ordine.\newline
Quindi, dati gli assiomi che regolano la struttura di una sequenza di naturali e che caratterizzano somma e prodotto esiste sicuramente almeno una proposizione $\phi$, formulata nel dato linguaggio dell'aritmetica di base, della quale non ᅵ possibile dimostrare, a partire dagli assiomi, nᅵ la sua affermazione, nᅵ la sua negazione.\footnote{Notiamo che comunque l'incompletabilit\`a non compromette l'aritmetica che c'\`e alla base.}\\
Questo fatto \`e proprio ci\`o che viene affermato nel primo teorema di G\"odel: ogni teoria aritmetica di base assiomatizzata rimane incompleta, qualsiasi sforzo si faccia per completarla con altri assiomi.\\
Quindi qualsiasi teoria matematica assiomatizzata con un'aritmetica di base costituita dalle operazioni di somma e moltiplicazione in $\mathbb{N}$ deve essere incompleta e incompletabile (consistente e con alcuni enunciati non dimostrabili).

\section{Costruzione di HA}
\subsection{Linguaggio LA}
Quello che ci poniamo di fare ora è definire il sistema assiomatico PA e di conseguenza il sistema HA. Infatti l'aritmetica di Heyting HA formalizza la teoria intuizionista dei numeri.\\ Essa ha gli stessi assiomi propri dell'aritmetica classica PA, a cui si aggiunge l'apparato deduttivo della logica intuizionista. \\I teoremi di HA sono quindi necessariamente un sottoinsieme proprio dei teoremi di PA. L'assioma di induzione appartiene sia a PA che ad HA e questo riflette le posizioni della matematica intuizionista sul carattere costruttivo della successione dei numeri naturali. Come abbiamo visto prima, per far ci\`o, dobbiamo specificare il \textsl{linguaggio}, gli \textsl{assiomi} e il \textsl{sistema di prova deduttivo}. Del primo parleremo ora.
Definiamo dunque:
\begin{enumerate}
 \item \underline{\texttt{il linguaggio al primo ordine LA}}\footnote{LA sta per \textsl{linguaggio dell'aritmetica}}, il cui vocabolario comprende:
\begin{itemize}
	\item [-] la \textsl{costante} individuale $0$
	\item[-] il predicato di \textsl{uguaglianza} "`$=$"' 
	\item[-] la funzione a un solo argomento \textsl{successore} $s(x)$
	\item[-] le funzioni a due argomenti $x+y$ (\textsl{somma}) e $x*y$ (\textsl{moltiplicazione})	
	\item[-] quantit\`a numerabile di variabili $x, y, z, w\ldots$
	\item[-] connettivi logici e quantificatori
	\item[-] i segni $,$, $($, $)$
\end{itemize}
\vspace{.5cm}
Seguendo quanto detto prima, dobbiamo ora dare le regole per la costruzione di espressioni accettate. Lo faremo
definendo in modo induttivo i termini e le formule. Intuitivamente i termini sono i mattoni con cui si costruiranno le formule, mentre queste ultime saranno le espressioni sulle quali lavoreremo.

\item L'\underline{\texttt{insieme dei termini $Trm$}} \`e definito in modo induttivo come segue:
\begin{itemize}
\item[-] La costante $0$ \`e un termine;
\item[-] Ogni variabile \`e un termine;
\vspace{0.3 cm}
\item[-] 
	$\prooftree
  t_1\in Trm ,t_2\in Trm,\dots,t_n\in Trm
   \justifies
f_n(t_1,t_2,\dots,t_n) \in Trm
\endprooftree$ 
\end{itemize}
\vspace{.5cm}
\item  L'\underline{\texttt{insieme delle formule $Frm$}} \`e definito in modo induttivo come segue:
\begin{itemize}
\item[-] Ogni \textit{formula atomica} \`e una formula\footnote{una formula atomica \`e del tipo $R_n(t_1,t_2,\dots,t_n)$ in cui $t_1,t_2,\dots, t_n$ sono termini e $R_n$ \`e una relazione in LA ad $n$ argomenti}
\item[-] Date $\alpha,\beta \in Frm$, $\alpha \circ \beta \in Frm$, con $\circ$ un qualsiasi connettivo di LA, cio\`e $Frm$  chiuso per i connettivi sopra elencati.
\item[-] 
	$\prooftree
  \alpha \in Frm \qquad x \in Variabili
   \justifies
\forall x.\alpha \in Frm
\endprooftree$ 
\vspace{0.3 cm}
\item[-] 
$\prooftree
  \alpha \in Frm \qquad x \in Variabili
   \justifies
\exists x.\alpha \in Frm
\endprooftree$ 

\end{itemize}

\end{enumerate}\par\ \par\noindent
\vspace{.5cm}

E con questo abbiamo deinito il linguaggio che useremo.
\subsection{Assiomi}
Questa teoria assiomatica mira alla costruzione dei numeri naturali e all'aritmetica tutta, quindi i suoi 
assiomi si baseranno sui postulati di Peano, che sono i seguenti:
\begin{enumerate}
	\item[(P1)] $0$ \`e un numero naturale
	\item[(P2)] Se $x$ \`e un numero naturale, allora esiste un altro numero na-turale, detto \textsl{successore} di $x$, denotato con $s(x)$
	\item[(P3)] $0\neq s(x)$ per qualsiasi numero naturale $x$
	\item[(P4)] Se $s(x)=s(y)$ allora $x=y$
	\item[(P5)] Sia $P$ una certa propriet\`a sui numeri naturali. Se $P$ vale per $0$ e se per ogni numero naturale $x$ che gode della propriet\`a si ha che $s(x)$ gode di $P$, allora $P$ vale per ogni naturale (non \`e altro che il \textsl{principio di induzione})
\end{enumerate}\par\ \par\noindent

Prima di mostrare come saranno fatti gli assiomi della nostra teoria vediamo un paio di osservazioni.

La prima \`e la seguente:
per noi che abbiamo in mente gi\`a la struttura dei Naturali, tutti questi assiomi, sembrano sovrabbondanti, e saremmo portati a dire che bastano solamente gli assiomi \textbf{(P1)} e \textbf{(P2)}.
Notiamo invece che, una struttura ad anello, che parte dallo 0 e dopo un certo numero di passi (iterazioni della funzione successore) torna sullo 0, soddisfa ai due assiomi, ma \`e ben lontana dal descrivere i Naturali.

Possiamo allora introdurre l'assioma \textbf{(P3)}, per dire che 0 non pu\`o essere il successore di alcun numero.
Ancora una volta per\`o possiamo facilmente trovare una struttura che renda veri gli assiomi ma che non rappresenti i Naturali. Tale struttura per esempio \`e data da una ``corona del rosario'', ovvero una struttura come la precedente con l'aggiunta dello 0 in testa.

Ancora allora possiamo aggiungere il postulato \textbf{(P4)}, e sperare che ci basti.
Ovviamente non \`e cos\`i come mostra la seguente struttura:
consideriamo come dominio di interpretazione l'insieme dato da tutte le sequenza del tipo \textit{0,00,000,...}
e le sequenze analoghe a partire da un qualche altro elemento \textit{a,aa,aaa,...}. Come funzione successore possiamo prendere quella che data una sequenza \textit{s} dell'insieme, restituisca la pi\`u piccola di quelle sequenze che contenga propriamente \textit{s}. Si dimostra facilmente che questa struttura soddisfa a tutti i primi quattro postulati, ma non all'induzione e, naturalmente, non rappresenta i Naturali (in realt\`a rappresenta la giustapposizione di due copie di $\mathbb{N}$).

Questo per dire che dobbiamo sempre tenere a mente che quello che stiamo cercando di fare \`e di trasmettere la nostra idea di $\mathbb{N}$ ad un robot, attraverso gli assiomi, ed in conclusione possiamo dire che ci servono tutti.

\vspace{0.3cm}
La seconda osservazione richiede un paio di definizioni e di spostarci da un sistema al I ordine, ad uno del II.
\begin{defi}
Dati un insieme $A$, un elemento $a_0$ ed una funzione $s:A\rightarrow A$, la terna $(A,a_0,s)$ si dice \textit{Sistema di Peano} se soddisfa a:
\begin{itemize}
\item $ a_0 \in A $
\item $ a\in A \rightarrow s(a) \in A $
\item $ s(x)=s(y)\rightarrow x=y $
\item $ \forall x (a_0\neq s(x)) $
\item $ \forall P $ propriet\`a $ (P(a_0)\&(P(a)\rightarrow P(s(a)))\rightarrow \forall a(P(a))) $
\end{itemize}
\end{defi}

Si nota subito che questa \`e una generalizzazione degli assiomi di Peano che abbiamo visto sopra e che la terna $(\mathbb{N},0,s)$ \`e un sistema di Peano.
La domanda sorge ora spontanea: visto che abbiamo generalizzato, questa terna \`e l'unico sistema di Peano possibile?
Si vede facilmente che cos\`i non \`e, ovvero possiamo esibire un'altra terna che non sia $(\mathbb{N},0,s)$ e che soddisfi a quei cinque assiomi, per esempio $(\mathbb{P},2,n\mapsto n+2)$, dove con $ \mathbb{P} $ indichiamo l'insieme dei numeri pari.
In realt\`a esiste una corrispondenza biunivoca, che ``mantiene il successore'' tra i due sistemi che abbiamo visto.
Pi\`u precisamente diamo la seguente 
\begin{defi}
Dati due sistemi di Peano $ (A,a_0,s) $ e $ (B,b_0,t) $, si dice Isomorfismo tra i due, il dato di due funzioni:
$ f:A\rightarrow B $, $ g:B\rightarrow A $, tali che:
\begin{enumerate}
\item $ f(a_0)=b_0 $
\item $ f(s(a))=t(f(a)) $ per ogni $ a\in A $
\item $ g\circ f=1_A $ e $ f\circ g=1_B $
\end{enumerate}
e se tali funzioni esistono allora i due sistemi si dicono isomorfi.
\end{defi}

Con questa definizione ginugiamo infine al seguente 
\begin{thm}
Ogni sistema di Peano $ (A,a_0,t) $ \`e isomorfo a $ (\mathbb{N},0,s) $ (e quindi sono tutti isomorfi tra loro).
\end{thm}

La dimostrazione \`e semplice e lasciata per esercizio, considerando le due funzioni:
$ f:\mathbb{N}\rightarrow A $ che manda lo 0 in $ a_0 $ ed il numero $ k $ nell'elemento $ tt...t(a_0) $ k-volte e
$ g:A\rightarrow \mathbb{N} $ costruita in maniera totalmente speculare. (L'unica nota a cui stare attenti \`e che sembra non ovvio che si possa scrivere ogni elemento di $ A $ come risultato dell'applicazione della funzione successore un numero finito di volte a partire da $ a_0 $; ma questo in realt\`a si pu\`o dimostrare per induzione prendendo proprio come propriet\`a questa appena descritta.)

\vspace{0.3cm}
Questa bella propriet\`a di $ \mathbb{N} $, ovvero di essere l'unico modello a meno di isomorfismi, non \`e pi\`u valido in una teoria al prim'ordine, come quella che stiamo costruendo noi. La dimostrazione non la daremo qui, ma \`e facilmente reperibile [\textit{cfr Zanardo, Metodi assiomatici e teoria degli insiemi, dispense}]

\vspace{0.5cm}
Siamo giunti finalmente a mostrare quali saranno i nostri assiomi.
Gli \underline{\textbf{assiomi}} di HA dunque sono:
\begin{enumerate}
	\item[(A1)] $\vdash 0\neq s(x)$ (corrisponde a (P3))
	\item[(A2)] $s(x)=s(y)\vdash x=y$ (corrisponde a (P4))
	\item[(A3)] $\vdash x+0=x$
	\item[(A4)] $\vdash x+s(y)=s(x+y)$
	\item[(A5)] $\vdash x*0=0$
	\item[(A6)] $\vdash x*s(y)=x*y+x$
	\item[(A7)] Regola di induzione (corrisponde a (P5)):\newline\newline
	$\prooftree
  \Gamma, \varphi(z) \vdash \varphi(s(z))
   \justifies
 \Gamma, \varphi(0) \vdash \varphi(t)
 \using
 {IR}
\endprooftree$.
\end{enumerate}\par\ \par\
Si noti che:
\begin{itemize}
\item (A1) afferma che $0$ non \`e l'immagine di $s$ e questo permette di escludere modelli in cui, iterando la funzione successore, si possa compiere un loop che ritorni al punto di partenza. Nel modello di Peano questo sta a indicare, ad esempio, che $0$ non \`e il "`successivo"' di nessun numero naturale
\item (A2) afferma che $s$ \`e una funzione iniettiva che permette di escludere i modelli in cui, partendo da $0$ e andando avanti ripetutamente da un elemento al successore, si possa tornare su un elemento gi\`a visitato e rimanere confinati in un ciclo. In altre parole si sta dicendo ad esempio che se $s$ sta ad indicare il "`successivo"' di un numero naturale, allora se $s(x)$ e $s(y)$ coincidono, anche i "`numeri"' di partenza necessariamente coincidono
\item (A3) e (A4) sono gli assiomi che definiscono la somma tra due numeri naturali
\item (A5) e (A6) definscono il prodotto tra due numeri naturali
\item (A7) permette di affermare che l'insieme dei naturali $\mathbb {N}$ \`e il pi\`u piccolo insieme che contenga lo $0$ e che contenga il successore di ogni suo elemento (cio\`e che sia chiuso per la funzione s) e quindi permette di escludere modelli in cui siano presenti degli elementi "`intrusi"' al di fuori della sequenza infinita dei successori dello zero
\end{itemize}

\par\ \par\noindent
\underline{\texttt{Osservazione}} Non sono stati dati, almeno esplicitamente, assiomi corrispondenti ai postulati (P1) e (P2). In realt\`a, essi seguono dal fatto che consideriamo $\mathbb{N}$ come dominio di interpretazione e dall'aver definito nel nostro linguaggio la costante $0$ ($0$ \`e automaticamente elemento di $\mathbb{N}$) e la funzione a un solo argomento successore (che vogliamo sia totale, dunque se $x$ \`e un elemento, anche $s(x)$ lo \`e).
\par\ \par\noindent

\vspace{0.5 cm}
Inoltre essendo HA una teoria con uguaglianza, si hanno anche gli assiomi:
\begin{enumerate}
	\item[(U1)] $x=y,x=z\vdash y=z$
	\item[(U2)] $x=y\vdash s(x)=s(y). $
\end{enumerate}
\par\ \par

Potrebbe sembrare che il linguaggio di HA sia povero di espressivit\`a e non ci permetta di parlare di una stringa finita di numeri ma, una volta che abbiamo l'esponenziazione, possiamo semplicemente \textsl{codificare} una sequenza finita di numeri.\\ G\"odel ha mostrato che l'esponenziazione \`e definibile tramite l'introduzione di qualche altra funzione nella nostra aritmetica (la $\beta$-function di cui si parler\`a in seguito). In realt\`a \`e stato provato che essa pu\`o essere introdotta in HA e questo ci permette dunque di codificare sequenze.\\ Quindi potrebbe sembrare che HA dotato di esponenziazione sia pi\`u forte di HA, ma a dire il vero, per quanto appena accennato, sono equivalenti e dunque possiamo eliminare l'esponenziazione dalle nostre assunzioni.\\ Oltre alla potenza di questo linguaggio, verr\`a mostrata pi\`u avanti la consistenza di HA (secondo teorema di G\"odel).  Noi dunque considereremo il sistema assiomatico HA senza estenderlo. \newline
Vedremo come sar\`a possibile definire i concetti di \textsl{numero primo} (che insieme all'esponenziazione \`e comodissimo per codificare, in particolare, sequenze finite di numeri naturali), \textsl{resto della divisione}, \textsl{sottrazione adeguata}, \textsl{divisibilit\`a}, \textsl{relazione d'ordine} e cos\`i via, giusto rimanendo nell'ambito di questa teoria formale. Quindi vedremo come in questo sistema assiomatico sia possibile \textsl{derivare} tutto ci\`o che ci serve per fare matematica. \newline
Si utilizzer\`a il sistema di regole che si basa sul calcolo dei sequenti.

\vspace{0.3 cm}
\subsection{Regole del calcolo dei sequenti}
Il nome "`sequente"' \`e la traduzione dal tedesco della parola {\itshape Sequenz}, introdotta da Gerhard Gentzen nel 1934, ed indica un formalismo che si presenta nella forma:
$$\Gamma_1,\Gamma_2,\dots,\Gamma_m\vdash\Delta_1,\Delta_2,\dots,\Delta_n$$
Il simbolo $\vdash$\ indica che le formule a sinistra (gli {\itshape antecedenti}) sono le ipotesi dalla cui asserzione segue la dichiarazione delle formule a destra (i {\itshape conseguenti}).\\
Occorrono anche altre abbreviazioni come ad esempio:
$$\frac{\Gamma\vdash\Delta}{\Gamma'\vdash\Delta'}$$
dove la barra orizzontale \`e dunque un modo veloce per scrivere `"'ci\`o che sta sopra comporta ci\`o che sta sotto"'.\\
Ovvero quello che sta sopra la barra orizzontale deve essere assunto come premessa, mentre quello che sta sotto corrisponde alle deduzioni fatte a partire da tali premesse.\\
Quindi elenchiamo le regole del calcolo dei sequenti (nella versione intuizionista) che utilizzeremo per costruire le prove:
\begin{enumerate}
\vspace{.3cm}
\item[] \textit{Assioma}:
$$\Gamma,A\vdash A \ ax$$

\item[] \textit{Ordine o Scambio}:
$$\prooftree
  A,B \vdash \Delta
   \justifies
 B,A\vdash \Delta
 \using
 {ord}
\endprooftree$$\\
\item[] \textit{Indebolimento}:
$$\prooftree
  \Gamma \vdash \Delta
   \justifies
 \Gamma, \Sigma \vdash \Delta
 \using
 {ind}
\endprooftree$$\\

\item[] \textit{Contrazione}:
$$\prooftree
  \Gamma, A, A \vdash \Delta
   \justifies
 \Gamma, A \vdash \Delta
 \using
 {cont}
\endprooftree$$\\

\item[] \textit{Taglio}:
$$\prooftree
  \Gamma \vdash A\qquad\Gamma', A\vdash \Delta
   \justifies
 \Gamma, \Gamma' \vdash \Delta
 \using
 {cut}
\endprooftree$$\\

	\item[] \textit{Regole per}\ \&:\par\ \par\noindent
$$\prooftree
  \Gamma, A,B \vdash \Delta
   \justifies
 \Gamma, A\ \mbox{\&}\ B \vdash \Delta
 \using
 {\mbox{\&}_{left}}
\endprooftree$$
\hspace{\stretch{1}}

$$\prooftree
  \Gamma \vdash A\qquad\Gamma \vdash B
   \justifies
 \Gamma \vdash A\mbox{\&}B
 \using
 {\mbox{\&}_{right}}
\endprooftree$$\\

\item[] \textit{Regole per $\vee$}:\\
$$\prooftree
  \Gamma, A \vdash \Delta\qquad\Gamma, B \vdash \Delta
   \justifies
 \Gamma, A\vee B \vdash \Delta
 \using
 {\vee_{left}}
\endprooftree$$\\
	$\prooftree
  \Gamma \vdash A
   \justifies
 \Gamma \vdash A\vee B
 \using
 {\vee_{right}}
\endprooftree$
\hspace{\stretch{1}}
$\prooftree
  \Gamma \vdash B
   \justifies
 \Gamma \vdash A\vee B
 \using
 {\vee_{right}}
\endprooftree$
\vspace{.8cm}

\item[] \textit{Regole per $\to$}:\\
$$\prooftree
  \Gamma_1 \vdash A\qquad\Gamma_2, B \vdash \Delta
   \justifies
 \Gamma_1, \Gamma_2, A\to B \vdash \Delta
 \using
 {\to_{left}}
\endprooftree$$\\
$$\prooftree
  \Gamma, A \vdash B
   \justifies
 \Gamma \vdash A\to B
 \using
 {\to_{right}}
\endprooftree$$\\

\item[] \textit{Regole per $\neg$}:\\
$$\prooftree
  \Gamma \vdash A
   \justifies
 \Gamma, \neg A \vdash \Delta
 \using
 {\neg_{left}}
\endprooftree$$\\
$$\prooftree
  \Gamma, A \vdash \bot
   \justifies
 \Gamma \vdash \neg A
 \using
 {\neg_{right}}
\endprooftree$$\\

\item[] \textit{Regole per $\forall$}:\\
$$\prooftree
  \Gamma, A(c) \vdash \Delta
   \justifies
 \Gamma, \forall\ x\ A(x) \vdash \Delta
 \using
 {\forall_{left}}
\endprooftree$$\\
$$\prooftree
  \Gamma \vdash A(z)
   \justifies
 \Gamma \vdash \forall\ x\ A(x)
 \using
 {\forall_{right}\ \mbox{*}}
\endprooftree$$\\

\item[] \textit{Regole per $\exists$}:\\
$$\prooftree
  \Gamma, A(z) \vdash \Delta
   \justifies
 \Gamma, \exists\ x\ A(x) \vdash \Delta
 \using
 {\exists_{left}\ \mbox{*}}
\endprooftree$$\\
$$\prooftree
  \Gamma \vdash A(c)
   \justifies
 \Gamma \vdash \exists\ x\ A(x)
 \using
 {\exists_{right}}
\endprooftree$$

\end{enumerate}
\par\ \par\noindent
\vspace{.3cm}

Contrassegnamo con * le regole in cui $z$ non \`e una variabile libera nel contesto $\Gamma, \Delta$.\\

\section{Espressivit\`a di HA}
Vediamo ora cosa \`e possibile derivare nel nostro sistema assiomatico.\\
Si cercher\`a di dimostrare ogni proposizione tramite una prova, ma si ricorda che le derivazioni non sono uniche e quindi potrebbero esistere altri modi per provare un determinato enunciato.\\
Inoltre prima di cominciare si noti che, per dire che qualcosa \`e derivabile nella nostra teoria, si user\`a il simbolo $\vdash$ anzich\`e $\vdash_{HA}$, ossia si lascia sottinteso il sistema in cui si lavora, salvo specifiche particolari.\\
Valendo: $$\prooftree
\vdash\Gamma\quad\[\vdash\varphi(0)\quad\[\varphi(z),\Gamma\vdash\varphi(s(z))\justifies \varphi(0),\Gamma\vdash \varphi(t)\using{IR}\]\justifies \Gamma\vdash\varphi(t)\using{cut}\]\justifies \vdash\varphi(t)\using{cut}
\endprooftree$$
dimostreremo per tutte le propriet\`a ricorsive solamente $\varphi(0)$ e $\varphi(z)\vdash\varphi(s(z))$. \newline
Infatti, per la validit\`a di:
$$\prooftree
   \vdash \varphi(t)
   \justifies
  \vdash \forall\ x\ \varphi(x)
 \using
 {\forall_{right}}
\endprooftree$$\\
una volta derivati il passo base e il passo induttivo, abbiamo tutti gli strumenti per provare che la propriet\`a $\varphi$ vale per qualunque numero naturale.
\newline
\begin{prop}
Dati $p$, $r$, $t$ termini, in HA valgono le seguenti propriet\`a:
\begin{enumerate}
	\item[(R1)] Simmetria destra:\\ $$\prooftree
  \Gamma \vdash t=r
   \justifies
 \Gamma \vdash r=t
 \using
 {sim_{dx}}
\endprooftree$$
  \item[(R2)] Simmetria sinistra:\\ $$\prooftree
  \Gamma,t=r \vdash\Gamma'
   \justifies
  \Gamma,r=t\vdash\Gamma'
 \using
 {sim_{sx}}
\endprooftree$$
	\item[(R3)] Transitivit\`a:\\ $$\prooftree
  \Gamma \vdash p=t \qquad \Gamma'\vdash t=r
   \justifies
 \Gamma, \Gamma' \vdash p=r
 \using
 {tran}
\endprooftree$$
\end{enumerate}
\end{prop}
\textsc{Dimostrazione} Assumiamo di aver dimostrato (1.2) e (1.3)
\begin{enumerate}
\item[(R1)]
{\scriptsize{
	$$\prooftree
  \Gamma \vdash t=r \qquad t=r\vdash_{1.2} r=t
   \justifies
 \Gamma \vdash r=t
 \using
 {cut}
\endprooftree$$}}
  \item[(R2)]
{\scriptsize{$$\prooftree
	t=r\vdash_{1.2} r=t\qquad \Gamma,r=t\vdash\Gamma'
  \justifies
 \Gamma,t=r \vdash\Gamma'
 \using
 {cut}
\endprooftree$$}}
\item[(R3)]
{\scriptsize{$$\prooftree
  \Gamma' \vdash t=r \qquad  \[\Gamma\vdash p=t\qquad p=t,t=r\vdash_{1.3} p=r\justifies \Gamma, t=r\vdash_{1.3} p=r\using {cut}\]
   \justifies
 \Gamma, \Gamma' \vdash p=r
 \using
 {cut}
\endprooftree$$}}
\end{enumerate}
\hspace{\stretch{1}} $\Box$\\
Usando la propriet\`a simmetria destra si dimostrano:
\begin{enumerate}
	\item[(S1)] $\vdash s(r+p)=r+s(p)$
	\vspace{.2cm}
	\item[(S2)] $\vdash r=r+0$
	\vspace{.2cm}
	\item[(S3)] $\vdash 0+t=t$
	\vspace{.2cm}
	\item[(S4)] $\vdash s(t)\neq 0$
\end{enumerate}
\vspace{.5cm}
\begin{prop}
Per ogni termine $p,\ r,\ t$ valgono:
\begin{enumerate}
	\item[(1.1)] $\vdash t=t$        (riflessivit\`a)
	\vspace{.2cm}
	\item[(1.2)] $t=r\vdash r=t$     (simmetria)
	\vspace{.2cm}
	\item[(1.3)] $p=t,t=r\vdash p=r$ (transitivit\`a)
	\vspace{.2cm}
	\item[(1.4)] $r=t,p=t\vdash r=p$
	\vspace{.2cm}
	\item[(1.5)] $t=r\vdash t+p=r+p$
	\vspace{.2cm}
	\item[(1.6)] $\vdash t=0+t$
	\vspace{.2cm}
	\item[(1.7)] $\vdash s(t)+p=s(t+p)$
	\vspace{.2cm}
	\item[(1.8)] $\vdash t+r=r+t$         (commutativa)
	\vspace{.2cm}
	\item[(1.9)] $t=r\vdash p+t=p+r$
	\vspace{.2cm}
	\item[(1.10)] $\vdash (t+r)+p=t+(r+p)$ (associativa)
	\vspace{.2cm}
	\item[(1.11)] $t=r\vdash t*p=r*p$
	\vspace{.2cm}
	\item[(1.12)] $\vdash 0*t=0$
	\vspace{.2cm}
	\item[(1.13)] $\vdash s(t)*r=t*r+r$
	\vspace{.2cm}
	\item[(1.14)] $\vdash t*r=r*t$         (commutativa)
	\vspace{.2cm}
	\item[(1.15)] $t=r\vdash p*t=p*r$
	\vspace{.2cm}
	\item[(1.16)] $\vdash (t*r)*p=t*(r*p)$ (associativa)
	\vspace{.2cm}
	\item[(1.17)] $p_1=t_1,p_2=t_2,p_1=p_2\vdash t_1=t_2$
	\vspace{.2cm}
	\item[(1.18)] $p_1=t_1,p_2=t_2\vdash p_1+p_2=t_1+t_2$
\end{enumerate}
\end{prop}
\vspace{1cm}
\textsc{Dimostrazione}
\vspace{.2cm}
\begin{enumerate}
\item[(1.1)] [\ $\vdash t=t$\ ]:
\par
{\scriptsize{
	$$\prooftree
	\vdash_{A3} t+0=t\qquad\[\vdash_{A3} t+0=t\qquad t+0=t,t+0=t\vdash_{U1} t=t\justifies t+0=t\vdash t=t\using{cut}\]\justifies \vdash t=t\using{cut}
	\endprooftree$$}}
\\
\item[(1.2)][\ $t=r\vdash r=t$\ ]:
\par
{\scriptsize{$$\prooftree
	\vdash_{1.1} t=t\qquad t=r,t=t\vdash_{U1} r=t \justifies t=r\vdash r=t\using{cut}
	\endprooftree$$}}
%\%justifies t=t,t=r\vdash r=t\using(ord)\
\\
\item[(1.3)][\ $p=t,t=r\vdash p=r$\ ]:
\par
{\scriptsize{	$$\prooftree
	p=t\vdash_{1.2} t=p\qquad t=p,t=r\vdash_{U1} p=r\justifies p=t,t=r\vdash p=r\using{cut}
	\endprooftree$$}}
\\
	\item[(1.4)][\ $r=t,p=t\vdash r=p$\ ]:
\par
{\scriptsize{	$$\prooftree
	p=t\vdash_{1.2} t=p\qquad r=t,t=p\vdash_{1.3} r=p\justifies r=t,p=t\vdash r=p\using{cut}
	\endprooftree$$}}
\\
\item[(1.5)][\ $t=r\vdash t+p=r+p$\ ]:
\vspace{0.5cm}
\\Per induzione su $p$.
\vspace{0.3cm}
\\In questo caso abbiamo: {\scriptsize{$$\varphi(p) :=\ t+p=r+p.$$}}
\\
Passo base:
\par
{\scriptsize{$$\prooftree
	\[\vdash_{A3} t+0=t\qquad t=r\vdash_{ax} t=r\justifies t=r\vdash t+0=r\using{tran}\]\qquad \vdash_{S2}r=r+0\justifies t=r\vdash t+0=r+0\using{tran}
	\endprooftree$$}}
\\
\vspace{2cm}
\\
Passo induttivo:
\par
{\scriptsize{$$\prooftree
	\vdash_{A4} t+s(p)=s(t+p)\qquad\[t=r,t+p=r+p\vdash_{U2}s(t+p)=s(r+p)\qquad \vdash_{S1} s(r+p)=r+s(p)\justifies t=r,t+p=r+p\vdash s(t+p)= r+s(p)\using{tran}\]\justifies t=r,t+p=r+p\vdash t+s(p)=r+s(p)\using{tran}
	\endprooftree$$}}
\\
\item[(1.6)][\ $\vdash t=0+t$\ ]:
\vspace{0.5cm}
\\Per induzione su $t$:
\vspace{0.3cm}
\\Passo base: {\scriptsize{$$\vdash_{S2} 0=0+0$$}}
\\Passo induttivo:
\par
{\scriptsize{$$\prooftree
	t=0+t\vdash_{U2} s(t)=s(0+t)\qquad\vdash_{S1} s(0+t)=0+s(t)\justifies t=0+t \vdash s(t)=0+s(t)\using{tran}
	\endprooftree$$}}
\\
\item[(1.7)][\ $\vdash s(t)+p=s(t+p)$\ ]:
\vspace{0.5cm}
\\Per induzione su $p$:
\vspace{0.3cm}
\\Passo base:
\par
{\scriptsize{$$\prooftree
	\vdash_{A3} s(t)+0=s(t)\qquad\[\vdash_{S2}t=t+0\qquad t=t+0\vdash_{U2} s(t)=s(t+0)\justifies \vdash s(t)=s(t+0)\using{cut}\]\justifies \vdash s(t)+0=s(t+0)\using{tran}
	\endprooftree$$}}
	\vspace{0.5cm}
\\Passo induttivo:
\vspace{0.3cm}
\\Dimostriamo prima (1A):
\par
{\scriptsize{$$\prooftree
	\[\vdash_{A4} t+s(p)=s(t+p)\qquad t+s(p)=s(t+p)\vdash_{U2} s(t+s(p))=s(s(t+p))\justifies\vdash s(t+s(p))=s(s(t+p))\using{cut}\]\justifies \vdash s(s(t+p))=s(t+s(p))\using{sim_{dx}}
	\endprooftree$$}}
\\
quindi si ha:
\par
{\tiny{$$\prooftree
	\vdash_{A4} s(t)+s(p)=s(s(t)+p)\qquad\[s(t)+p=s(t+p)\vdash_{U2}s(s(t)+p)=s(s(t+p))\qquad \vdash_{1A} s(s(t+p))=s(t+s(p))\justifies s(t)+p=s(t+p)\vdash s(s(t)+p)=s(t+s(p))\using{tran}\]\justifies s(t)+p=s(t+p)\vdash s(t)+s(p)=s(t+s(p))\using{tran}
	\endprooftree$$}}
\\
\item[(1.8)][\ $\vdash t+r=r+t$\ ]:
\vspace{0.5cm}
\\Per induzione su $r$.
\vspace{0.3cm}
\\Passo base:
\vspace{0.3cm}
{\scriptsize{$$\prooftree
	\vdash_{A3} t+0=t\qquad \vdash_{1.6} t=0+t\justifies \vdash t+0=0+t\using{tran}
	\endprooftree$$}}
\vspace{0.5cm}
\\Passo induttivo:
\vspace{0.3cm}
{\scriptsize{$$\prooftree \vdash_{A4}t+s(r)=s(t+r)\qquad\[\[\vdash_{1.7}s(r)+t=s(r+t)\qquad\[t+r=r+t\vdash_{U2}s(t+r)=s(r+t)\justifies t+r=r+t\vdash s(r+t)=s(t+r)\using{sim_{dx}}\]\justifies t+r=r+t\vdash s(r)+t=s(t+r)\using{tran}\]\justifies t+r=r+t\vdash s(t+r)=s(r)+t\using{sim_{dx}}\]\justifies t+r=r+t\vdash t+s(r)=s(r)+t\using{tran}
	\endprooftree$$}}
\\
\item[(1.9)][\ $t=r\vdash p+t=p+r$\ ]:
\vspace{0.5cm}
\\Per induzione su $p$.
\vspace{0.3cm}
\\Passo base:
\par
{\scriptsize{	$$\prooftree
	\vdash_{S3} 0+t=t\qquad\[\[\vdash_{S3} 0+r=r\qquad t=r\vdash_{1.2}r=t\justifies t=r\vdash 0+r=t\using{tran}\]\justifies t=r\vdash t=0+r\using{sim_{dx}}\]\justifies t=r\vdash 0+t=0+r\using{tran}
	\endprooftree$$}}
	\vspace{1cm}
\\Passo induttivo:
\vspace{0.3cm}
{\scriptsize{$$\prooftree
	\vdash_{1.7}s(p)+t=s(p+t)\qquad\[t=r,p+t=p+r\vdash_{U2}s(p+t)=s(p+r) \qquad\[\vdash_{1.7} s(p)+r=s(p+r)\justifies \vdash s(p+r)=s(p)+r\using{sim_{dx}}\]\justifies t=r,p+t=p+r\vdash s(p+t)=s(p)+r\using{tran}\]\justifies t=r,p+t=p+r\vdash s(p)+t=s(p)+r\using{tran}
	\endprooftree$$}}
\\
\item[(1.10)][\ $\vdash (t+r)+p=t+(r+p)$\ ]:
\vspace{0.5cm}
\\Per induzione su $p$.
\vspace{0.3cm}
\\Passo base:
\par
{\scriptsize{$$\prooftree
	\vdash_{A3} (t+r)+0=t+r\qquad \[\vdash_{S2} r=r+0\qquad r=r+0\vdash_{1.9}t+r=t+(r+0)\justifies \vdash t+r=t+(r+0)\using{cut}\]\justifies\vdash (t+r)+0=t+(r+0)\using{tran}
	\endprooftree$$}}
\vspace{0.5cm}
\\Passo induttivo:
\vspace{0.3cm}
\\Dimostriamo prima (1B)
\vspace{0.3cm}
  {\scriptsize{$$\prooftree
  \vdash_{A4} (t+r)+s(p)=s((t+r)+p)\qquad (t+r)+p=t+(r+p)\vdash_{U2} s((t+r)+p)=s(t+(r+p))\justifies (t+r)+p=t+(r+p)\vdash (t+r)+s(p)=s(t+(r+p))\using{tran}
	\endprooftree$$}}
	\vspace{3cm}
\\e (1C):
\vspace{0.2cm}
  {\tiny{$$\prooftree
  \vdash_{S1} s(t+(r+p))=t+s(r+p)\qquad\[\vdash_{S1} s(r+p)=r+s(p)\qquad s(r+p)=r+s(p)\vdash_{1.9} t+s(r+p)=t+(r+s(p))\justifies \vdash t+s(r+p)=t+(r+s(p))\using{cut}\]\justifies \vdash s(t+(r+p))=t+(r+s(p))\using{tran}
	\endprooftree$$}}
	\vspace{0.5cm}
\\quindi otteniamo:
\vspace{0.3cm}
{\scriptsize{	$$\prooftree
	(t+r)+p=t+(r+p)\vdash_{1B} (t+r)+s(p)=s(t+(r+p))\qquad\vdash_{1C} s(t+(r+p))=t+(r+s(p))\justifies (t+r)+p=t+(r+p)\vdash (t+r)+s(p)=t+(r+s(p))\using{tran}
	\endprooftree$$}}
\vspace{.3cm}
\item[(1.11)] [\ $t=r\vdash t*p=r*p$\ ]:
\vspace{.2cm}
\\Per induzione su $p$, analogamente a (1.5).
\item[(1.12)] [\ $\vdash 0*t=0$\ ]:
\vspace{.2cm}
\\Per induzione su $t$, analogamente a (1.6).
\vspace{0.5cm}
\item[(1.13)] [\ $\vdash s(t)*r=t*r+r$\ ]:
\vspace{.2cm}
\\Per induzione su $r$, analogamente a (1.7).
\vspace{0.5cm}
\item[(1.14)] [\ $\vdash t*r=r*t$ \ ]:
\vspace{.2cm}
\\Per induzione su $r$, analogamente a (1.8).
\vspace{0.5cm}
\item[(1.15)] [\ $t=r\vdash p*t=p*r$\ ]:
\vspace{.2cm}
\\Per induzione su $p$, analogamente a (1.9).
\vspace{0.5cm}
\item[(1.16)] [\ $\vdash (t*r)*p=t*(r*p)$\ ]:
\vspace{.2cm}
\\Per induzione su $p$, analogamente a (1.10).
\vspace{0.5cm}
\item[(1.17)] [\ $p_1=t_1,p_2=t_2,p_1=p_2\vdash t_1=t_2$\ ]:
\vspace{.2cm}
\\Si dimostra usando (1.3), (1.4) e (U1).
\vspace{0.5cm}
\item[(1.18)] [\ $p_1=t_1,p_2=t_2\vdash p_1+p_2=t_1+t_2$\ ]:
\vspace{.2cm}
\\Come il punto precedente\end{enumerate}
\hspace{\stretch{1}} $\Box$\\


\newpage
\begin{prop}
Per ogni termine $p,\ r,\ t,$ valgono:
\begin{enumerate}
	\item[(2.1)] $\vdash t*(r+p)=(t*r)+(t*p)$ (distributiva)
	\vspace{.2cm}
	\item[(2.2)] $\vdash (r+p)*t=(r*t)+(p*t)$ (distributiva)
	\vspace{.2cm}
	\item[(2.3)] $t+p=r+p\vdash t=r$ (cancellativa)
\end{enumerate}
\end{prop}
\vspace{.5cm}
\textsc{Dimostrazione}
\vspace{.2cm}
\begin{enumerate}
\item[(2.1)] [\ $\vdash t*(r+p)=(t*r)+(t*p)$\ ]:
\vspace{.5cm}
\\Per induzione su $p$.
\vspace{0.5cm}
\\Passo base:
\vspace{.2cm}
\\Dimostriamo prima (2A)
\vspace{.2cm}
	{\scriptsize{$$\prooftree
	\[\[\vdash_{A5}t*0=0\qquad t*0=0\vdash_{1.9}t*r+t*0=t*r+0\justifies\vdash (t*r)+(t*0)=t*r+0\using{cut}\]\qquad\vdash_{A3}t*r+0=t*r\justifies\vdash (t*r)+(t*0)=t*r\using{tran}\]\justifies\vdash t*r=(t*r)+(t*0)\using{sim_{dx}}
	\endprooftree$$}}\\
da cui segue
\vspace{.2cm}
	{\scriptsize{$$\prooftree
	\[\vdash_{A3}r+0=r\qquad r+0=r\vdash_{1.15}t*(r+0)=t*r\justifies\vdash t*(r+0)=t*r\using{cut}\]\quad\vdash_{2A} t*r=(t*r)+(t*0)\justifies\vdash t*(r+0)=(t*r)+(t*0)\using{tran}
	\endprooftree.$$}}
	\vspace{.5cm}
\\Passo induttivo:
\vspace{.2cm}
\\Ci serviranno inoltre:
\vspace{.2cm}
\\(2B)
{\scriptsize{	$$\prooftree
	\[\vdash_{A6}(t*s(p))=(t*p)+t\qquad (t*s(p))=(t*p)+t\vdash_{1.9}(t*r)+(t*s(p))=(t*r)+(t*p)+t\justifies\vdash  (t*r)+(t*s(p))=(t*r)+(t*p)+t\using{cut}\]
	\justifies \vdash(t*r)+(t*p)+t=(t*r)+(t*s(p))\using{sim_{dx}}
	\endprooftree$$	}}
	\vspace{.2cm}
	\\(2C)
	{\tiny{$$\prooftree
	t*(r+p)=(t*r)+(t*p)\vdash_{1.5} t*(r+p)+t=(t*r)+(t*p)+t\qquad
	\vdash_{2B}(t*r)+(t*p)+t=(t*r)+(t*s(p))\using{tran}
	\justifies t*(r+p)=(t*r)+(t*p)\vdash t*(r+p)+t=(t*r)+(t*s(p))
	\endprooftree$$	}}
	\vspace{.2cm}
	\\(2D)
	{\scriptsize{$$\prooftree
	\vdash_{A6}t*s(r+p)=t*(r+p)+t\qquad
	t*(r+p)=(t*r)+(t*p)\vdash_{2C} t*(r+p)+t=(t*r)+(t*s(p))
	\justifies t*(r+p)=(t*r)+(t*p)\vdash t*s(r+p)=(t*r)+(t*s(p))\using{tran}
	\endprooftree$$}}
	\vspace{.2cm}
	\\e (2E)
{\scriptsize{	$$\prooftree
	\vdash_{A4}r+s(p)=s(r+p)\qquad r+s(p)=s(r+p)\vdash_{1.15}t*(r+s(p))=t*s(r+p)\justifies \vdash t*(r+s(p))=t*s(r+p)\using{cut}
	\endprooftree$$}}
	\vspace{.2cm}
	\\da cui segue
	{\scriptsize{$$\prooftree
	\vdash_{2E} t*(r+s(p))=t*s(r+p)\qquad
		t*(r+p)=(t*r)+(t*p)\vdash_{2D} t*s(r+p)=(t*r)+(t*s(p))
	\justifies t*(r+p)=(t*r)+(t*p)\vdash t*(r+s(p))=(t*r)+(t*s(p))\using{tran}
	\endprooftree$$}}
	\vspace{.5cm}
\item[(2.2)] [\ $\vdash (r+p)*t=(r*t)+(p*t)$\ ]:
\vspace{.2cm}
\\Similmente al punto precedente, per induzione su $p$; oppure si sfruttano il punto precedente (2.1), la commutativit\`a del prodotto (1.14), e la (1.18).
\vspace{.5cm}
\item[(2.3)] [\ $t+p=r+p\vdash t=r$\ ]:
\vspace{.2cm}
\\Per induzione su $p$.
\vspace{.2cm}
\\Passo base:

	{\scriptsize{$$\prooftree\vdash_{A3} t+0=t\qquad\[t+0=t,t+0=r+0\vdash_{U1}t=r+0\qquad\[\vdash_{A3}r+0=r\qquad t=r+0, r+0=r\vdash_{1.3} t=r\justifies t=r+0\vdash t=r\using{cut}\]\justifies t+0=t, t+0=r+0\vdash t=r\using{cut}\]\justifies t+0=r+0\vdash t=r\using{cut}
	\endprooftree$$}}
\vspace{.5cm}
\\Passo induttivo:
\vspace{.2cm}
\\Dimostriamo prima (2F)

$$
\tiny{
\prooftree
\[\vdash_{A4} t+s(p)=s(t+p) \qquad t+s(p)=s(t+p),t+s(p)=r+s(p)\vdash_{U1} s(t+p)=r+s(p) \justifies t+s(p)=r+s(p) \vdash s(t+p)=r+s(p)\using_{cut}\]\qquad \vdash_{A4} r+s(p)=s(r+p)\justifies t+s(p)=r+s(p) \vdash s(t+p)=s(r+p)\using_{tran}\
\endprooftree}
$$\\
\vspace{.2cm}
da cui (2G):
$${\prooftree
 t+s(p)=r+s(p)\vdash_{2F} s(t+p)=s(r+p) \qquad s(t+p)=s(r+p)\vdash_{A2}t+p=r+p\justifies t+s(p)=r+s(p)\vdash t+p=r+p\using{cut}
\endprooftree}
$$
\\quindi si ha:
\vspace{.2cm}
{\scriptsize{$$\prooftree t+s(p)=r+s(p)\vdash_{2G} t+p=r+p\qquad t+s(p)=r+s(p),t=r \vdash_{ax} t=r\justifies t+p=r+p\rightarrow t=r,t+s(p)=r+s(p)\vdash t=r\using{\rightarrow_{left}}
\endprooftree$$}}
\end{enumerate}
\hspace{\stretch{1}} $\Box$


\vspace{.6cm}

\begin{defi}
Chiameremo \underline{numerali} ($\mathcal{N}$) i termini $0$,\ $s(0)$,\ $s(s(0))$,\ $s(s(s(0)))$ e cos\`i via e li denoteremo con $\overline{0},\ \overline{1},\ \overline{2},\ \overline{3}$ etc. Pi\`u precisamente $\overline{0}$ \`e $0$ e per ogni naturale $n$, $\overline{n+1}$ \`e $s(\overline{n})$, ossia il numerale $\overline{n}$, con $n$ naturale, sta per $0$ a cui si applica $n$ volte la funzione $s$. E\-qui\-va\-len\-te\-men\-te, i numerali si possono definire induttivamente tramite le regole:
\begin{enumerate}
\item[-]$0$ \`e un numerale
\item[-]se $u$ \`e un numerale, allora anche $s(u)$ lo \`e.
\end{enumerate}
\end{defi}

In pratica un numerale \`e la formalizzazione, nel nostro sistema, di un numero naturale; si noti per\`o non rientra nel linguaggio come termine, perch\'e rappresenta una abbreviazione della scrittura corretta pi\`u formale $sss\dots s(0)$.

I numerali vengono introdotti qui per poter enunciare alcune proprietᅵ in cui vengono coinvolti dei naturali specifici, e bisogna indicarli con i termini corrispondenti.
\newpage
\begin{prop}
Per ogni termine $\ p,\ r,\ t$ valgono:
\begin{enumerate}
	\item[(3.1)] $\vdash t+\overline{1}=s(t)$
	\vspace{.2cm}
	\item[(3.2)] $\vdash t*\overline{1}=t$
	\vspace{.2cm}
	\item[(3.3)] $\vdash t*\overline{2}=t+t$
	\vspace{.2cm}
	\item[(3.4)] $t+p=0\vdash t=0\ \mbox{\emph{\&}}\ p=0$
	\vspace{.2cm}
	\item[(3.5)] $t\neq 0,p*t=0\vdash p=0$
	\vspace{.2cm}
	\item[(3.6)] $t+p=\overline{1}\vdash (t=0\ \mbox{\emph{\&}}\ p=\overline{1})\ \vee\ (t=\overline{1}\ \mbox{\emph{\&}}\ p=0)$
	\vspace{.2cm}
	\item[(3.7)] $t*p=\overline{1}\vdash t=\overline{1}\ \mbox{\emph{\&}}\ p=\overline{1}$
	\vspace{.2cm}
	\item[(3.8)] $t\neq 0\vdash \exists\ y (t=s(y))$
	\vspace{.2cm}
	\item[(3.9)] $p\neq 0,t*p=r*p\vdash t=r$
	\vspace{.2cm}
	\item[(3.10)] $t\neq 0,t\neq \overline{1}\vdash \exists\ y(t=s(s(y)))$
\end{enumerate}
\end{prop}
\vspace{.5cm}
\textsc{Dimostrazione}\\
 Per la definizione di nume\-rale, possiamo sostituire il termine $\overline{n}$ con ${\overbrace{sss\dots s(0)}^{n\ volte}}$.
\begin{enumerate}
\vspace{.2cm}
	\item[(3.1)] [ $\vdash t+\overline{1}=s(t)$ ]:
	\vspace{.2cm}
{\scriptsize{	$$\prooftree
	\vdash_{A4}t+s(0)=s(t+0)\qquad \[\vdash_{A3}t+0=t\qquad t+0=t\vdash_{U2}s(t+0)=s(t)\justifies \vdash s(t+0)=s(t)\using{cut}\]\justifies \vdash t+s(0)=s(t)\using{tran}
	\endprooftree$$}}
	\vspace{.2cm}
	\item[(3.2)] [ $\vdash t*\overline{1}=t$ ]:
	\vspace{.2cm}
{\scriptsize{	$$\prooftree
	\vdash_{A6} t*s(0)=t*0+t\qquad\[\[\vdash_{A5}t*0=0\qquad t*0=0\vdash_{1.5}t*0+t=0+t\justifies\vdash t*0+t=0+t\using{cut}\]\qquad\vdash_{S3} 0+t=t\justifies \vdash t*0+t=t\using{tran}\]\justifies\vdash t*s(0)=t\using{tran}
	\endprooftree$$}}
	\vspace{.2cm}
	\item[(3.3)] [ $\vdash t*\overline{2}=t+t$ ]:
	\vspace{.2cm}
	\\Basta usare il punto precedente e A6.
	\vspace{.2cm}
	\item[(3.4)] [ $t+p=0\vdash t=0\ \mbox{\&} \ p=0$ ]:
	\vspace{.2cm}
	\\Per induzione su $p$.
	\vspace{.2cm}
\\Passo base:
\vspace{.2cm}
{\scriptsize{$$\prooftree
	\[\vdash_{A3}t+0=t\qquad t+0=t,t+0=0\vdash_{U1}t=0\justifies t+0=0\vdash t=0\using{cut}\]\qquad\[\vdash_{1.1} 0=0\justifies t+0=0\vdash 0=0\using{ind}\]\justifies t+0=0\vdash t=0\mbox{\&}\ 0=0\using{\mbox{\&}_{right}}
	\endprooftree$$}}
	\vspace{.5cm}
	\\Per il passo induttivo, dimostriamo (3A)
	\vspace{.2cm}
	{\scriptsize{$$\prooftree
	\[\vdash_{S1}s(t+p)=t+s(p)\qquad s(t+p)=t+s(p),t+s(p)=0\vdash_{1.3}s(t+p)=0\justifies t+s(p)=0\vdash s(t+p)=0\using{cut}\]\qquad\[0=s(t+p)\vdash_{A1}\bot\justifies s(t+p)=0\vdash\bot\using{sim_{sx}}\]\justifies t+s(p)=0\vdash\bot\using{cut}
	\endprooftree$$}}
	\vspace{.2cm}
	\\da cui segue:
	\vspace{.2cm}
{\scriptsize{	$$\prooftree
	\[t+s(p)=0\vdash_{3A}\bot\qquad\bot\vdash t=0\justifies t+s(p)=0\vdash t=0\using{cut}\]\[t+s(p)=0\vdash_{3A}\bot\qquad\bot\vdash s(p)=0\justifies t+s(p)=0\vdash s(p)=0\using{cut}\]\justifies t+p=0\rightarrow t=0\mbox{\&} p=0, t+s(p)=0\vdash t=0\mbox{\&} s(p)=0\using{\mbox{\&}_{right}+ind}
	\endprooftree$$}}
	\vspace{.5cm}
	\item[(3.5)] [ $t\neq 0,p*t=0\vdash p=0$ ]:
	\vspace{.2cm}
	\\Per induzione su $p$.
	\vspace{.2cm}
\\Passo base:
\vspace{.2cm}
{\scriptsize{$$\prooftree
	\vdash_{1.1} 0=0\justifies t\neq 0,0*t=0\vdash 0=0\using{ind}
	\endprooftree$$}}
	\vspace{.5cm}\\
	\\Per dimostrare il passo induttivo, proviamo prima (3B)
\vspace{.2cm}
{\scriptsize{	$$\prooftree
	p*t+t=0\vdash_{3.4} p*t=0\mbox{\&} t=0 \qquad\[\[t=0 \vdash_{ax}t=0\justifies p*t=0\mbox{\&}t=0\vdash t=0\using{\mbox{\&}_{left}}\]\justifies  \justifies p*t=0\mbox{\&} t=0, t\neq 0\vdash\bot\using{\neg_{left}}\]\justifies p*t+t=0, t\neq 0\vdash\bot\using{cut}
	\endprooftree$$}}
		\vspace{.2cm}
	\\e (3C)
	\vspace{.2cm}
{\scriptsize{$$\prooftree
	s(p)*t=p*t+t,s(p)*t=0\vdash_{U1}p*t+t=0\qquad p*t+t=0, t\neq 0\vdash_{3B}\bot\justifies s(p)*t=p*t+t,s(p)*t=0, t\neq 0\vdash\bot\using{cut}
	\endprooftree$$}}
\vspace{.2cm}
\\da cui segue:
	{\scriptsize{$$\prooftree
	\[\[\vdash_{1.13}s(p)*t=p*t+t\qquad s(p)*t=p*t+t,s(p)*t=0, t\neq 0\vdash_{3C}\bot\justifies  t\neq 0,s(p)*t=0\vdash\bot\using{cut}\]\qquad\bot\vdash s(p)=0\justifies t\neq 0,s(p)*t=0\vdash s(p)=0\using{cut}\]\justifies t\neq 0,p*t=0\rightarrow p=0, t\neq 0,s(p)*t=0\vdash s(p)=0\using{ind}
	\endprooftree$$}}
	\vspace{.5cm}
	\item[(3.6)] [ $t+p=\overline{1}\vdash (t=0\ \mbox{\&} \ p=\overline{1})\ \vee\ (t=\overline{1}\ \mbox{\&} \ p=0)$ ]:
	\vspace{.2cm}
	\\Per induzione su $p$.
	\vspace{.2cm}
\\Passo base:
\vspace{.2cm}
	{\scriptsize{$$\prooftree
	\[\[\vdash_{A3} t+0=t\qquad t+0=t,t+0=s(0)\vdash_{U1}t=s(0)\justifies t+0=s(0)\vdash t=s(0)\using{cut}\]\[\vdash_{1.1}0=0\justifies t+0=s(0)\vdash 0=0\using{ind}\]\justifies t+0=s(0)\vdash t=s(0)\mbox{\&} 0=0\using{\mbox{\&}_{right}}\]\justifies t+0=s(0)\vdash (t=0\mbox{\&} 0=s(0))\vee (t=s(0)\mbox{\&} 0=0)\using{\vee_{right}}
	\endprooftree$$}}
	\vspace{.2cm}
	\\Per il passo induttivo dimostriamo prima:
	\vspace{.2cm}
  \\(3D)
  \vspace{.2cm}
	{\scriptsize{$$\prooftree
	\vdash_{A4}t+s(p)=s(t+p)\qquad\[t+s(p)=s(t+p),t+s(p)=s(0)\vdash_{U1} s(t+p)=s(0)\qquad s(t+p)=s(0)\vdash_{A2}t+p=0\justifies t+s(p)=s(t+p),t+s(p)=s(0)\vdash t+p=0\using{cut}\]\justifies t+s(p)=s(0)\vdash t+p=0\using{cut}
	\endprooftree$$	}}
	\vspace{.2cm}
	\\(3E)
	{\scriptsize{$$\prooftree
	t+s(p)=s(0)\vdash_{3D}t+p=0\qquad\[t+p=0\vdash_{3.4}t=0\mbox{\&} p=0\qquad\[t=0\vdash_{ax} t=0\justifies t=0\mbox{\&} p=0\vdash t=0\using{\mbox{\&}_{left}}\]\justifies t+p=0\vdash t=0\using{cut}\]\justifies t+s(p)=s(0)\vdash t=0\using{cut}
	\endprooftree$$}}
	\vspace{.2cm}
	\\(3F)
	\vspace{.2cm}
{\scriptsize{	$$\prooftree
	t+s(p)=s(0)\vdash_{3D}t+p=0\qquad\[t+p=0\vdash_{3.4}t=0\mbox{\&} p=0\qquad\[p=0\vdash_{U2} s(p)=s(0)\justifies t=0\mbox{\&} p=0\vdash s(p)=s(0)\using{\mbox{\&}_{left}}\]\justifies t+p=0\vdash s(p)=s(0)\using{cut}\]\justifies t+s(p)=s(0)\vdash s(p)=s(0)\using{cut}
	\endprooftree$$}}
\vspace{.2cm}
	\\e quindi segue:
	{\tiny{$$\prooftree
	\[t+s(p)=s(0)\vdash_{3E}t=0\qquad t+s(p)=s(0)\vdash_{3F}s(p)=s(0)\justifies t+s(p)=s(0)\vdash t=0\mbox{\&} s(p)=s(0)\using{\mbox{\&}_{right}}\]\justifies t+p=s(0)\rightarrow(t=0\mbox{\&} p=s(0))\vee(t=s(0)\mbox{\&} p=0),t+s(p)=s(0)\vdash(t=0\mbox{\&} s(p)=s(0))\vee(t=s(0)\mbox{\&} s(p)=0)\using{ind+\vee_{right}}
	\endprooftree$$}}

	\vspace{.5cm}
	\item[(3.7)] [ $t*p=\overline{1}\vdash t=\overline{1}\ \mbox{\&} \ p=\overline{1}$ ]:
	\vspace{.2cm}
	\\Per induzione su $p$, usando il punto precedente.
	\vspace{.5cm}
	\item[(3.8)] [ $t\neq 0\vdash \exists\ y (t=s(y))$ ]:
	\vspace{.2cm}
	\\Per induzione su $t$.
	\vspace{.2cm}
\\Passo base:
\vspace{.2cm}
	{\scriptsize{$$\prooftree
	\[\vdash_{1.1}0=0\justifies  0\neq 0\vdash\bot\using{\neg_{left}}\]\qquad\bot\vdash\exists y(0=s(y))\justifies 0\neq 0\vdash\exists y(0=s(y))\using{cut}
	\endprooftree$$}}
	\vspace{.2cm}
\\Passo induttivo:
\vspace{.2cm}
{\scriptsize{$$\prooftree
	\[\vdash_{1.1} s(t)=s(t)\justifies\vdash\exists y(s(t)=s(y))\using{\exists_{right}}\]\justifies t\neq 0\rightarrow\exists y(t=s(y)), s(t)\neq 0\vdash\exists y(s(t)=s(y))\using{ind}
	\endprooftree$$}}
	\vspace{.5cm}
	\item[(3.9)] [ $p\neq 0,t*p=r*p\vdash t=r$ ]:
	\vspace{.2cm}
	\\Per induzione su $t$, utilizzando anche la (3.5).
	\vspace{.5cm}
	\item[(3.10)] [ $t\neq 0,t\neq \overline{1}\vdash \exists\ y(t=s(s(y)))$ ]:
	\vspace{.2cm}
	\\Per induzione su $t$, usando anche la (3.8).
\end{enumerate}
\hspace{\stretch{1}} $\Box$\\
\vspace{1cm}
\begin{prop}
Siano $m$ e $n$ due numeri naturali qualsiasi. Allora si ha:
\begin{enumerate}
  \item[(a.)] Se $m=n$ allora $\vdash \overline{m} = \overline{n}$
  \vspace{.2cm}
	\item[(b.)] Se $m\neq n$ allora $\vdash \overline{m} \neq \overline{n}$
	\vspace{.2cm}
	\item[(c.)] $\vdash \overline{m+n} = \overline{m} + \overline{n}$ e $\vdash \overline{m*n} = \overline{m} * \overline{n}$
\end{enumerate}
\vspace{.5cm}
\end{prop}
\textsc{Dimostrazione}
\begin{enumerate}
\vspace{.2cm}
	\item[(a.)] Ovvio, dalla definizione di numerale.
	\item[(b.)] Sia $p=|m-n|\neq 0$ in quanto $n\neq m$ per ipotesi. Supponendo $m>n$,
	{\scriptsize{$$\prooftree
	\[\overline{m}=\overline{n}\vdash_{A2} \overline{m-1}=\overline{n-1}\qquad\[\[s(\overline{p})=s(0)\vdash_{A2}\overline{p}=0\qquad\overline{p}=0\vdash\bot\justifies ...\]\justifies \overline{m-1}=\overline{n-1}\vdash\bot\using{A2\ (n-2)-volte}\]\justifies \overline{m}=\overline{n}\vdash\bot\using{cut}\]\justifies \vdash \overline{m}\neq \overline{n}\using{\neg_{right}}
	\endprooftree$$}}
	\vspace{.2cm}
	\item[(c.)] Dimostriamo la prima parte per induzione su $n$ (per la seconda si procede analogamente).
	\vspace{.2cm}
\\Passo base:
\vspace{.2cm}
	{\scriptsize{$$\prooftree \vdash_{A3}\overline{m+0}=\overline{m}\qquad\vdash_{S2}\overline{m}=\overline{m}+\overline{0}\justifies\vdash\overline{m+0}=\overline{m}+\overline{0}\using{tran}
	\endprooftree$$}}
	\vspace{.2cm}
\newline
Passo induttivo:
\vspace{.2cm}
{\scriptsize{$$\prooftree	 \vdash_{A4+a}\overline{m+s(n)}=\overline{s(m+n)}\qquad\[\[\vdash_{A4}\overline{m}+\overline{s(n)}=s(\overline{m}+\overline{n})\quad\[\overline{m+n}=\overline{m}+\overline{n}\vdash_{U2}\overline{s(m+n)}=s(\overline{m}+\overline{n})\justifies \overline{m+n}=\overline{m}+\overline{n}\vdash s(\overline{m}+\overline{n})=\overline{s(m+n)}\using{sim_{dx}*}\]\justifies\overline{m+n}=\overline{m}+\overline{n}\vdash\overline{m}+\overline{s(n)}=\overline{s(m+n)}\using{tran}\]\justifies\overline{m+n}=\overline{m}+\overline{n}\vdash\overline{s(m+n)}=\overline{m}+\overline{s(n)}\using{sim_{dx}}\]\justifies \overline{m+n}=\overline{m}+\overline{n}\vdash\overline{m+s(n)}=\overline{m}+\overline{s(n)}\using{tran}
	\endprooftree$$}}
	\vspace{.5cm}
\\In quest'ultima dimostrazione nel punto * si nota facilmente che vale l'uguaglianza $\overline{s(m+n)}=s(\overline{m+n})$. Infatti, per esempio $s(\overline{2})=s(s(s(0)))$ ma anche $\overline{s(2)}=s(s(s(0)))$.

\end{enumerate}
\hspace{\stretch{1}} $\Box$
\vspace{.5cm}
\\

\texttt{Osservazione}\\
\vspace{.2cm}

Ci\`o che abbiamo affermato nella proposizione 3.4 \`e di fondamentale importanza: esprime il fatto che il sistema formale HA, di fronte alle operazioni di somma e prodotto applicate ai \underline{defi}, sa procedere esattamente come siamo in grado di fare noi sui \underline{naturali}, ovvero \`e stato istruito cos\`i bene che possiamo dimostrare che calcola esattamente ci\`o che noi ci aspettiamo . Quindi, nonostante la netta distinzione tra noi e il robot, aver dimostrato questa sorta di parallelismo ci garantisce l'efficienza (non in senso computazionale, ma come coerenza, buona costruzione) del sistema formale: questo, infatti, \`e stato istruito bene a tal punto da saper riconoscere, a partire da un insieme finito di assiomi e regole, qualsiasi altra espressione con i numerali, e conseguentemente, sa agire su essa proprio come noi vogliamo. La distinzione rimane comunque marcata in quanto noi abbiamo istruito il robot, e possiamo riconoscere un senso a ciᅵ che elabora, aggiungendo l'interpretazione, mentre quest'ultimo resta solo un esecutore.

\vspace{.5cm}

\begin{defi}
Definiamo le relazioni d'ordine:
\vspace{.2cm}
\begin{itemize}
\item $t<z :=\ \exists\ w(w\neq 0\ \mbox{\emph{\&}}\ w+t=z)$ o, equivalentemente, $t<z :=\ \exists\ w(s(w)+t=z)$
\vspace{.2cm}
\item $t\leq z\ :=\ t<z\ \vee\ t=z$, cio\`e $t\leq z\ :=\ \exists\ w(w+t=z)$
\vspace{.2cm}
\item $t>z\ :=\ z<t$
\vspace{.2cm}
\item $t\geq z\ :=\ z\leq t$
\vspace{.2cm}
\item $t\not< z\ :=\ \neg(t<z).$
\end{itemize}
\end{defi}

\vspace{.5cm}
\begin{prop}
Per ogni termine $p,\ r,\ t$ valgono:
\vspace{.2cm}
\begin{enumerate}
	\item[(5.1)] $ \vdash t\not< t$ (antiriflessiva)
	\vspace{.2cm}
	\item[(5.2)] $t<p, p<r\vdash t<r$ (transitiva)
	\vspace{.2cm}
	\item[(5.3)] $t<p\vdash p\not< t$ (antisimmetrica)
	\vspace{.2cm}
	\item[(5.4)] $\vdash t<p\ \leftrightarrow\ t+r<p+r$
	\vspace{.2cm}
	\item[(5.5)] $\vdash t\leq t$
	\vspace{.2cm}
	\item[(5.6)] $t\leq p,p\leq r\vdash t\leq r$
	\vspace{.2cm}
	\item[(5.7)] $\vdash t\leq p\ \leftrightarrow\ t+r\leq p+r$
	\vspace{.2cm}
	\item[(5.8)] $t\leq p,p<r\vdash t<r$
	\vspace{.2cm}
	\item[(5.9)] $\vdash 0\leq t$
	\vspace{.2cm}
	\item[(5.10)] $\vdash 0<s(t)$
	\vspace{.2cm}
	\item[(5.11)] $\vdash t<r\ \leftrightarrow\ s(t)\leq r$
	\vspace{.2cm}
	\item[(5.12)] $\vdash t\leq r\ \leftrightarrow\ t<s(r)$
	\vspace{.2cm}
	\item[(5.13)] $\vdash t<s(t)$
	\vspace{.2cm}
	\item[(5.14)] $\vdash 0<\overline{1},\ \overline{1}<\overline{2},...$
	\vspace{.2cm}
	\item[(5.15)] $t\neq r\vdash t<r\ \vee\ r<t$
	\vspace{.2cm}
	\item[(5.16)] $\vdash t=r\ \vee\ t<r\ \vee\ r<t$
	\vspace{.2cm}
	\item[(5.17)] $\vdash t\leq r\ \vee\ r\leq t$
	\vspace{.2cm}
	\item[(5.18)] $\vdash t+r\geq t$
	\vspace{.2cm}
	\item[(5.19)] $r\neq 0\vdash t+r>t$
	\vspace{.2cm}
	\item[(5.20)] $r\neq 0\vdash t*r\geq t$
	\vspace{.2cm}
	\item[(5.21)] $\vdash r\neq 0\ \leftrightarrow\ r>0$
	\vspace{.2cm}
	\item[(5.22)] $r>0,t>0\vdash r*t>0$
	\vspace{.2cm}
	\item[(5.23)] $r\neq 0,t>\overline{1}\vdash t*r>r$
	\vspace{.2cm}
	\item[(5.24)] $r\neq 0\vdash t<p\ \leftrightarrow\ t*r<p*r$
	\vspace{.2cm}
	\item[(5.25)] $r\neq 0\vdash t\leq p\ \leftrightarrow\ t*r\leq p*r$
	\vspace{.2cm}
	\item[(5.26)] $\vdash t\not< 0$
	\vspace{.2cm}
	\item[(5.27)] $t\leq r\ \mbox{\emph{\&}}\ r\leq t\vdash t=r$
\end{enumerate}
\end{prop}

\vspace{.5cm}
\textsc{Dimostrazione}
\begin{enumerate}
	\item[(5.1)] [ $ \vdash t\not< t$ (antiriflessiva) ]:
	\vspace{.2cm}
	\\Basta applicare la definizione di $\not<$ e quanto precedentemente mostrato in (1.6) e (2.3).
	\vspace{.5cm}
	\item[(5.2)] [ $t<p, p<r\vdash t<r$ (transitiva) ]:
	\vspace{.2cm}
	\\Applicare la definizione di $<$, utilizzare i punti (1.8) e (1.5) e scegliere $w=v+z$ nell'$\exists_{right}$, dove $v$ e $z$ vengono scelte rispettivamente nell'$\exists_{left}$ in $t<p$ e $p<r$.
	\vspace{.5cm}
	\item[(5.3)] [$t<p\vdash p\not< t$ (antisimmetrica) ]:
	\vspace{.2cm}
	\\Proviamo prima (5A)

	{\scriptsize{$$\prooftree
	\vdash_{1.6}t=0+t\qquad \[v+w+t=t,t=0+t\vdash_{1.3}v+w+t=0+t\qquad v+w+t=0+t\vdash_{2.4}v+w=0\justifies v+w+t=t,t=0+t\vdash v+w=0\using{cut}\]\justifies v+w+t=t\vdash v+w=0\using{cut}
	\endprooftree$$	}}
\vspace{.5cm}
	(5B)
{\scriptsize{$$\prooftree
	v+w+t=t\vdash_{5A} v+w=0\qquad\[v+w=0\vdash_{3.4}v=0\mbox{\&} w=0\qquad\[\[v=0\vdash_{ax} v=0\justifies v=0,v\neq 0\vdash\bot\using{\neg_{left}}\]\justifies v=0\mbox{\&} w=0,v\neq 0\vdash\bot\using{\mbox{\&}_{left}}\]\justifies v+w=0,v\neq 0\vdash\bot\using{cut}\]\justifies v+w+t=t,v\neq 0\vdash\bot\using{cut}
	\endprooftree$$}}	\vspace{.2cm}
	e (5C)
	{\tiny{$$\prooftree
	p=w+t\vdash_{1.9}v+p=v+w+t\qquad\[v+p=v+w+t,v+p=t\vdash_{U1}v+w+t=t\qquad v+w+t=t,v\neq 0\vdash_{5B}\bot\justifies v\neq 0,v+p=v+w+t,v+p=t\vdash\bot\using{cut}\]\justifies p=w+t,v\neq 0,v+p=t\vdash\bot\using{cut}
	\endprooftree$$	}}
	\vspace{.2cm}
{\scriptsize{$$\prooftree
	\[\[w+t=p\vdash_{1.2}p=w+t\quad w\neq 0,p=w+t,v\neq 0,v+p=t\vdash_{5C}\bot\justifies w\neq 0,w+t=p,v\neq 0\mbox{\&} v+p=t\vdash\bot\using{cut+\mbox{\&}_{left}}\]\justifies\exists w(w\neq 0\mbox{\&}\ w+t=p),\exists v(v\neq 0\mbox{\&} v+p=t)\vdash\bot\using{\exists_{left}+\mbox{\&}_{left}}\]\justifies \exists w(w\neq 0\mbox{\&} w+t=p)\vdash\neg\exists v(v\neq 0\mbox{\&} v+p=t)\using{\neg_{right}}
	\endprooftree$$}}
	\vspace{.5cm}
	\item[(5.4)] [ $\vdash t<p\ \leftrightarrow\ t+r<p+r$ ]:
	\vspace{.2cm}
	\\Bisogna mostrare le due implicazioni \textsl{separatamente}, applicando la definizione di $<$ e scegliendo in entrambi i casi $w$ dell'$\exists_{right}$ uguale alla variabile $v$ ottenuta applicando l'$\exists_{left}$.
	\vspace{.5cm}
	\item[(5.5)] [ $\vdash t\leq t$ ]:
	\vspace{.2cm}
	\\Basta applicare la definizione di $\leq$ e scegliere, nell'$\vee_{right}$, $t=t$, dimostrata in precedenza nel punto(1.1).
	\vspace{.5cm}
	\item[(5.6)] [ $t\leq p,p\leq r\vdash t\leq r$ ]:
	\vspace{.2cm}
	\\Basta applicare la definizione di $\leq$, usando i punti (5.2) e (1.3).
	\vspace{.5cm}
	\item[(5.7)] [ $\vdash t\leq p\ \leftrightarrow\ t+r\leq p+r$ ]:
	\vspace{.2cm}
\begin{itemize}
	\item[($\leftarrow$)]
{\scriptsize{	$$\prooftree
	\[t+r<p+r\vdash_{5.4}t<p\justifies t+r<p+r\vdash t<p\vee t=p\using{\vee_{right}}\]\[t+r=p+r\vdash_{2.3}t=p\justifies t+r=p+r\vdash t<p\vee t=p\using{\vee_{right}}\]\justifies t+r<p+r\vee t+r=p+r\vdash t<p\vee t=p\using{\vee_{left}}
	\endprooftree$$}}
	\vspace{.2cm}
	\item[($\rightarrow$)] Proviamo prima (5D)
	{\scriptsize{$$\prooftree
	\[\[\[\[w\neq 0\vdash_{ax} w\neq 0\justifies w\neq 0\mbox{\&} w+t=p\vdash w\neq 0\using{\mbox{\&}_{left}}\]\quad\[w+t=p\vdash_{1.5}w+t+r=p+r\justifies w\neq 0\mbox{\&} w+t=p\vdash w+t+r=p+r\using{\mbox{\&}_{left}}\]\justifies w\neq 0\mbox{\&} w+t=p\vdash w\neq 0\mbox{\&} w+t+r=p+r\using{\mbox{\&}_{right}}\]\justifies w\neq 0\mbox{\&} w+t=p\vdash\exists v(v\neq 0\mbox{\&} v+t+r=p+r)\using{\exists_{right}}\]\justifies \exists w(w\neq 0\mbox{\&} w+t=p)\vdash\exists v(v\neq 0\mbox{\&} v+t+r=p+r)\using{\exists_{left}}\]\justifies\exists w(w\neq 0\mbox{\&} w+t=p)\vdash (t+r<p+r)\vee (t+r=p+r)\using{def.\leq }
	\endprooftree$$}}
	\vspace{.4cm}
	\\da cui segue:
	\vspace{.2cm}
	{\scriptsize{$$\prooftree
  \exists w(w\neq 0\mbox{\&} w+t=p)\vdash_{5D} (t+r<p+r)\vee (t+r=p+r)\qquad\[
t=p\vdash_{1.5}t+r=p+r\justifies t=p\vdash (t+r<p+r)\vee(t+r=p+r)\using{\vee_{right}}\]\justifies t<p\vee t=p\vdash (t+r<p+r)\vee (t+r=p+r)\using{\vee_{left}}
	\endprooftree$$}}

\vspace{.5cm}
\end{itemize}
	\item[(5.8)] [ $t\leq p,p<r\vdash t<r$ ]:
	\vspace{.2cm}
	\\Basta applicare le definizioni di $<$ e $\leq$ e ricordare il punto (5.2).
	\vspace{.5cm}
	\item[(5.9)] [ $\vdash 0\leq t$ ]:
	\vspace{.2cm}
	\\Applicando la definizione di $\leq$, e scegliere $w=t$ nell'$\exists_{right}$.
	\vspace{.5cm}
	\item[(5.10)] [ $\vdash 0<s(t)$ ]:
	\vspace{.2cm}
	\\Basta applicare la definizione di $<$, scegliendo $w=s(t)$ nell'$\exists_{right}$.
	\vspace{.5cm}
	\item[(5.11)] [ $\vdash t<r\ \leftrightarrow\ s(t)\leq r$ ]:
	\vspace{.2cm}
	\\Dimostrare i due versi \textsl{separatamente}, tenendo presente che in ($\rightarrow$) bisogna procedere per induzione su $r$.
	\vspace{.5cm}
	\item[(5.12)] [ $\vdash t\leq r\ \leftrightarrow\ t<s(r)$ ]:
	\vspace{.2cm}
	\\Basta applicare le definizioni di $<$ e $\leq$.
	\vspace{.5cm}
	\item[(5.13)] [ $\vdash t<s(t)$ ]:
	\vspace{.2cm}
	\\Basta applicare la definizione di $<$ e scegliere nell'$\exists_{right}$ $w=s(0)$.
	\vspace{.5cm}
	\item[(5.14)] [ $\vdash 0<\overline{1},\ \overline{1}<\overline{2},...$ ]:
	\vspace{.2cm}
	\\Considerare il punto precedente.
	\vspace{.5cm}
	\item[(5.15)] [ $t\neq r\vdash t<r\ \vee\ r<t$ ]:
	\vspace{.2cm}
	\\Per induzione su $r$.
	\vspace{.2cm}
\\Passo base:
\vspace{.2cm}
{\scriptsize{	$$\prooftree
	\[\[t\neq 0\vdash_{ax} t\neq 0\qquad\[\vdash_{A3}t+0=t\justifies t\neq 0\vdash t+0=t\using{ind}\]\justifies t\neq 0\vdash t\neq 0\mbox{\&} t+0=t\using{\mbox{\&}_{right}}\]\justifies t\neq 0\vdash\exists v(v\neq 0\mbox{\&} v+0=t)\using{\exists_{right}}\]\justifies t\neq 0\vdash t<0\vee 0<t\using{\vee_{right}}
	\endprooftree$$}}
	\vspace{.5cm}
	Per il passo induttivo mostriamo prima
	\vspace{.2cm}
	\\(5E)
 \vspace{.2cm}
	{\scriptsize{$$\prooftree
	\vdash_{A4}v+s(r)=s(v+r)\qquad s(v+r)=t\vdash_{ax} s(v+r)=t\justifies s(v+r)=t\vdash v+s(r)=t\using{tran}
	\endprooftree$$}}
	\vspace{.2cm}
	\\(5F)
	\vspace{.2cm}
	{\scriptsize{$$\prooftree
	\vdash_{1.7}s(v)+r=s(v+r)\qquad\[s(v)+r=s(v+r),s(v)+r=t\vdash_{U1}s(v+r)=t\qquad s(v+r)=t\vdash_{5E} v+s(r)=t\justifies
	s(v)+r=s(v+r),s(v)+r=t\vdash v+s(r)=t\using{cut}\]\justifies t\neq s(r),s(v)+r=t\vdash v+s(r)=t\using{cut+ind}
	\endprooftree$$}}
	\vspace{.2cm}
	\\(5G)
	\vspace{.2cm}
	{\tiny{$$\prooftree
	\[\[v=0\vdash_{1.5}v+r=0+r\qquad\vdash_{S3}0+r=r\justifies v=0\vdash v+r=r\using{tran}\]v+r=r\vdash_{U2} s(v+r)=s(r)\justifies v=0\vdash s(v+r)=s(r)\using{cut}\]\quad s(v+r)=t,s(v+r)=s(r)\vdash_{U1} t=s(r)\justifies v=0,s(v+r)=t\vdash t=s(r)\using{cut}
	\endprooftree$$}}
	\vspace{.2cm}
  (5H)
	\vspace{.2cm}
	{\scriptsize{$$\prooftree
	\[\[\vdash_{1.7}s(v)+r=s(v+r)\qquad\[s(v)+r=s(v+r),s(v)+r=t\vdash_{U1}s(v+r)=t\qquad v=0,s(v+r)=t\vdash_{5G} t=s(r)\justifies v=0,s(v)+r=s(v+r),s(v)+r=t\vdash t=s(r)\using{cut}\]\justifies v=0,s(v)+r=t\vdash t=s(r)\using{cut}\]\justifies s(v)+r=t,v=0,t\neq s(r)\vdash\bot\using{\neg_{left}}\]\justifies t\neq s(r),s(v)+r=t\vdash v\neq 0\using{\neg_{right}}
	\endprooftree$$}}
        \\
	\vspace{3cm}
	\\(5I)
	\vspace{.2cm}
	{\tiny{$$\prooftree
	\[\[\[\[\vdash_{S4}s(w)\neq 0\justifies w\neq 0\mbox{\&} w+t=r\vdash s(w)\neq 0\using{ind}\]\qquad\[\vdash_{1.7} s(w)+t=s(w+t)\qquad\[v\neq 0,w+t=r\vdash_{U2}s(w+t)=s(r)\justifies w\neq 0\mbox{\&} w+t=r\vdash s(w+t)=s(r)\using{\mbox{\&}_{left}}\]\justifies w\neq 0\mbox{\&} w+t=r\vdash s(w)+t=s(r)\using{tran}\]\justifies w\neq 0\mbox{\&} w+t=r\vdash s(w)\neq 0\mbox{\&} s(w)+t=s(r)\using{\mbox{\&}_{right}}\]\justifies w\neq 0\mbox{\&} w+t=r\vdash\exists v(v\neq
 0\mbox{\&} v+t=s(r))\using{\exists_{right}}\]\justifies\exists w(w\neq 0\mbox{\&} w+t=r)\vdash\exists v(v\neq 0\mbox{\&} v+t=s(r))\using{\exists_{left}}\]\justifies t\neq r,t<r,t\neq s(r)\vdash t<s(r)\using{ind+def. < }
	\endprooftree$$}}
 (5L)
 \vspace{.2cm}
{\scriptsize{$$\prooftree
\[\[\[\[t\neq s(r),s(v)+r=t\vdash_{5F}v+s(r)=t\qquad t\neq s(r),s(v)+r=t\vdash_{5H}v\neq 0\justifies t\neq s(r),s(v)+r=t\vdash v\neq 0\mbox{\&} v+s(r)=t\using{\mbox{\&}_{right}}\]\justifies t\neq s(r),s(v)+r=t\vdash\exists w(w\neq 0\mbox{\&} w+s(r)=t)\using{\exists_{right}}\]\justifies t\neq s(r),\exists v(s(v)+r=t)\vdash\exists w(w\neq 0\mbox{\&} w+s(r)=t)\using{\exists_{left}}\]\justifies t\neq r,r<t,t\neq s(r)\vdash s(r)<t\using{ind}\]\justifies t\neq r,r<t,t\neq s(r)\vdash t<s(r)\vee s(r)<t\using{\vee_{right}}
\endprooftree$$}}
\vspace{.5cm}
{\scriptsize{$$\prooftree
\[t\neq r,t<r,t\neq s(r)\vdash_{5I} t<s(r)\justifies t\neq r,t<r,t\neq s(r)\vdash t<s(r)\vee s(r)<t\using{\vee_{right}}\]\quad t\neq r,r<t,t\neq s(r)\vdash_{5L} t<s(r) \vee s(r)<t \justifies t\neq r \to t<r\vee r<t,t\neq s(r)\vdash t<s(r)\vee s(r)<t\using{\vee_{left}}
	\endprooftree$$}}
\vspace{.5cm}
	\item[(5.16)] [ $\vdash t=r\ \vee\ t<r\ \vee\ r<t$ ]:
	\vspace{.2cm}
	\\Per induzione su $t$.
	\vspace{.2cm}
	\\Passo base
	\vspace{.2cm}
		$$
  \prooftree \[\vdash_{5.9} 0\leq r \justifies \vdash 0=r \vee 0<r \using{def.\leq}\] \justifies \vdash 0=r \vee 0<r \vee r<0   \using{\vee_{right}}
	\endprooftree $$
	\vspace{.5cm}
	\\Passo induttivo
	\vspace{.2cm}
	$${\scriptsize{
\prooftree \[r\leq t \vdash_{5.12} r<s(t) \justifies r \leq t \vdash s(t)=r \vee s(t)<r \vee r<s(t)\using{\vee_{right}}\] \quad \[t<r\vdash_{5.11} s(t)\leq r \justifies t<r \vdash s(t)=r \vee s(t)<r \vee r<s(t)\using{\vee_{right}}\] \justifies t=r \vee t<r \vee r<t \vdash s(t)=r \vee s(t)<r \vee r<s(t) \using{\vee_{left}}
\endprooftree}
}$$	
\vspace{.5cm}
	\item[(5.17)] [ $\vdash t\leq r\ \vee\ r\leq t$ ]:
	\vspace{.2cm}
		$$\prooftree\vdash_{5.16} t=r \vee t<r \vee t>r \justifies  \vdash t \leq r \vee r \leq t \using{def. \leq + \vee_{right}}
	\endprooftree$$		
	\vspace{.5cm}
	\item[(5.18)] [ $\vdash t+r\geq t$ ]:
	\vspace{.2cm}
	\\Applicare la definizione di $\leq$ e procedere per induzione su $r$.
	\vspace{.5cm}
	\item[(5.19)] [ $r\neq 0\vdash t+r>t$  ]:
	\vspace{.2cm}
	\\Basta applicare la definizione di $>$, scegliendo $w=r$ nell'$\exists_{right}$.
	\vspace{.5cm}
	\item[(5.20)] [ $r\neq 0\vdash t*r\geq t$ ]:
	\vspace{.2cm}
	\\Applicare la definizione e procedere per induzione su $t$.
	\vspace{.5cm}
	\item[(5.21)] [ $\vdash r\neq 0\ \leftrightarrow\ r>0$ ]:
	\vspace{.2cm}
	\\Basta semplicemente applicare la definizione di $>$ e mostrare le due direzioni \textsl{separatamente}.
	\vspace{.5cm}
	\item[(5.22)] [ $r>0,t>0\vdash r*t>0$ ]:
	\vspace{.2cm}
	\\Applicare la definizione di $>$ e scegliere $w=r*t$ nell'$\exists_{right}$.
	\vspace{.5cm}
	\item[(5.23)] [ $r\neq 0,t>\overline{1}\vdash t*r>r$ ]:
	\vspace{.2cm}
	\\Applicare la definizione di $>$ e scegliere $w=v*r$ nell'$\exists_{right}$, ove $v$ \`e la variabile ottenuta nell'$\exists_{left}$.
	\vspace{.5cm}
	\item[(5.24)] [ $r\neq 0\vdash t<p\ \leftrightarrow\ t*r<p*r$ ]:
	\vspace{.2cm}
	\\Applicare la definizione di $<$ e mostrare le due direzioni \textsl{separatamente}; in $(\rightarrow)$ scegliere $w=u*r$ nell'$\exists_{right}$, dove $u$ \`e la variabile ottenuta nell'$\exists_{left}$.
	\vspace{.5cm}
	\item[(5.25)] [ $r\neq 0\vdash t\leq p\ \leftrightarrow\ t*r\leq p*r$ ]:
	\vspace{.2cm}
	\\Applicare la definizione di $\leq$, ricordando in $(\rightarrow)$ punti (1.11), e (5.24$\rightarrow$), e in $(\leftarrow)$ i punti (3.9) e (5.24$\leftarrow$).
	\vspace{.5cm}
	\item[(5.26)] [ $\vdash t\not< 0$ ]:
	\vspace{.2cm}
	\\Basta applicare la definizione di $\not<$ e il punto (3.4).
	\vspace{.5cm}
	\item[(5.27)] [ $t\leq r\ \mbox{\&}\ r\leq t\vdash t=r$ ]:
	\vspace{.2cm}
	\\Applicare la definizione di $\leq$.
\end{enumerate}
\hspace{\stretch{1}}$\Box$\\

\vspace{.5cm}
\newpage
\begin{prop}
Per ogni numero naturale $k$, e ogni $\varphi$ formula si hanno:
\vspace{.2cm}
\begin{enumerate}
	\item[(6.1.1)] $\vdash x=0\ \vee\ ...\ \vee\ x=\overline{k}\ \leftrightarrow\ x\leq \overline{k}$
	\vspace{.2cm}
	\item[(6.1.2)] $\vdash \varphi(0)\ \mbox{\emph{\&}}\ ...\ \mbox{\emph{\&}}\ \varphi(\overline{k})\ \leftrightarrow\ \forall x (x\leq \overline{k}\ \to\ \varphi(x))$
	\vspace{.2cm}
	\item[(6.2.1)] $\vdash x=0\ \vee\ ...\ \vee\ x=\overline{k-1}\ \leftrightarrow\ x<\overline{k}$ dove $k>0$
	\vspace{.2cm}
	\item[(6.2.2)] $\vdash \varphi(0)\ \mbox{\emph{\&}}\ ...\ \mbox{\emph{\&}}\ \varphi(\overline{k-1})\ \leftrightarrow\ \forall x (x<\overline{k}\ \to\ \varphi(x))$, dove $k>0$
	\vspace{.2cm}
	\item[(6.3)] $\vdash (\forall x (x<y\ \to\ \varphi(x))\ \mbox{\emph{\&}}\ \forall x (x\geq y\ \to\ \psi(x)))\ \to\ \forall x (\varphi(x)\ \vee\ \psi(x))$
\end{enumerate}
\end{prop}
\vspace{.6cm}
\textsc{Dimostrazione}
\vspace{.2cm}
\begin{enumerate}
  \item[(6.1.1)] Vediamo prima ($\rightarrow$)\\
  \vspace{.2cm}
	$$ \scriptsize{\prooftree
	\[\[\[\[x=0 \vdash_{1.5} x+\overline{k}=0+\overline{k}\qquad \vdash_{S3} 0+\overline{k}=\overline{k}\justifies x=0 \vdash x+\overline{k}= \overline{k}\using{tran}\]\justifies x=0 	\vdash \exists w (w+x)=\overline{k}\using{\exists_{right}}\] \justifies x=0 \vdash x \leq \overline{k} \using{def. \leq}\]\quad \cdots \quad \[ \cdots \justifies x= \overline{k} \vdash x \leq \overline{k}\] \justifies x=0 \vee x= \overline{1} \vee \cdots \vee x=\overline{k} \vdash x\leq \overline{k} \using{\vee_{left}}\] \justifies \vdash x=0 \vee x=\overline{1} \vee \cdots \vee x= \overline{k} \rightarrow x \leq \overline{k}\using{\rightarrow_{right}}
	\endprooftree}$$
  \vspace{.2cm}\\
  ($\leftarrow$) Per induzione su k, ricordando la (5.12)\\
\\ Passo base:
\vspace{.2cm}
$$
{\prooftree
\[x=0 \vdash_{ax} x=0 \justifies x=0 \vdash x=0\]  \quad \[x<0 \vdash \bot \quad \bot \vdash x=0 \justifies x<0 \vdash x=0 \using{cut}\] \justifies x \leq 0 \vdash x=0
\endprooftree}
$$
\vspace{.2cm}
Passo induttivo:\\ Dimostriamo prima (6A)
$$
{\prooftree
\vdash_{5.12} x < s(\overline{k}) \rightarrow x \leq \overline{k} \quad \[x<s(\overline{k}) \vdash_{ax} x<s(\overline{k})\quad x \leq \overline{k} \vdash_{ax} x \leq \overline{k} \justifies x<s(\overline{k}), x<s(\overline{k})\rightarrow x\leq \overline{k} \vdash x \leq \overline{k} \using{\rightarrow_{left}}\] \justifies x<s(\overline{k}) \vdash x\leq \overline{k} \using{cut}
\endprooftree}
$$
E (6B):\\
\vspace{.2cm}
$$
{\prooftree
x \leq \overline{k} \vdash_{ax} x \leq \overline{k} \qquad x=0\ \vee\ ...\ \vee\ x=\overline{k} \vdash_{ax} x=0\ \vee\ ...\ \vee\ x=\overline{k}
\justifies x \leq \overline{k}, x\leq\overline{k}\rightarrow x=0\ \vee\ ...\ \vee\ x=\overline{k} \vdash  x=0\ \vee\ ...\ \vee\ x=\overline{k} \using{\rightarrow_{left}}
\endprooftree}
$$\\
\vspace{.2cm}
Allora si ottiene:
$$
\scriptsize{
\prooftree
\[\[x<s(\overline{k})\vdash_{6A} x\leq\overline{k}\quad x\leq\overline{k}, hp. indutt.\vdash_{6B} x=0\ \vee\ ...\ \vee\ x=\overline{k} \justifies x<s(\overline{k}), hp. ind.\vdash x=0\ \vee\ ...\ \vee x=s(\overline{k}) \using{cut + \vee_{right}}\]  \quad x=s(\overline{k}) \vdash x=0\ \vee\ ...\ \vee x=s(\overline{k}) \justifies x\leq s(\overline{k}), hp. ind. \vdash x=0\ \vee\ ...\ \vee x=s(\overline{k}) \using{\vee_{left}}\] \justifies hp. induttiva\vdash x\leq s(\overline{k}) \rightarrow x=0\ \vee\ ...\ \vee x=s(\overline{k}) \using{\rightarrow{right}}
\endprooftree}
$$
\vspace{.5cm}
  I tre punti seguenti 6.1.2, 6.2.1, 6.2.2 derivano dal precedente.
  \vspace{.5cm}
  \item[(6.3)]
  \vspace{.2cm}
 {\scriptsize $$\prooftree
  \[\[\vdash_{6C}z<y\vee y\leq z\qquad \[\[\[z<y\vdash_{ax} z<y\qquad\varphi(z)\vdash_{ax}\varphi(z)\justifies z<y,z<y\rightarrow\varphi(z)\vdash\varphi(z)\using{\rightarrow_{left}}\]
\justifies z<y,z<y\rightarrow\varphi(z)\vdash\varphi(z)\vee\psi(z)\using{\vee{right}}\]\qquad\[\[y\leq z\vdash_{ax} y\leq z\qquad\psi(z)\vdash_{ax}\psi(z)\justifies y\leq z,y\leq z\rightarrow\psi(z)\vdash\psi(z)\using{\rightarrow_{left}}\]\justifies y\leq z,y\leq z\rightarrow\psi(z)\vdash\psi(z)\vee\psi(z)\using{\vee_{right}}\]\justifies z<y\vee y\leq z,z<y\rightarrow\varphi(z),z\geq y\rightarrow\psi(z)\vdash\varphi(z)\vee\psi(z)\using{\vee_{left}+ind}\]\justifies z<y\rightarrow\varphi(z),z\geq y\rightarrow\psi(z)\vdash\varphi(z)\vee\psi(z)\using{cut}\]\justifies\forall x(x<y\rightarrow\varphi(x))\mbox{\&}\forall x(x\geq y\rightarrow\psi(x))\vdash\varphi(z)\vee\psi(z)\using{\forall_{left}+\mbox{\&}_{left}}\]\justifies\forall x(x<y\rightarrow\varphi(x))\mbox{\&}\forall x(x\geq y\rightarrow\psi(x))\vdash\forall x(\varphi(x)\vee\psi(x))\using{\forall_{right}}
  \endprooftree$$}
  \vspace{.2cm}
  dove (6C) deriva direttamente da (5.16) e dalla definizione di $\leq$.
\end{enumerate}
\hspace{\stretch{1}} $\Box$\\

\vspace{.5cm}
La definizione di una funzione in HA viene fatta indirettamente ogniqualvolta si prenda un termine $t(x)$ (ovvero in cui sia presente la variabile $x$ libera), e vi si sostituisca dentro un numerale $\overline{n}$, ottenendo cos\`i $t(\overline{n})$; tuttavia, nel formalismo base di HA sono stati definiti segni per rappresentare solo l'addizione e la moltiplicazione, ma manca il termine $t$ per cui $t(\overline{n})=\overline{f(x)}$, quindi non si pu\`o controllare che il formalismo rappresenti proprio la funzione $f$ da noi voluta; solo nel caso dell'addizione e del prodotto, infatti, si pu\`o dimostrare (come abbiamo fatto nella proposizione (3.4)) che il risultato di tali operazioni sui numerali \`e proprio quello corretto. Per indicare altre funzioni diverse da somma e prodotto, dobbiamo ricorrere alla loro caratterizzazione tramite formule, mentre per la loro rappresentabilit\`a, bisogna attendere il prossimo capitolo.

\begin{defi}
Diciamo che $t$ divide $p$ se esiste $z$ che, moltiplicato per $t$, d\`a $p$. Ossia\ $t|p\ :=\ \exists\ z(p=t*z)$.
\end{defi}

\begin{prop} Per ogni termine $p$, $r$, $t$ valgono:
\vspace{.2cm}
\begin{enumerate}
	\item[(7.1)] $\vdash t|t$
	\vspace{.2cm}
	\item[(7.2)] $\vdash\overline{1}|t$
	\vspace{.2cm}
	\item[(7.3)] $\vdash t|0$
	\vspace{.2cm}
	\item[(7.4)] $t|p\ \mbox{\emph{\&}}\ p|r\vdash t|r$
	\vspace{.2cm}
	\item[(7.5)] $p\neq 0\ \mbox{\emph{\&}}\ t|p\vdash t\leq p$
	\vspace{.2cm}
	\item[(7.6)] $t|p\ \mbox{\emph{\&}}\ p|t\vdash p=t$
	\vspace{.2cm}
	\item[(7.7)] $t|p\vdash t|(r*p)$
	\vspace{.2cm}
	\item[(7.8)] $t|p\ \mbox{\emph{\&}}\ t|r\vdash t|(p+r)$
\end{enumerate}
\end{prop}

\vspace{.5cm}
\textsc{Dimostrazione}
\begin{enumerate}
  \item[(7.1)] Applicare la definizione e scegliere $z=s(0)$ nell'$\exists_{right}$.
  \vspace{.5cm}
  \item[(7.2)] Applicare la definizione e scegliere $z=t$ nell'\ $\exists_{right}$.
  \vspace{.5cm}
  \item[(7.3)] Applicare la definizione e scegliere $z=0$ nell'\ $\exists_{right}$.
  \vspace{.5cm}
  \item[(7.4)] Applicare la definizione e scegliere $z=v*w$ nell'\ $\exists_{right}$, ove $v$ e $w$ sono le variabili ottenute nelle $\exists_{left}$.
  \vspace{.5cm}
  \item[(7.5)] Per induzione su $z$, ove $z$ \`e la variabile ottenuta nell'\ $\exists_{left}$.
  \vspace{.5cm}
  \item[(7.6)] Applicare la definizione.
  \vspace{.5cm}
  \item[(7.7)] Applicare la definizione e scegliere $z=v*r$ nell'\ $\exists_{right}$, ove $v$ \`e la variabile ottenuta nell'\ $\exists_{left}$.
  \vspace{.5cm}
  \item[(7.8)] Applicare la definizione e scegliere $z=v+w$ nell'\ $\exists_{right}$, ove $v$ e $w$ sono le variabili ottenute nelle $\exists_{left}$.
\end{enumerate}
\hspace{\stretch{1}} $\Box$\\

\vspace{.5cm}
Dimostriamo ora l' esistenza e l' unicit\`a del resto e del quoto nell'algoritmo di divisione.

\begin{prop}
{\tiny{$$y\neq 0\vdash \exists q\ \exists r((x=y*q+r\ \mbox{\emph{\&}} r<y)\ \mbox{\emph{\&}} \forall u\ \forall v\ ((x=y*u+v\ \mbox{\emph{\&}} v<y)\ \to\ q=u\ \mbox{\emph{\&}} r=v)))$$}}
\end{prop}
\vspace{.5cm}
\textsc{Dimostrazione}
\\La dimostrazione sarebbe molto lunga con le derivazioni, per questo motivo procederemo diversamente, indicando solo come proseguire.\footnote{Comunque \textsl{esiste} una derivazione}\newline
Proviamo l'esistenza: sia $\varphi(x):=\ y\neq 0\vdash \exists q\ \exists r(x=y*q+r\ \mbox{\&}\ r<y)$. $\varphi(x)$ si dimostra facilmente per induzione su $x$. Per $x=0$ si utilizzano la (3.4) e la (3.5) per dimostrare che $q=0$ e $r=0$ vanno bene.\\ Nel passo induttivo, bisogna ragionare cos\`i: siano $q$, $r$ tali che $x=y*q+r$; se $r+1$ resta minore di $y$, allora $q$ e $r+1$ sono il quoto e il resto cercati per $s(x)$, invece se $r+1=y$, allora $q+1$ e $0$ sono il quoto e il resto cercati.\newline
Per provare l'unicit\`a, si assuma $y\neq 0$, $x=y*q+r \mbox{\&}\ r<y$ e $x=y*u+v\mbox{\&}\ v<y$. Dalla (5.16), si ha che, nel confrontare $q$ e $u$, ci sono solo tre possibilit\`a: $q=u\vee q<u\vee q>u$.  $q<u$ non si pu\`o presentare, perch\`e equivale a $\exists\ w(w\neq 0\ \mbox{\&}\ w+q=u)$; questo implica $y*q+r= y*(w+q)+v$, e quindi $r=y*w+v$, da cui $r\geq y$, poich\`e $y*w\geq y$, essendo $w$ non nullo, per la (5.20). \\Pertanto, abbiamo dimostrato che $q<u \vdash \bot $, quindi $\neg (q<u)$. Analogamente, neanche $q>u$ si presenta, perch\`e equivale a $\exists\ w(w\neq 0\ \mbox{\&}\ w+u=q)$, che implica $y*u+v= y*(w+u)+r$, cio\`e $v=y*w+r$, che implica $v\geq y$, per cui anche qui si arriva a $\neg (q>u)$. \\Resta quindi valida solo la possibilit\`a $q=u$. Per l'unicit\`a del resto della divisione, si sfrutta l'unicit\`a del quoziente appena dimostrata; si ha che $x=y*q+r=y*q+v$, e per i resti si hanno le tre possibilit\`a $r=v\vee r<v\vee r>v$. Se $r>v$ vuol dire che $\exists w(w\neq 0\ \mbox{\&}\ w+v=r)$, da cui $y*q+(w+v)=y*q+v$, che implica $w=0$; quindi anche qui abbiamo dimostrato $r>v \vdash \bot$. Lo stesso si fa con $r<v$, e l'unica possibilit\`a che resta \`e $r=v$, come voluto.
\hspace{\stretch{1}} $\Box$\\

\newpage Concludiamo ora facendo vedere come sia possibile in HA definire formalmente alcuni concetti fondamentali in matematica.

\begin{defi}
Definiamo le seguenti:
\vspace{.1cm}
\begin{itemize}
\item sottrazione adeguata($\dot{-}$)\footnote{$z$ \`e legato a $x$ e $y$ tramite $\dot{-}$ se soddisfa alla propriet\`a nella definizione. Questo \`e diverso dal dire che $z$ \`e il risultato della funzione $\dot{-}$ applicata a $x$ e $y$, nonostante la notazione matematica sia la stessa; \`e il problema a cui si \`e accennato prima di introdurre la divisione}: {\footnotesize{$$z=y\dot{-} x:=\ (x<y\ \mbox{\emph{\&}} x+z=y)\ \vee\ (y\leq x\ \mbox{\emph{\&}} z=0)$$}}
\vspace{.1cm}
\item $z$ \`e il resto della divisione di $y$ per $x$:{\footnotesize{ $$z=rm(x,y)\ :=\ (x\neq 0\ \ \exists u(y=u*x+z)\ \mbox{\emph{\&}} z<x)\ \vee\ (x=0\ \mbox{\emph{\&}} z=y)$$}}
\vspace{.1cm}
\item $x$ \`e un numero primo: {\footnotesize{$$Prime(x):=\ x>1\ \mbox{\emph{\&}} \forall yz\  (x=y*z\ \to\ y=x\ \vee\ y=1)$$}}
\end{itemize}
\end{defi}

\vspace{.2cm}
\begin{prop} Valgono:
\vspace{.2cm}
\begin{enumerate}
	\item[(11.1)] $\vdash Prime(x)\ \leftrightarrow\ x>1\ \mbox{\emph{\&}} \forall y\ (y|x\ \to\ y=1\ \vee\ y=x)$
	\vspace{.2cm}
	\item[(11.2)] $\vdash Prime(x)\ \leftrightarrow\ x>1\ \mbox{\emph{\&}}\ \forall yz\ (x|y*z\ \to\ x|y\ \vee\ x|z)$
\end{enumerate}
\end{prop}
\vspace{.5cm}
\textsc{Dimostrazione}
Basta applicare le definizioni.
\hspace{\stretch{1}} $\Box$\\



%\include{05_rappresentabilita}
%\include{06_aritmetizzazione}
%\include{07_logica_dimostrabilita}
%\include{08_incompletezza_primo}
%\include{09_incompletezza_secondo}
%\include{10_conseguenze}

\begin{thebibliography} {}
\bibitem{Tur} \textsc{Turing}, \textsl{On computable numbers, with an
application to the entscheidungproblem}, 1936: abelard.org.

\bibitem{Stnfrd} \textsc{Online Stanford Enciclopedia}: plato.stanford.edu.

\bibitem{CompNr} \textsl{Computable Numbers and the Turing Machine}, 1936:
turing.org.uk.

\bibitem{WTM} \textsc{Jack Copeland}, \textsl{What is a Turing Machine?}, 2000:
alanturing.net.

\bibitem{Cut} \textsc{N. Cutland}, \textsl{Computability. An
  Introduction to Recursive Function Theory}, 1980.

\bibitem{Dis} \textsc{Unipd}, \textsl{Dispense del corso di Logica 2},
  2009.
\bibitem{Wik} \textsc{Wikipedia}, \url{http://en.wikipedia.org}.
\bibitem{TK} \textsc{G. Takeuti}, \textsl{Proof Theory}, 1987.

\bibitem{key-1}{[}Hilbert, 18.. D.HILBERT. Ricerche sui fondamenti
della matematica, a cura di V.Michele Abrusci, Bibliopolis, 1984.

\bibitem{key-4}{[}Sambin, 1987 G. SAMBIN. Alla ricerca della certezza
perduta. 

\bibitem{key-5}{[}Cantini, 1979 A. CANTINI. I fondamenti della matematica,
Loescher, 1979.

\bibitem{key-8}{[}Hintikka, 19.. HINTIKKA. On Godel.

\bibitem{key-1}{[}Avigad, 2001 J.AVIGAD \& E.H.RECH. {}``Clarifyng
the nature of infinite'': the development of metamathematics and
proof theory.

\bibitem{key-1}{[}Zach, .... R.ZACH. Hilbert's program then and now.

\bibitem{Boolos} \textsc{G. Boolos}, \textsl{The Logic of Provability}, Cambridge University Press, 1993.

\bibitem{Visser} \textsc{A. Visser}, \textsl{Aspects of Diagonalization and Provability}, PhD thesis, Utrecht, 1981.

\end{thebibliography}

\end{document}
