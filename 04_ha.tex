\chapter{Sistema assiomatico HA}
\section{Introduzione}

Quotidianamente l'uomo al fine di argomentare, strutturare discorsi e giungere a conclusioni fa uso della logica e di un linguaggio formale ben preciso che gli permette di evitare ambiguit\`a e ridondanze durante un colloquio.\\
Di fondamentale importanza sono le \emph{deduzioni formali} in quanto forzano la verit\`a delle nostre argomentazioni perch\`e ci permettono di definire in modo chiaro le premesse da cui partiamo e le varie regole di inferenza che utilizziamo per arrivare alla conclusione.\\
Quello che a noi interessa quindi sono le \emph{prove} che portiamo per mostrare se determinate proposizioni sono vere e perch\`e lo sono.\\
L'esempio pi\`u antico di teoria assiomatica in questo senso \`e dato dagli "Elementi" di Euclide e si pu\`o dire che, ad oggi, quasi tutta la matematica \`e sviluppata in modo assiomatico, anche se spesso non vengono specificati a quale insieme di assiomi si faccia riferimento.

Quello di cui noi abbiamo bisogno \`e una \emph{teoria assiomatica formale}, cio\`e  una teoria formale in cui \`e possibile esprimere enunciati e dedurre le loro conseguenze logiche in modo del tutto formale e meccanico.\\

Una \emph{teoria assiomatica} $T$ nel linguaggio $\mathcal{L}$ \`e determinata da un calcolo logico $L$ e da un insieme $\Sigma$ (anche infinito) di formule di $\mathcal{L}$ chiamate assiomi di $T$. Si dice che $\varphi$ \`e conseguenza logica di $\Gamma$\footnote{Con $\Gamma$ indichiamo l'insieme di tutte le assunzioni (premesse) di $T$} in $T$ se esiste una deduzione in $L$ con conclusione $\varphi$ e assunzioni contenute in $\Gamma \cup \Sigma$. In particolare, si dice che $\varphi$ \`e un teorema di $T$ se esiste una deduzione di $\varphi$ con assunzioni tutte contenute in $\Sigma$


Noi vogliamo inoltre che una teoria assiomatica soddisfi le seguenti propriet\`a:
\begin{description}
\item[Coerenza]
La coerenza, o non contraddittoriet\`a \`e la caratteristica di una teoria assiomatica di non permettere di dedurre una qualche proposizione e, contemporaneamente, la sua negazione. 
Ovviamente se la teoria \`e in grado di dimostrare una proposizione e la sua negazione la teoria non ha alcun senso, perché da quella possiamo (\textit{Ex falso quodlibet}) dedurre qualsiasi proposizione, quindi la teoria perde di significato.
\item[Decidibilit\`a degli assiomi] Significa che deve esistere una procedura effettiva per poter stabilire se una qualche proposizione \`e o meno un assioma della teoria.
\item[Indipendenza] Questa propriet\`a non \`e in realta, indispensabile, quanto le altre, per\`o torna utile.
In sostanza una teoria si dice indipendente se ogni suo assioma \`e indipendente dai restanti assiomi. 
\end{description}

\vspace{0.4 cm}
Vediamo allora gli elementi fondamentali di una teoria assiomatica:
\begin{enumerate}
	\item un \textsl{\textbf{linguaggio formalizzato}} $\mathcal{L}$
\item un insieme di enunciati nel linguaggio che trattiamo come \textsl{\textbf{assiomi}} della teoria
\item un \textsl{\textbf{sistema deduttivo}} per costruire prove, ossia per derivare teoremi da assiomi

\end{enumerate}
Un linguaggio formalizzato $\mathcal{L}$ consiste di:
\begin{itemize}
\item un \underline{alfabeto} $S$ di simboli che possono essere le variabili, i connettivi e i quantificatori e i simboli relazionali
\item \underline{regole di costruzione sintattica} che permettono di costruire espressioni accettate dal sistema.
\end{itemize}
\vspace{0.5 cm}

Dal momento che le prove sono composte da un numero finito di passaggi logici, allora per ogni prova esister\`a solo un numero finito di assiomi coinvolti. \\
Specifichiamo infine che fornire una prova non significa decidere in anticipo se questa prova esiste o meno. Noi siamo interessati non ad argomentazioni semantiche ma a derivazioni nel sistema formale HA, che \`e l'equivalente \textsl{costruttivo} della teoria assiomatica di Peano, di cui parleremo tra breve.

\vspace{0.5 cm}

\underline{Perch\'e ci serve tutto questo?}\\
\vspace{0.2 cm}

Il primo teorema di incompletezza di G\"odel pone dei limiti alle teorie assiomatiche dell'aritmetica, o meglio alle teorie formali assiomatiche dell'aritmetica, il cui linguaggio di base consiste nella definizione della struttura dei numeri naturali, della somma e della moltiplicazione in $\mathbb{N}$ e in un apparato logico del primo ordine.\newline
Quindi, dati gli assiomi che regolano la struttura di una sequenza di naturali e che caratterizzano somma e prodotto esiste sicuramente almeno una proposizione $\phi$, formulata nel dato linguaggio dell'aritmetica di base, della quale non ᅵ possibile dimostrare, a partire dagli assiomi, nᅵ la sua affermazione, nᅵ la sua negazione.\footnote{Notiamo che comunque l'incompletabilit\`a non compromette l'aritmetica che c'\`e alla base.}\\
Questo fatto \`e proprio ci\`o che viene affermato nel primo teorema di G\"odel: ogni teoria aritmetica di base assiomatizzata rimane incompleta, qualsiasi sforzo si faccia per completarla con altri assiomi.\\
Quindi qualsiasi teoria matematica assiomatizzata con un'aritmetica di base costituita dalle operazioni di somma e moltiplicazione in $\mathbb{N}$ deve essere incompleta e incompletabile (consistente e con alcuni enunciati non dimostrabili).

\section{Costruzione di HA}
\subsection{Linguaggio LA}
Quello che ci poniamo di fare ora è definire il sistema assiomatico PA e di conseguenza il sistema HA. Infatti l'aritmetica di Heyting HA formalizza la teoria intuizionista dei numeri.\\ Essa ha gli stessi assiomi propri dell'aritmetica classica PA, a cui si aggiunge l'apparato deduttivo della logica intuizionista. \\I teoremi di HA sono quindi necessariamente un sottoinsieme proprio dei teoremi di PA. L'assioma di induzione appartiene sia a PA che ad HA e questo riflette le posizioni della matematica intuizionista sul carattere costruttivo della successione dei numeri naturali. Come abbiamo visto prima, per far ci\`o, dobbiamo specificare il \textsl{linguaggio}, gli \textsl{assiomi} e il \textsl{sistema di prova deduttivo}. Del primo parleremo ora.
Definiamo dunque:
\begin{enumerate}
 \item \underline{\texttt{il linguaggio al primo ordine LA}}\footnote{LA sta per \textsl{linguaggio dell'aritmetica}}, il cui vocabolario comprende:
\begin{itemize}
	\item [-] la \textsl{costante} individuale $0$
	\item[-] il predicato di \textsl{uguaglianza} "`$=$"' 
	\item[-] la funzione a un solo argomento \textsl{successore} $s(x)$
	\item[-] le funzioni a due argomenti $x+y$ (\textsl{somma}) e $x*y$ (\textsl{moltiplicazione})	
	\item[-] quantit\`a numerabile di variabili $x, y, z, w\ldots$
	\item[-] connettivi logici e quantificatori
	\item[-] i segni $,$, $($, $)$
\end{itemize}
\vspace{.5cm}
Seguendo quanto detto prima, dobbiamo ora dare le regole per la costruzione di espressioni accettate. Lo faremo
definendo in modo induttivo i termini e le formule. Intuitivamente i termini sono i mattoni con cui si costruiranno le formule, mentre queste ultime saranno le espressioni sulle quali lavoreremo.

\item L'\underline{\texttt{insieme dei termini $Trm$}} \`e definito in modo induttivo come segue:
\begin{itemize}
\item[-] La costante $0$ \`e un termine;
\item[-] Ogni variabile \`e un termine;
\vspace{0.3 cm}
\item[-] 
	$\prooftree
  t_1\in Trm ,t_2\in Trm,\dots,t_n\in Trm
   \justifies
f_n(t_1,t_2,\dots,t_n) \in Trm
\endprooftree$ 
\end{itemize}
\vspace{.5cm}
\item  L'\underline{\texttt{insieme delle formule $Frm$}} \`e definito in modo induttivo come segue:
\begin{itemize}
\item[-] Ogni \textit{formula atomica} \`e una formula\footnote{una formula atomica \`e del tipo $R_n(t_1,t_2,\dots,t_n)$ in cui $t_1,t_2,\dots, t_n$ sono termini e $R_n$ \`e una relazione in LA ad $n$ argomenti}
\item[-] Date $\alpha,\beta \in Frm$, $\alpha \circ \beta \in Frm$, con $\circ$ un qualsiasi connettivo di LA, cio\`e $Frm$  chiuso per i connettivi sopra elencati.
\item[-] 
	$\prooftree
  \alpha \in Frm \qquad x \in Variabili
   \justifies
\forall x.\alpha \in Frm
\endprooftree$ 
\vspace{0.3 cm}
\item[-] 
$\prooftree
  \alpha \in Frm \qquad x \in Variabili
   \justifies
\exists x.\alpha \in Frm
\endprooftree$ 

\end{itemize}

\end{enumerate}\par\ \par\noindent
\vspace{.5cm}

E con questo abbiamo deinito il linguaggio che useremo.
\subsection{Assiomi}
Questa teoria assiomatica mira alla costruzione dei numeri naturali e all'aritmetica tutta, quindi i suoi 
assiomi si baseranno sui postulati di Peano, che sono i seguenti:
\begin{enumerate}
	\item[(P1)] $0$ \`e un numero naturale
	\item[(P2)] Se $x$ \`e un numero naturale, allora esiste un altro numero na-turale, detto \textsl{successore} di $x$, denotato con $s(x)$
	\item[(P3)] $0\neq s(x)$ per qualsiasi numero naturale $x$
	\item[(P4)] Se $s(x)=s(y)$ allora $x=y$
	\item[(P5)] Sia $P$ una certa propriet\`a sui numeri naturali. Se $P$ vale per $0$ e se per ogni numero naturale $x$ che gode della propriet\`a si ha che $s(x)$ gode di $P$, allora $P$ vale per ogni naturale (non \`e altro che il \textsl{principio di induzione})
\end{enumerate}\par\ \par\noindent

Prima di mostrare come saranno fatti gli assiomi della nostra teoria vediamo un paio di osservazioni.

La prima \`e la seguente:
per noi che abbiamo in mente gi\`a la struttura dei Naturali, tutti questi assiomi, sembrano sovrabbondanti, e saremmo portati a dire che bastano solamente gli assiomi \textbf{(P1)} e \textbf{(P2)}.
Notiamo invece che, una struttura ad anello, che parte dallo 0 e dopo un certo numero di passi (iterazioni della funzione successore) torna sullo 0, soddisfa ai due assiomi, ma \`e ben lontana dal descrivere i Naturali.

Possiamo allora introdurre l'assioma \textbf{(P3)}, per dire che 0 non pu\`o essere il successore di alcun numero.
Ancora una volta per\`o possiamo facilmente trovare una struttura che renda veri gli assiomi ma che non rappresenti i Naturali. Tale struttura per esempio \`e data da una ``corona del rosario'', ovvero una struttura come la precedente con l'aggiunta dello 0 in testa.

Ancora allora possiamo aggiungere il postulato \textbf{(P4)}, e sperare che ci basti.
Ovviamente non \`e cos\`i come mostra la seguente struttura:
consideriamo come dominio di interpretazione l'insieme dato da tutte le sequenza del tipo \textit{0,00,000,...}
e le sequenze analoghe a partire da un qualche altro elemento \textit{a,aa,aaa,...}. Come funzione successore possiamo prendere quella che data una sequenza \textit{s} dell'insieme, restituisca la pi\`u piccola di quelle sequenze che contenga propriamente \textit{s}. Si dimostra facilmente che questa struttura soddisfa a tutti i primi quattro postulati, ma non all'induzione e, naturalmente, non rappresenta i Naturali (in realt\`a rappresenta la giustapposizione di due copie di $\mathbb{N}$).

Questo per dire che dobbiamo sempre tenere a mente che quello che stiamo cercando di fare \`e di trasmettere la nostra idea di $\mathbb{N}$ ad un robot, attraverso gli assiomi, ed in conclusione possiamo dire che ci servono tutti.

\vspace{0.3cm}
La seconda osservazione richiede un paio di definizioni e di spostarci da un sistema al I ordine, ad uno del II.
\begin{defi}
Dati un insieme $A$, un elemento $a_0$ ed una funzione $s:A\rightarrow A$, la terna $(A,a_0,s)$ si dice \textit{Sistema di Peano} se soddisfa a:
\begin{itemize}
\item $ a_0 \in A $
\item $ a\in A \rightarrow s(a) \in A $
\item $ s(x)=s(y)\rightarrow x=y $
\item $ \forall x (a_0\neq s(x)) $
\item $ \forall P $ propriet\`a $ (P(a_0)\&(P(a)\rightarrow P(s(a)))\rightarrow \forall a(P(a))) $
\end{itemize}
\end{defi}

Si nota subito che questa \`e una generalizzazione degli assiomi di Peano che abbiamo visto sopra e che la terna $(\mathbb{N},0,s)$ \`e un sistema di Peano.
La domanda sorge ora spontanea: visto che abbiamo generalizzato, questa terna \`e l'unico sistema di Peano possibile?
Si vede facilmente che cos\`i non \`e, ovvero possiamo esibire un'altra terna che non sia $(\mathbb{N},0,s)$ e che soddisfi a quei cinque assiomi, per esempio $(\mathbb{P},2,n\mapsto n+2)$, dove con $ \mathbb{P} $ indichiamo l'insieme dei numeri pari.
In realt\`a esiste una corrispondenza biunivoca, che ``mantiene il successore'' tra i due sistemi che abbiamo visto.
Pi\`u precisamente diamo la seguente 
\begin{defi}
Dati due sistemi di Peano $ (A,a_0,s) $ e $ (B,b_0,t) $, si dice Isomorfismo tra i due, il dato di due funzioni:
$ f:A\rightarrow B $, $ g:B\rightarrow A $, tali che:
\begin{enumerate}
\item $ f(a_0)=b_0 $
\item $ f(s(a))=t(f(a)) $ per ogni $ a\in A $
\item $ g\circ f=1_A $ e $ f\circ g=1_B $
\end{enumerate}
e se tali funzioni esistono allora i due sistemi si dicono isomorfi.
\end{defi}

Con questa definizione ginugiamo infine al seguente 
\begin{thm}
Ogni sistema di Peano $ (A,a_0,t) $ \`e isomorfo a $ (\mathbb{N},0,s) $ (e quindi sono tutti isomorfi tra loro).
\end{thm}

La dimostrazione \`e semplice e lasciata per esercizio, considerando le due funzioni:
$ f:\mathbb{N}\rightarrow A $ che manda lo 0 in $ a_0 $ ed il numero $ k $ nell'elemento $ tt...t(a_0) $ k-volte e
$ g:A\rightarrow \mathbb{N} $ costruita in maniera totalmente speculare. (L'unica nota a cui stare attenti \`e che sembra non ovvio che si possa scrivere ogni elemento di $ A $ come risultato dell'applicazione della funzione successore un numero finito di volte a partire da $ a_0 $; ma questo in realt\`a si pu\`o dimostrare per induzione prendendo proprio come propriet\`a questa appena descritta.)

\vspace{0.3cm}
Questa bella propriet\`a di $ \mathbb{N} $, ovvero di essere l'unico modello a meno di isomorfismi, non \`e pi\`u valido in una teoria al prim'ordine, come quella che stiamo costruendo noi. La dimostrazione non la daremo qui, ma \`e facilmente reperibile [\textit{cfr Zanardo, Metodi assiomatici e teoria degli insiemi, dispense}]

\vspace{0.5cm}
Siamo giunti finalmente a mostrare quali saranno i nostri assiomi.
Gli \underline{\textbf{assiomi}} di HA dunque sono:
\begin{enumerate}
	\item[(A1)] $\vdash 0\neq s(x)$ (corrisponde a (P3))
	\item[(A2)] $s(x)=s(y)\vdash x=y$ (corrisponde a (P4))
	\item[(A3)] $\vdash x+0=x$
	\item[(A4)] $\vdash x+s(y)=s(x+y)$
	\item[(A5)] $\vdash x*0=0$
	\item[(A6)] $\vdash x*s(y)=x*y+x$
	\item[(A7)] Regola di induzione (corrisponde a (P5)):\newline\newline
	$\prooftree
  \Gamma, \varphi(z) \vdash \varphi(s(z))
   \justifies
 \Gamma, \varphi(0) \vdash \varphi(t)
 \using
 {IR}
\endprooftree$.
\end{enumerate}\par\ \par\
Si noti che:
\begin{itemize}
\item (A1) afferma che $0$ non \`e l'immagine di $s$ e questo permette di escludere modelli in cui, iterando la funzione successore, si possa compiere un loop che ritorni al punto di partenza. Nel modello di Peano questo sta a indicare, ad esempio, che $0$ non \`e il "`successivo"' di nessun numero naturale
\item (A2) afferma che $s$ \`e una funzione iniettiva che permette di escludere i modelli in cui, partendo da $0$ e andando avanti ripetutamente da un elemento al successore, si possa tornare su un elemento gi\`a visitato e rimanere confinati in un ciclo. In altre parole si sta dicendo ad esempio che se $s$ sta ad indicare il "`successivo"' di un numero naturale, allora se $s(x)$ e $s(y)$ coincidono, anche i "`numeri"' di partenza necessariamente coincidono
\item (A3) e (A4) sono gli assiomi che definiscono la somma tra due numeri naturali
\item (A5) e (A6) definscono il prodotto tra due numeri naturali
\item (A7) permette di affermare che l'insieme dei naturali $\mathbb {N}$ \`e il pi\`u piccolo insieme che contenga lo $0$ e che contenga il successore di ogni suo elemento (cio\`e che sia chiuso per la funzione s) e quindi permette di escludere modelli in cui siano presenti degli elementi "`intrusi"' al di fuori della sequenza infinita dei successori dello zero
\end{itemize}

\par\ \par\noindent
\underline{\texttt{Osservazione}} Non sono stati dati, almeno esplicitamente, assiomi corrispondenti ai postulati (P1) e (P2). In realt\`a, essi seguono dal fatto che consideriamo $\mathbb{N}$ come dominio di interpretazione e dall'aver definito nel nostro linguaggio la costante $0$ ($0$ \`e automaticamente elemento di $\mathbb{N}$) e la funzione a un solo argomento successore (che vogliamo sia totale, dunque se $x$ \`e un elemento, anche $s(x)$ lo \`e).
\par\ \par\noindent

\vspace{0.5 cm}
Inoltre essendo HA una teoria con uguaglianza, si hanno anche gli assiomi:
\begin{enumerate}
	\item[(U1)] $x=y,x=z\vdash y=z$
	\item[(U2)] $x=y\vdash s(x)=s(y). $
\end{enumerate}
\par\ \par

Potrebbe sembrare che il linguaggio di HA sia povero di espressivit\`a e non ci permetta di parlare di una stringa finita di numeri ma, una volta che abbiamo l'esponenziazione, possiamo semplicemente \textsl{codificare} una sequenza finita di numeri.\\ G\"odel ha mostrato che l'esponenziazione \`e definibile tramite l'introduzione di qualche altra funzione nella nostra aritmetica (la $\beta$-function di cui si parler\`a in seguito). In realt\`a \`e stato provato che essa pu\`o essere introdotta in HA e questo ci permette dunque di codificare sequenze.\\ Quindi potrebbe sembrare che HA dotato di esponenziazione sia pi\`u forte di HA, ma a dire il vero, per quanto appena accennato, sono equivalenti e dunque possiamo eliminare l'esponenziazione dalle nostre assunzioni.\\ Oltre alla potenza di questo linguaggio, verr\`a mostrata pi\`u avanti la consistenza di HA (secondo teorema di G\"odel).  Noi dunque considereremo il sistema assiomatico HA senza estenderlo. \newline
Vedremo come sar\`a possibile definire i concetti di \textsl{numero primo} (che insieme all'esponenziazione \`e comodissimo per codificare, in particolare, sequenze finite di numeri naturali), \textsl{resto della divisione}, \textsl{sottrazione adeguata}, \textsl{divisibilit\`a}, \textsl{relazione d'ordine} e cos\`i via, giusto rimanendo nell'ambito di questa teoria formale. Quindi vedremo come in questo sistema assiomatico sia possibile \textsl{derivare} tutto ci\`o che ci serve per fare matematica. \newline
Si utilizzer\`a il sistema di regole che si basa sul calcolo dei sequenti.

\vspace{0.3 cm}
\subsection{Regole del calcolo dei sequenti}
Il nome "`sequente"' \`e la traduzione dal tedesco della parola {\itshape Sequenz}, introdotta da Gerhard Gentzen nel 1934, ed indica un formalismo che si presenta nella forma:
$$\Gamma_1,\Gamma_2,\dots,\Gamma_m\vdash\Delta_1,\Delta_2,\dots,\Delta_n$$
Il simbolo $\vdash$\ indica che le formule a sinistra (gli {\itshape antecedenti}) sono le ipotesi dalla cui asserzione segue la dichiarazione delle formule a destra (i {\itshape conseguenti}).\\
Occorrono anche altre abbreviazioni come ad esempio:
$$\frac{\Gamma\vdash\Delta}{\Gamma'\vdash\Delta'}$$
dove la barra orizzontale \`e dunque un modo veloce per scrivere `"'ci\`o che sta sopra comporta ci\`o che sta sotto"'.\\
Ovvero quello che sta sopra la barra orizzontale deve essere assunto come premessa, mentre quello che sta sotto corrisponde alle deduzioni fatte a partire da tali premesse.\\
Quindi elenchiamo le regole del calcolo dei sequenti (nella versione intuizionista) che utilizzeremo per costruire le prove:
\begin{enumerate}
\vspace{.3cm}
\item[] \textit{Assioma}:
$$\Gamma,A\vdash A \ ax$$

\item[] \textit{Ordine o Scambio}:
$$\prooftree
  A,B \vdash \Delta
   \justifies
 B,A\vdash \Delta
 \using
 {ord}
\endprooftree$$\\
\item[] \textit{Indebolimento}:
$$\prooftree
  \Gamma \vdash \Delta
   \justifies
 \Gamma, \Sigma \vdash \Delta
 \using
 {ind}
\endprooftree$$\\

\item[] \textit{Contrazione}:
$$\prooftree
  \Gamma, A, A \vdash \Delta
   \justifies
 \Gamma, A \vdash \Delta
 \using
 {cont}
\endprooftree$$\\

\item[] \textit{Taglio}:
$$\prooftree
  \Gamma \vdash A\qquad\Gamma', A\vdash \Delta
   \justifies
 \Gamma, \Gamma' \vdash \Delta
 \using
 {cut}
\endprooftree$$\\

	\item[] \textit{Regole per}\ \&:\par\ \par\noindent
$$\prooftree
  \Gamma, A,B \vdash \Delta
   \justifies
 \Gamma, A\ \mbox{\&}\ B \vdash \Delta
 \using
 {\mbox{\&}_{left}}
\endprooftree$$
\hspace{\stretch{1}}

$$\prooftree
  \Gamma \vdash A\qquad\Gamma \vdash B
   \justifies
 \Gamma \vdash A\mbox{\&}B
 \using
 {\mbox{\&}_{right}}
\endprooftree$$\\

\item[] \textit{Regole per $\vee$}:\\
$$\prooftree
  \Gamma, A \vdash \Delta\qquad\Gamma, B \vdash \Delta
   \justifies
 \Gamma, A\vee B \vdash \Delta
 \using
 {\vee_{left}}
\endprooftree$$\\
	$\prooftree
  \Gamma \vdash A
   \justifies
 \Gamma \vdash A\vee B
 \using
 {\vee_{right}}
\endprooftree$
\hspace{\stretch{1}}
$\prooftree
  \Gamma \vdash B
   \justifies
 \Gamma \vdash A\vee B
 \using
 {\vee_{right}}
\endprooftree$
\vspace{.8cm}

\item[] \textit{Regole per $\to$}:\\
$$\prooftree
  \Gamma_1 \vdash A\qquad\Gamma_2, B \vdash \Delta
   \justifies
 \Gamma_1, \Gamma_2, A\to B \vdash \Delta
 \using
 {\to_{left}}
\endprooftree$$\\
$$\prooftree
  \Gamma, A \vdash B
   \justifies
 \Gamma \vdash A\to B
 \using
 {\to_{right}}
\endprooftree$$\\

\item[] \textit{Regole per $\neg$}:\\
$$\prooftree
  \Gamma \vdash A
   \justifies
 \Gamma, \neg A \vdash \Delta
 \using
 {\neg_{left}}
\endprooftree$$\\
$$\prooftree
  \Gamma, A \vdash \bot
   \justifies
 \Gamma \vdash \neg A
 \using
 {\neg_{right}}
\endprooftree$$\\

\item[] \textit{Regole per $\forall$}:\\
$$\prooftree
  \Gamma, A(c) \vdash \Delta
   \justifies
 \Gamma, \forall\ x\ A(x) \vdash \Delta
 \using
 {\forall_{left}}
\endprooftree$$\\
$$\prooftree
  \Gamma \vdash A(z)
   \justifies
 \Gamma \vdash \forall\ x\ A(x)
 \using
 {\forall_{right}\ \mbox{*}}
\endprooftree$$\\

\item[] \textit{Regole per $\exists$}:\\
$$\prooftree
  \Gamma, A(z) \vdash \Delta
   \justifies
 \Gamma, \exists\ x\ A(x) \vdash \Delta
 \using
 {\exists_{left}\ \mbox{*}}
\endprooftree$$\\
$$\prooftree
  \Gamma \vdash A(c)
   \justifies
 \Gamma \vdash \exists\ x\ A(x)
 \using
 {\exists_{right}}
\endprooftree$$

\end{enumerate}
\par\ \par\noindent
\vspace{.3cm}

Contrassegnamo con * le regole in cui $z$ non \`e una variabile libera nel contesto $\Gamma, \Delta$.\\

\section{Espressivit\`a di HA}
Vediamo ora cosa \`e possibile derivare nel nostro sistema assiomatico.\\
Si cercher\`a di dimostrare ogni proposizione tramite una prova, ma si ricorda che le derivazioni non sono uniche e quindi potrebbero esistere altri modi per provare un determinato enunciato.\\
Inoltre prima di cominciare si noti che, per dire che qualcosa \`e derivabile nella nostra teoria, si user\`a il simbolo $\vdash$ anzich\`e $\vdash_{HA}$, ossia si lascia sottinteso il sistema in cui si lavora, salvo specifiche particolari.\\
Valendo: $$\prooftree
\vdash\Gamma\quad\[\vdash\varphi(0)\quad\[\varphi(z),\Gamma\vdash\varphi(s(z))\justifies \varphi(0),\Gamma\vdash \varphi(t)\using{IR}\]\justifies \Gamma\vdash\varphi(t)\using{cut}\]\justifies \vdash\varphi(t)\using{cut}
\endprooftree$$
dimostreremo per tutte le propriet\`a ricorsive solamente $\varphi(0)$ e $\varphi(z)\vdash\varphi(s(z))$. \newline
Infatti, per la validit\`a di:
$$\prooftree
   \vdash \varphi(t)
   \justifies
  \vdash \forall\ x\ \varphi(x)
 \using
 {\forall_{right}}
\endprooftree$$\\
una volta derivati il passo base e il passo induttivo, abbiamo tutti gli strumenti per provare che la propriet\`a $\varphi$ vale per qualunque numero naturale.
\newline
\begin{prop}
Dati $p$, $r$, $t$ termini, in HA valgono le seguenti propriet\`a:
\begin{enumerate}
	\item[(R1)] Simmetria destra:\\ $$\prooftree
  \Gamma \vdash t=r
   \justifies
 \Gamma \vdash r=t
 \using
 {sim_{dx}}
\endprooftree$$
  \item[(R2)] Simmetria sinistra:\\ $$\prooftree
  \Gamma,t=r \vdash\Gamma'
   \justifies
  \Gamma,r=t\vdash\Gamma'
 \using
 {sim_{sx}}
\endprooftree$$
	\item[(R3)] Transitivit\`a:\\ $$\prooftree
  \Gamma \vdash p=t \qquad \Gamma'\vdash t=r
   \justifies
 \Gamma, \Gamma' \vdash p=r
 \using
 {tran}
\endprooftree$$
\end{enumerate}
\end{prop}
\textsc{Dimostrazione} Assumiamo di aver dimostrato (1.2) e (1.3)
\begin{enumerate}
\item[(R1)]
{\scriptsize{
	$$\prooftree
  \Gamma \vdash t=r \qquad t=r\vdash_{1.2} r=t
   \justifies
 \Gamma \vdash r=t
 \using
 {cut}
\endprooftree$$}}
  \item[(R2)]
{\scriptsize{$$\prooftree
	t=r\vdash_{1.2} r=t\qquad \Gamma,r=t\vdash\Gamma'
  \justifies
 \Gamma,t=r \vdash\Gamma'
 \using
 {cut}
\endprooftree$$}}
\item[(R3)]
{\scriptsize{$$\prooftree
  \Gamma' \vdash t=r \qquad  \[\Gamma\vdash p=t\qquad p=t,t=r\vdash_{1.3} p=r\justifies \Gamma, t=r\vdash_{1.3} p=r\using {cut}\]
   \justifies
 \Gamma, \Gamma' \vdash p=r
 \using
 {cut}
\endprooftree$$}}
\end{enumerate}
\hspace{\stretch{1}} $\Box$\\
Usando la propriet\`a simmetria destra si dimostrano:
\begin{enumerate}
	\item[(S1)] $\vdash s(r+p)=r+s(p)$
	\vspace{.2cm}
	\item[(S2)] $\vdash r=r+0$
	\vspace{.2cm}
	\item[(S3)] $\vdash 0+t=t$
	\vspace{.2cm}
	\item[(S4)] $\vdash s(t)\neq 0$
\end{enumerate}
\vspace{.5cm}
\begin{prop}
Per ogni termine $p,\ r,\ t$ valgono:
\begin{enumerate}
	\item[(1.1)] $\vdash t=t$        (riflessivit\`a)
	\vspace{.2cm}
	\item[(1.2)] $t=r\vdash r=t$     (simmetria)
	\vspace{.2cm}
	\item[(1.3)] $p=t,t=r\vdash p=r$ (transitivit\`a)
	\vspace{.2cm}
	\item[(1.4)] $r=t,p=t\vdash r=p$
	\vspace{.2cm}
	\item[(1.5)] $t=r\vdash t+p=r+p$
	\vspace{.2cm}
	\item[(1.6)] $\vdash t=0+t$
	\vspace{.2cm}
	\item[(1.7)] $\vdash s(t)+p=s(t+p)$
	\vspace{.2cm}
	\item[(1.8)] $\vdash t+r=r+t$         (commutativa)
	\vspace{.2cm}
	\item[(1.9)] $t=r\vdash p+t=p+r$
	\vspace{.2cm}
	\item[(1.10)] $\vdash (t+r)+p=t+(r+p)$ (associativa)
	\vspace{.2cm}
	\item[(1.11)] $t=r\vdash t*p=r*p$
	\vspace{.2cm}
	\item[(1.12)] $\vdash 0*t=0$
	\vspace{.2cm}
	\item[(1.13)] $\vdash s(t)*r=t*r+r$
	\vspace{.2cm}
	\item[(1.14)] $\vdash t*r=r*t$         (commutativa)
	\vspace{.2cm}
	\item[(1.15)] $t=r\vdash p*t=p*r$
	\vspace{.2cm}
	\item[(1.16)] $\vdash (t*r)*p=t*(r*p)$ (associativa)
	\vspace{.2cm}
	\item[(1.17)] $p_1=t_1,p_2=t_2,p_1=p_2\vdash t_1=t_2$
	\vspace{.2cm}
	\item[(1.18)] $p_1=t_1,p_2=t_2\vdash p_1+p_2=t_1+t_2$
\end{enumerate}
\end{prop}
\vspace{1cm}
\textsc{Dimostrazione}
\vspace{.2cm}
\begin{enumerate}
\item[(1.1)] [\ $\vdash t=t$\ ]:
\par
{\scriptsize{
	$$\prooftree
	\vdash_{A3} t+0=t\qquad\[\vdash_{A3} t+0=t\qquad t+0=t,t+0=t\vdash_{U1} t=t\justifies t+0=t\vdash t=t\using{cut}\]\justifies \vdash t=t\using{cut}
	\endprooftree$$}}
\\
\item[(1.2)][\ $t=r\vdash r=t$\ ]:
\par
{\scriptsize{$$\prooftree
	\vdash_{1.1} t=t\qquad t=r,t=t\vdash_{U1} r=t \justifies t=r\vdash r=t\using{cut}
	\endprooftree$$}}
%\%justifies t=t,t=r\vdash r=t\using(ord)\
\\
\item[(1.3)][\ $p=t,t=r\vdash p=r$\ ]:
\par
{\scriptsize{	$$\prooftree
	p=t\vdash_{1.2} t=p\qquad t=p,t=r\vdash_{U1} p=r\justifies p=t,t=r\vdash p=r\using{cut}
	\endprooftree$$}}
\\
	\item[(1.4)][\ $r=t,p=t\vdash r=p$\ ]:
\par
{\scriptsize{	$$\prooftree
	p=t\vdash_{1.2} t=p\qquad r=t,t=p\vdash_{1.3} r=p\justifies r=t,p=t\vdash r=p\using{cut}
	\endprooftree$$}}
\\
\item[(1.5)][\ $t=r\vdash t+p=r+p$\ ]:
\vspace{0.5cm}
\\Per induzione su $p$.
\vspace{0.3cm}
\\In questo caso abbiamo: {\scriptsize{$$\varphi(p) :=\ t+p=r+p.$$}}
\\
Passo base:
\par
{\scriptsize{$$\prooftree
	\[\vdash_{A3} t+0=t\qquad t=r\vdash_{ax} t=r\justifies t=r\vdash t+0=r\using{tran}\]\qquad \vdash_{S2}r=r+0\justifies t=r\vdash t+0=r+0\using{tran}
	\endprooftree$$}}
\\
\vspace{2cm}
\\
Passo induttivo:
\par
{\scriptsize{$$\prooftree
	\vdash_{A4} t+s(p)=s(t+p)\qquad\[t=r,t+p=r+p\vdash_{U2}s(t+p)=s(r+p)\qquad \vdash_{S1} s(r+p)=r+s(p)\justifies t=r,t+p=r+p\vdash s(t+p)= r+s(p)\using{tran}\]\justifies t=r,t+p=r+p\vdash t+s(p)=r+s(p)\using{tran}
	\endprooftree$$}}
\\
\item[(1.6)][\ $\vdash t=0+t$\ ]:
\vspace{0.5cm}
\\Per induzione su $t$:
\vspace{0.3cm}
\\Passo base: {\scriptsize{$$\vdash_{S2} 0=0+0$$}}
\\Passo induttivo:
\par
{\scriptsize{$$\prooftree
	t=0+t\vdash_{U2} s(t)=s(0+t)\qquad\vdash_{S1} s(0+t)=0+s(t)\justifies t=0+t \vdash s(t)=0+s(t)\using{tran}
	\endprooftree$$}}
\\
\item[(1.7)][\ $\vdash s(t)+p=s(t+p)$\ ]:
\vspace{0.5cm}
\\Per induzione su $p$:
\vspace{0.3cm}
\\Passo base:
\par
{\scriptsize{$$\prooftree
	\vdash_{A3} s(t)+0=s(t)\qquad\[\vdash_{S2}t=t+0\qquad t=t+0\vdash_{U2} s(t)=s(t+0)\justifies \vdash s(t)=s(t+0)\using{cut}\]\justifies \vdash s(t)+0=s(t+0)\using{tran}
	\endprooftree$$}}
	\vspace{0.5cm}
\\Passo induttivo:
\vspace{0.3cm}
\\Dimostriamo prima (1A):
\par
{\scriptsize{$$\prooftree
	\[\vdash_{A4} t+s(p)=s(t+p)\qquad t+s(p)=s(t+p)\vdash_{U2} s(t+s(p))=s(s(t+p))\justifies\vdash s(t+s(p))=s(s(t+p))\using{cut}\]\justifies \vdash s(s(t+p))=s(t+s(p))\using{sim_{dx}}
	\endprooftree$$}}
\\
quindi si ha:
\par
{\tiny{$$\prooftree
	\vdash_{A4} s(t)+s(p)=s(s(t)+p)\qquad\[s(t)+p=s(t+p)\vdash_{U2}s(s(t)+p)=s(s(t+p))\qquad \vdash_{1A} s(s(t+p))=s(t+s(p))\justifies s(t)+p=s(t+p)\vdash s(s(t)+p)=s(t+s(p))\using{tran}\]\justifies s(t)+p=s(t+p)\vdash s(t)+s(p)=s(t+s(p))\using{tran}
	\endprooftree$$}}
\\
\item[(1.8)][\ $\vdash t+r=r+t$\ ]:
\vspace{0.5cm}
\\Per induzione su $r$.
\vspace{0.3cm}
\\Passo base:
\vspace{0.3cm}
{\scriptsize{$$\prooftree
	\vdash_{A3} t+0=t\qquad \vdash_{1.6} t=0+t\justifies \vdash t+0=0+t\using{tran}
	\endprooftree$$}}
\vspace{0.5cm}
\\Passo induttivo:
\vspace{0.3cm}
{\scriptsize{$$\prooftree \vdash_{A4}t+s(r)=s(t+r)\qquad\[\[\vdash_{1.7}s(r)+t=s(r+t)\qquad\[t+r=r+t\vdash_{U2}s(t+r)=s(r+t)\justifies t+r=r+t\vdash s(r+t)=s(t+r)\using{sim_{dx}}\]\justifies t+r=r+t\vdash s(r)+t=s(t+r)\using{tran}\]\justifies t+r=r+t\vdash s(t+r)=s(r)+t\using{sim_{dx}}\]\justifies t+r=r+t\vdash t+s(r)=s(r)+t\using{tran}
	\endprooftree$$}}
\\
\item[(1.9)][\ $t=r\vdash p+t=p+r$\ ]:
\vspace{0.5cm}
\\Per induzione su $p$.
\vspace{0.3cm}
\\Passo base:
\par
{\scriptsize{	$$\prooftree
	\vdash_{S3} 0+t=t\qquad\[\[\vdash_{S3} 0+r=r\qquad t=r\vdash_{1.2}r=t\justifies t=r\vdash 0+r=t\using{tran}\]\justifies t=r\vdash t=0+r\using{sim_{dx}}\]\justifies t=r\vdash 0+t=0+r\using{tran}
	\endprooftree$$}}
	\vspace{1cm}
\\Passo induttivo:
\vspace{0.3cm}
{\scriptsize{$$\prooftree
	\vdash_{1.7}s(p)+t=s(p+t)\qquad\[t=r,p+t=p+r\vdash_{U2}s(p+t)=s(p+r) \qquad\[\vdash_{1.7} s(p)+r=s(p+r)\justifies \vdash s(p+r)=s(p)+r\using{sim_{dx}}\]\justifies t=r,p+t=p+r\vdash s(p+t)=s(p)+r\using{tran}\]\justifies t=r,p+t=p+r\vdash s(p)+t=s(p)+r\using{tran}
	\endprooftree$$}}
\\
\item[(1.10)][\ $\vdash (t+r)+p=t+(r+p)$\ ]:
\vspace{0.5cm}
\\Per induzione su $p$.
\vspace{0.3cm}
\\Passo base:
\par
{\scriptsize{$$\prooftree
	\vdash_{A3} (t+r)+0=t+r\qquad \[\vdash_{S2} r=r+0\qquad r=r+0\vdash_{1.9}t+r=t+(r+0)\justifies \vdash t+r=t+(r+0)\using{cut}\]\justifies\vdash (t+r)+0=t+(r+0)\using{tran}
	\endprooftree$$}}
\vspace{0.5cm}
\\Passo induttivo:
\vspace{0.3cm}
\\Dimostriamo prima (1B)
\vspace{0.3cm}
  {\scriptsize{$$\prooftree
  \vdash_{A4} (t+r)+s(p)=s((t+r)+p)\qquad (t+r)+p=t+(r+p)\vdash_{U2} s((t+r)+p)=s(t+(r+p))\justifies (t+r)+p=t+(r+p)\vdash (t+r)+s(p)=s(t+(r+p))\using{tran}
	\endprooftree$$}}
	\vspace{3cm}
\\e (1C):
\vspace{0.2cm}
  {\tiny{$$\prooftree
  \vdash_{S1} s(t+(r+p))=t+s(r+p)\qquad\[\vdash_{S1} s(r+p)=r+s(p)\qquad s(r+p)=r+s(p)\vdash_{1.9} t+s(r+p)=t+(r+s(p))\justifies \vdash t+s(r+p)=t+(r+s(p))\using{cut}\]\justifies \vdash s(t+(r+p))=t+(r+s(p))\using{tran}
	\endprooftree$$}}
	\vspace{0.5cm}
\\quindi otteniamo:
\vspace{0.3cm}
{\scriptsize{	$$\prooftree
	(t+r)+p=t+(r+p)\vdash_{1B} (t+r)+s(p)=s(t+(r+p))\qquad\vdash_{1C} s(t+(r+p))=t+(r+s(p))\justifies (t+r)+p=t+(r+p)\vdash (t+r)+s(p)=t+(r+s(p))\using{tran}
	\endprooftree$$}}
\vspace{.3cm}
\item[(1.11)] [\ $t=r\vdash t*p=r*p$\ ]:
\vspace{.2cm}
\\Per induzione su $p$, analogamente a (1.5).
\item[(1.12)] [\ $\vdash 0*t=0$\ ]:
\vspace{.2cm}
\\Per induzione su $t$, analogamente a (1.6).
\vspace{0.5cm}
\item[(1.13)] [\ $\vdash s(t)*r=t*r+r$\ ]:
\vspace{.2cm}
\\Per induzione su $r$, analogamente a (1.7).
\vspace{0.5cm}
\item[(1.14)] [\ $\vdash t*r=r*t$ \ ]:
\vspace{.2cm}
\\Per induzione su $r$, analogamente a (1.8).
\vspace{0.5cm}
\item[(1.15)] [\ $t=r\vdash p*t=p*r$\ ]:
\vspace{.2cm}
\\Per induzione su $p$, analogamente a (1.9).
\vspace{0.5cm}
\item[(1.16)] [\ $\vdash (t*r)*p=t*(r*p)$\ ]:
\vspace{.2cm}
\\Per induzione su $p$, analogamente a (1.10).
\vspace{0.5cm}
\item[(1.17)] [\ $p_1=t_1,p_2=t_2,p_1=p_2\vdash t_1=t_2$\ ]:
\vspace{.2cm}
\\Si dimostra usando (1.3), (1.4) e (U1).
\vspace{0.5cm}
\item[(1.18)] [\ $p_1=t_1,p_2=t_2\vdash p_1+p_2=t_1+t_2$\ ]:
\vspace{.2cm}
\\Come il punto precedente\end{enumerate}
\hspace{\stretch{1}} $\Box$\\


\newpage
\begin{prop}
Per ogni termine $p,\ r,\ t,$ valgono:
\begin{enumerate}
	\item[(2.1)] $\vdash t*(r+p)=(t*r)+(t*p)$ (distributiva)
	\vspace{.2cm}
	\item[(2.2)] $\vdash (r+p)*t=(r*t)+(p*t)$ (distributiva)
	\vspace{.2cm}
	\item[(2.3)] $t+p=r+p\vdash t=r$ (cancellativa)
\end{enumerate}
\end{prop}
\vspace{.5cm}
\textsc{Dimostrazione}
\vspace{.2cm}
\begin{enumerate}
\item[(2.1)] [\ $\vdash t*(r+p)=(t*r)+(t*p)$\ ]:
\vspace{.5cm}
\\Per induzione su $p$.
\vspace{0.5cm}
\\Passo base:
\vspace{.2cm}
\\Dimostriamo prima (2A)
\vspace{.2cm}
	{\scriptsize{$$\prooftree
	\[\[\vdash_{A5}t*0=0\qquad t*0=0\vdash_{1.9}t*r+t*0=t*r+0\justifies\vdash (t*r)+(t*0)=t*r+0\using{cut}\]\qquad\vdash_{A3}t*r+0=t*r\justifies\vdash (t*r)+(t*0)=t*r\using{tran}\]\justifies\vdash t*r=(t*r)+(t*0)\using{sim_{dx}}
	\endprooftree$$}}\\
da cui segue
\vspace{.2cm}
	{\scriptsize{$$\prooftree
	\[\vdash_{A3}r+0=r\qquad r+0=r\vdash_{1.15}t*(r+0)=t*r\justifies\vdash t*(r+0)=t*r\using{cut}\]\quad\vdash_{2A} t*r=(t*r)+(t*0)\justifies\vdash t*(r+0)=(t*r)+(t*0)\using{tran}
	\endprooftree.$$}}
	\vspace{.5cm}
\\Passo induttivo:
\vspace{.2cm}
\\Ci serviranno inoltre:
\vspace{.2cm}
\\(2B)
{\scriptsize{	$$\prooftree
	\[\vdash_{A6}(t*s(p))=(t*p)+t\qquad (t*s(p))=(t*p)+t\vdash_{1.9}(t*r)+(t*s(p))=(t*r)+(t*p)+t\justifies\vdash  (t*r)+(t*s(p))=(t*r)+(t*p)+t\using{cut}\]
	\justifies \vdash(t*r)+(t*p)+t=(t*r)+(t*s(p))\using{sim_{dx}}
	\endprooftree$$	}}
	\vspace{.2cm}
	\\(2C)
	{\tiny{$$\prooftree
	t*(r+p)=(t*r)+(t*p)\vdash_{1.5} t*(r+p)+t=(t*r)+(t*p)+t\qquad
	\vdash_{2B}(t*r)+(t*p)+t=(t*r)+(t*s(p))\using{tran}
	\justifies t*(r+p)=(t*r)+(t*p)\vdash t*(r+p)+t=(t*r)+(t*s(p))
	\endprooftree$$	}}
	\vspace{.2cm}
	\\(2D)
	{\scriptsize{$$\prooftree
	\vdash_{A6}t*s(r+p)=t*(r+p)+t\qquad
	t*(r+p)=(t*r)+(t*p)\vdash_{2C} t*(r+p)+t=(t*r)+(t*s(p))
	\justifies t*(r+p)=(t*r)+(t*p)\vdash t*s(r+p)=(t*r)+(t*s(p))\using{tran}
	\endprooftree$$}}
	\vspace{.2cm}
	\\e (2E)
{\scriptsize{	$$\prooftree
	\vdash_{A4}r+s(p)=s(r+p)\qquad r+s(p)=s(r+p)\vdash_{1.15}t*(r+s(p))=t*s(r+p)\justifies \vdash t*(r+s(p))=t*s(r+p)\using{cut}
	\endprooftree$$}}
	\vspace{.2cm}
	\\da cui segue
	{\scriptsize{$$\prooftree
	\vdash_{2E} t*(r+s(p))=t*s(r+p)\qquad
		t*(r+p)=(t*r)+(t*p)\vdash_{2D} t*s(r+p)=(t*r)+(t*s(p))
	\justifies t*(r+p)=(t*r)+(t*p)\vdash t*(r+s(p))=(t*r)+(t*s(p))\using{tran}
	\endprooftree$$}}
	\vspace{.5cm}
\item[(2.2)] [\ $\vdash (r+p)*t=(r*t)+(p*t)$\ ]:
\vspace{.2cm}
\\Similmente al punto precedente, per induzione su $p$; oppure si sfruttano il punto precedente (2.1), la commutativit\`a del prodotto (1.14), e la (1.18).
\vspace{.5cm}
\item[(2.3)] [\ $t+p=r+p\vdash t=r$\ ]:
\vspace{.2cm}
\\Per induzione su $p$.
\vspace{.2cm}
\\Passo base:

	{\scriptsize{$$\prooftree\vdash_{A3} t+0=t\qquad\[t+0=t,t+0=r+0\vdash_{U1}t=r+0\qquad\[\vdash_{A3}r+0=r\qquad t=r+0, r+0=r\vdash_{1.3} t=r\justifies t=r+0\vdash t=r\using{cut}\]\justifies t+0=t, t+0=r+0\vdash t=r\using{cut}\]\justifies t+0=r+0\vdash t=r\using{cut}
	\endprooftree$$}}
\vspace{.5cm}
\\Passo induttivo:
\vspace{.2cm}
\\Dimostriamo prima (2F)

$$
\tiny{
\prooftree
\[\vdash_{A4} t+s(p)=s(t+p) \qquad t+s(p)=s(t+p),t+s(p)=r+s(p)\vdash_{U1} s(t+p)=r+s(p) \justifies t+s(p)=r+s(p) \vdash s(t+p)=r+s(p)\using_{cut}\]\qquad \vdash_{A4} r+s(p)=s(r+p)\justifies t+s(p)=r+s(p) \vdash s(t+p)=s(r+p)\using_{tran}\
\endprooftree}
$$\\
\vspace{.2cm}
da cui (2G):
$${\prooftree
 t+s(p)=r+s(p)\vdash_{2F} s(t+p)=s(r+p) \qquad s(t+p)=s(r+p)\vdash_{A2}t+p=r+p\justifies t+s(p)=r+s(p)\vdash t+p=r+p\using{cut}
\endprooftree}
$$
\\quindi si ha:
\vspace{.2cm}
{\scriptsize{$$\prooftree t+s(p)=r+s(p)\vdash_{2G} t+p=r+p\qquad t+s(p)=r+s(p),t=r \vdash_{ax} t=r\justifies t+p=r+p\rightarrow t=r,t+s(p)=r+s(p)\vdash t=r\using{\rightarrow_{left}}
\endprooftree$$}}
\end{enumerate}
\hspace{\stretch{1}} $\Box$


\vspace{.6cm}

\begin{defi}
Chiameremo \underline{numerali} ($\mathcal{N}$) i termini $0$,\ $s(0)$,\ $s(s(0))$,\ $s(s(s(0)))$ e cos\`i via e li denoteremo con $\overline{0},\ \overline{1},\ \overline{2},\ \overline{3}$ etc. Pi\`u precisamente $\overline{0}$ \`e $0$ e per ogni naturale $n$, $\overline{n+1}$ \`e $s(\overline{n})$, ossia il numerale $\overline{n}$, con $n$ naturale, sta per $0$ a cui si applica $n$ volte la funzione $s$. E\-qui\-va\-len\-te\-men\-te, i numerali si possono definire induttivamente tramite le regole:
\begin{enumerate}
\item[-]$0$ \`e un numerale
\item[-]se $u$ \`e un numerale, allora anche $s(u)$ lo \`e.
\end{enumerate}
\end{defi}

In pratica un numerale \`e la formalizzazione, nel nostro sistema, di un numero naturale; si noti per\`o non rientra nel linguaggio come termine, perch\'e rappresenta una abbreviazione della scrittura corretta pi\`u formale $sss\dots s(0)$.

I numerali vengono introdotti qui per poter enunciare alcune proprietᅵ in cui vengono coinvolti dei naturali specifici, e bisogna indicarli con i termini corrispondenti.
\newpage
\begin{prop}
Per ogni termine $\ p,\ r,\ t$ valgono:
\begin{enumerate}
	\item[(3.1)] $\vdash t+\overline{1}=s(t)$
	\vspace{.2cm}
	\item[(3.2)] $\vdash t*\overline{1}=t$
	\vspace{.2cm}
	\item[(3.3)] $\vdash t*\overline{2}=t+t$
	\vspace{.2cm}
	\item[(3.4)] $t+p=0\vdash t=0\ \mbox{\emph{\&}}\ p=0$
	\vspace{.2cm}
	\item[(3.5)] $t\neq 0,p*t=0\vdash p=0$
	\vspace{.2cm}
	\item[(3.6)] $t+p=\overline{1}\vdash (t=0\ \mbox{\emph{\&}}\ p=\overline{1})\ \vee\ (t=\overline{1}\ \mbox{\emph{\&}}\ p=0)$
	\vspace{.2cm}
	\item[(3.7)] $t*p=\overline{1}\vdash t=\overline{1}\ \mbox{\emph{\&}}\ p=\overline{1}$
	\vspace{.2cm}
	\item[(3.8)] $t\neq 0\vdash \exists\ y (t=s(y))$
	\vspace{.2cm}
	\item[(3.9)] $p\neq 0,t*p=r*p\vdash t=r$
	\vspace{.2cm}
	\item[(3.10)] $t\neq 0,t\neq \overline{1}\vdash \exists\ y(t=s(s(y)))$
\end{enumerate}
\end{prop}
\vspace{.5cm}
\textsc{Dimostrazione}\\
 Per la definizione di nume\-rale, possiamo sostituire il termine $\overline{n}$ con ${\overbrace{sss\dots s(0)}^{n\ volte}}$.
\begin{enumerate}
\vspace{.2cm}
	\item[(3.1)] [ $\vdash t+\overline{1}=s(t)$ ]:
	\vspace{.2cm}
{\scriptsize{	$$\prooftree
	\vdash_{A4}t+s(0)=s(t+0)\qquad \[\vdash_{A3}t+0=t\qquad t+0=t\vdash_{U2}s(t+0)=s(t)\justifies \vdash s(t+0)=s(t)\using{cut}\]\justifies \vdash t+s(0)=s(t)\using{tran}
	\endprooftree$$}}
	\vspace{.2cm}
	\item[(3.2)] [ $\vdash t*\overline{1}=t$ ]:
	\vspace{.2cm}
{\scriptsize{	$$\prooftree
	\vdash_{A6} t*s(0)=t*0+t\qquad\[\[\vdash_{A5}t*0=0\qquad t*0=0\vdash_{1.5}t*0+t=0+t\justifies\vdash t*0+t=0+t\using{cut}\]\qquad\vdash_{S3} 0+t=t\justifies \vdash t*0+t=t\using{tran}\]\justifies\vdash t*s(0)=t\using{tran}
	\endprooftree$$}}
	\vspace{.2cm}
	\item[(3.3)] [ $\vdash t*\overline{2}=t+t$ ]:
	\vspace{.2cm}
	\\Basta usare il punto precedente e A6.
	\vspace{.2cm}
	\item[(3.4)] [ $t+p=0\vdash t=0\ \mbox{\&} \ p=0$ ]:
	\vspace{.2cm}
	\\Per induzione su $p$.
	\vspace{.2cm}
\\Passo base:
\vspace{.2cm}
{\scriptsize{$$\prooftree
	\[\vdash_{A3}t+0=t\qquad t+0=t,t+0=0\vdash_{U1}t=0\justifies t+0=0\vdash t=0\using{cut}\]\qquad\[\vdash_{1.1} 0=0\justifies t+0=0\vdash 0=0\using{ind}\]\justifies t+0=0\vdash t=0\mbox{\&}\ 0=0\using{\mbox{\&}_{right}}
	\endprooftree$$}}
	\vspace{.5cm}
	\\Per il passo induttivo, dimostriamo (3A)
	\vspace{.2cm}
	{\scriptsize{$$\prooftree
	\[\vdash_{S1}s(t+p)=t+s(p)\qquad s(t+p)=t+s(p),t+s(p)=0\vdash_{1.3}s(t+p)=0\justifies t+s(p)=0\vdash s(t+p)=0\using{cut}\]\qquad\[0=s(t+p)\vdash_{A1}\bot\justifies s(t+p)=0\vdash\bot\using{sim_{sx}}\]\justifies t+s(p)=0\vdash\bot\using{cut}
	\endprooftree$$}}
	\vspace{.2cm}
	\\da cui segue:
	\vspace{.2cm}
{\scriptsize{	$$\prooftree
	\[t+s(p)=0\vdash_{3A}\bot\qquad\bot\vdash t=0\justifies t+s(p)=0\vdash t=0\using{cut}\]\[t+s(p)=0\vdash_{3A}\bot\qquad\bot\vdash s(p)=0\justifies t+s(p)=0\vdash s(p)=0\using{cut}\]\justifies t+p=0\rightarrow t=0\mbox{\&} p=0, t+s(p)=0\vdash t=0\mbox{\&} s(p)=0\using{\mbox{\&}_{right}+ind}
	\endprooftree$$}}
	\vspace{.5cm}
	\item[(3.5)] [ $t\neq 0,p*t=0\vdash p=0$ ]:
	\vspace{.2cm}
	\\Per induzione su $p$.
	\vspace{.2cm}
\\Passo base:
\vspace{.2cm}
{\scriptsize{$$\prooftree
	\vdash_{1.1} 0=0\justifies t\neq 0,0*t=0\vdash 0=0\using{ind}
	\endprooftree$$}}
	\vspace{.5cm}\\
	\\Per dimostrare il passo induttivo, proviamo prima (3B)
\vspace{.2cm}
{\scriptsize{	$$\prooftree
	p*t+t=0\vdash_{3.4} p*t=0\mbox{\&} t=0 \qquad\[\[t=0 \vdash_{ax}t=0\justifies p*t=0\mbox{\&}t=0\vdash t=0\using{\mbox{\&}_{left}}\]\justifies  \justifies p*t=0\mbox{\&} t=0, t\neq 0\vdash\bot\using{\neg_{left}}\]\justifies p*t+t=0, t\neq 0\vdash\bot\using{cut}
	\endprooftree$$}}
		\vspace{.2cm}
	\\e (3C)
	\vspace{.2cm}
{\scriptsize{$$\prooftree
	s(p)*t=p*t+t,s(p)*t=0\vdash_{U1}p*t+t=0\qquad p*t+t=0, t\neq 0\vdash_{3B}\bot\justifies s(p)*t=p*t+t,s(p)*t=0, t\neq 0\vdash\bot\using{cut}
	\endprooftree$$}}
\vspace{.2cm}
\\da cui segue:
	{\scriptsize{$$\prooftree
	\[\[\vdash_{1.13}s(p)*t=p*t+t\qquad s(p)*t=p*t+t,s(p)*t=0, t\neq 0\vdash_{3C}\bot\justifies  t\neq 0,s(p)*t=0\vdash\bot\using{cut}\]\qquad\bot\vdash s(p)=0\justifies t\neq 0,s(p)*t=0\vdash s(p)=0\using{cut}\]\justifies t\neq 0,p*t=0\rightarrow p=0, t\neq 0,s(p)*t=0\vdash s(p)=0\using{ind}
	\endprooftree$$}}
	\vspace{.5cm}
	\item[(3.6)] [ $t+p=\overline{1}\vdash (t=0\ \mbox{\&} \ p=\overline{1})\ \vee\ (t=\overline{1}\ \mbox{\&} \ p=0)$ ]:
	\vspace{.2cm}
	\\Per induzione su $p$.
	\vspace{.2cm}
\\Passo base:
\vspace{.2cm}
	{\scriptsize{$$\prooftree
	\[\[\vdash_{A3} t+0=t\qquad t+0=t,t+0=s(0)\vdash_{U1}t=s(0)\justifies t+0=s(0)\vdash t=s(0)\using{cut}\]\[\vdash_{1.1}0=0\justifies t+0=s(0)\vdash 0=0\using{ind}\]\justifies t+0=s(0)\vdash t=s(0)\mbox{\&} 0=0\using{\mbox{\&}_{right}}\]\justifies t+0=s(0)\vdash (t=0\mbox{\&} 0=s(0))\vee (t=s(0)\mbox{\&} 0=0)\using{\vee_{right}}
	\endprooftree$$}}
	\vspace{.2cm}
	\\Per il passo induttivo dimostriamo prima:
	\vspace{.2cm}
  \\(3D)
  \vspace{.2cm}
	{\scriptsize{$$\prooftree
	\vdash_{A4}t+s(p)=s(t+p)\qquad\[t+s(p)=s(t+p),t+s(p)=s(0)\vdash_{U1} s(t+p)=s(0)\qquad s(t+p)=s(0)\vdash_{A2}t+p=0\justifies t+s(p)=s(t+p),t+s(p)=s(0)\vdash t+p=0\using{cut}\]\justifies t+s(p)=s(0)\vdash t+p=0\using{cut}
	\endprooftree$$	}}
	\vspace{.2cm}
	\\(3E)
	{\scriptsize{$$\prooftree
	t+s(p)=s(0)\vdash_{3D}t+p=0\qquad\[t+p=0\vdash_{3.4}t=0\mbox{\&} p=0\qquad\[t=0\vdash_{ax} t=0\justifies t=0\mbox{\&} p=0\vdash t=0\using{\mbox{\&}_{left}}\]\justifies t+p=0\vdash t=0\using{cut}\]\justifies t+s(p)=s(0)\vdash t=0\using{cut}
	\endprooftree$$}}
	\vspace{.2cm}
	\\(3F)
	\vspace{.2cm}
{\scriptsize{	$$\prooftree
	t+s(p)=s(0)\vdash_{3D}t+p=0\qquad\[t+p=0\vdash_{3.4}t=0\mbox{\&} p=0\qquad\[p=0\vdash_{U2} s(p)=s(0)\justifies t=0\mbox{\&} p=0\vdash s(p)=s(0)\using{\mbox{\&}_{left}}\]\justifies t+p=0\vdash s(p)=s(0)\using{cut}\]\justifies t+s(p)=s(0)\vdash s(p)=s(0)\using{cut}
	\endprooftree$$}}
\vspace{.2cm}
	\\e quindi segue:
	{\tiny{$$\prooftree
	\[t+s(p)=s(0)\vdash_{3E}t=0\qquad t+s(p)=s(0)\vdash_{3F}s(p)=s(0)\justifies t+s(p)=s(0)\vdash t=0\mbox{\&} s(p)=s(0)\using{\mbox{\&}_{right}}\]\justifies t+p=s(0)\rightarrow(t=0\mbox{\&} p=s(0))\vee(t=s(0)\mbox{\&} p=0),t+s(p)=s(0)\vdash(t=0\mbox{\&} s(p)=s(0))\vee(t=s(0)\mbox{\&} s(p)=0)\using{ind+\vee_{right}}
	\endprooftree$$}}

	\vspace{.5cm}
	\item[(3.7)] [ $t*p=\overline{1}\vdash t=\overline{1}\ \mbox{\&} \ p=\overline{1}$ ]:
	\vspace{.2cm}
	\\Per induzione su $p$, usando il punto precedente.
	\vspace{.5cm}
	\item[(3.8)] [ $t\neq 0\vdash \exists\ y (t=s(y))$ ]:
	\vspace{.2cm}
	\\Per induzione su $t$.
	\vspace{.2cm}
\\Passo base:
\vspace{.2cm}
	{\scriptsize{$$\prooftree
	\[\vdash_{1.1}0=0\justifies  0\neq 0\vdash\bot\using{\neg_{left}}\]\qquad\bot\vdash\exists y(0=s(y))\justifies 0\neq 0\vdash\exists y(0=s(y))\using{cut}
	\endprooftree$$}}
	\vspace{.2cm}
\\Passo induttivo:
\vspace{.2cm}
{\scriptsize{$$\prooftree
	\[\vdash_{1.1} s(t)=s(t)\justifies\vdash\exists y(s(t)=s(y))\using{\exists_{right}}\]\justifies t\neq 0\rightarrow\exists y(t=s(y)), s(t)\neq 0\vdash\exists y(s(t)=s(y))\using{ind}
	\endprooftree$$}}
	\vspace{.5cm}
	\item[(3.9)] [ $p\neq 0,t*p=r*p\vdash t=r$ ]:
	\vspace{.2cm}
	\\Per induzione su $t$, utilizzando anche la (3.5).
	\vspace{.5cm}
	\item[(3.10)] [ $t\neq 0,t\neq \overline{1}\vdash \exists\ y(t=s(s(y)))$ ]:
	\vspace{.2cm}
	\\Per induzione su $t$, usando anche la (3.8).
\end{enumerate}
\hspace{\stretch{1}} $\Box$\\
\vspace{1cm}
\begin{prop}
Siano $m$ e $n$ due numeri naturali qualsiasi. Allora si ha:
\begin{enumerate}
  \item[(a.)] Se $m=n$ allora $\vdash \overline{m} = \overline{n}$
  \vspace{.2cm}
	\item[(b.)] Se $m\neq n$ allora $\vdash \overline{m} \neq \overline{n}$
	\vspace{.2cm}
	\item[(c.)] $\vdash \overline{m+n} = \overline{m} + \overline{n}$ e $\vdash \overline{m*n} = \overline{m} * \overline{n}$
\end{enumerate}
\vspace{.5cm}
\end{prop}
\textsc{Dimostrazione}
\begin{enumerate}
\vspace{.2cm}
	\item[(a.)] Ovvio, dalla definizione di numerale.
	\item[(b.)] Sia $p=|m-n|\neq 0$ in quanto $n\neq m$ per ipotesi. Supponendo $m>n$,
	{\scriptsize{$$\prooftree
	\[\overline{m}=\overline{n}\vdash_{A2} \overline{m-1}=\overline{n-1}\qquad\[\[s(\overline{p})=s(0)\vdash_{A2}\overline{p}=0\qquad\overline{p}=0\vdash\bot\justifies ...\]\justifies \overline{m-1}=\overline{n-1}\vdash\bot\using{A2\ (n-2)-volte}\]\justifies \overline{m}=\overline{n}\vdash\bot\using{cut}\]\justifies \vdash \overline{m}\neq \overline{n}\using{\neg_{right}}
	\endprooftree$$}}
	\vspace{.2cm}
	\item[(c.)] Dimostriamo la prima parte per induzione su $n$ (per la seconda si procede analogamente).
	\vspace{.2cm}
\\Passo base:
\vspace{.2cm}
	{\scriptsize{$$\prooftree \vdash_{A3}\overline{m+0}=\overline{m}\qquad\vdash_{S2}\overline{m}=\overline{m}+\overline{0}\justifies\vdash\overline{m+0}=\overline{m}+\overline{0}\using{tran}
	\endprooftree$$}}
	\vspace{.2cm}
\newline
Passo induttivo:
\vspace{.2cm}
{\scriptsize{$$\prooftree	 \vdash_{A4+a}\overline{m+s(n)}=\overline{s(m+n)}\qquad\[\[\vdash_{A4}\overline{m}+\overline{s(n)}=s(\overline{m}+\overline{n})\quad\[\overline{m+n}=\overline{m}+\overline{n}\vdash_{U2}\overline{s(m+n)}=s(\overline{m}+\overline{n})\justifies \overline{m+n}=\overline{m}+\overline{n}\vdash s(\overline{m}+\overline{n})=\overline{s(m+n)}\using{sim_{dx}*}\]\justifies\overline{m+n}=\overline{m}+\overline{n}\vdash\overline{m}+\overline{s(n)}=\overline{s(m+n)}\using{tran}\]\justifies\overline{m+n}=\overline{m}+\overline{n}\vdash\overline{s(m+n)}=\overline{m}+\overline{s(n)}\using{sim_{dx}}\]\justifies \overline{m+n}=\overline{m}+\overline{n}\vdash\overline{m+s(n)}=\overline{m}+\overline{s(n)}\using{tran}
	\endprooftree$$}}
	\vspace{.5cm}
\\In quest'ultima dimostrazione nel punto * si nota facilmente che vale l'uguaglianza $\overline{s(m+n)}=s(\overline{m+n})$. Infatti, per esempio $s(\overline{2})=s(s(s(0)))$ ma anche $\overline{s(2)}=s(s(s(0)))$.

\end{enumerate}
\hspace{\stretch{1}} $\Box$
\vspace{.5cm}
\\

\texttt{Osservazione}\\
\vspace{.2cm}

Ci\`o che abbiamo affermato nella proposizione 3.4 \`e di fondamentale importanza: esprime il fatto che il sistema formale HA, di fronte alle operazioni di somma e prodotto applicate ai \underline{defi}, sa procedere esattamente come siamo in grado di fare noi sui \underline{naturali}, ovvero \`e stato istruito cos\`i bene che possiamo dimostrare che calcola esattamente ci\`o che noi ci aspettiamo . Quindi, nonostante la netta distinzione tra noi e il robot, aver dimostrato questa sorta di parallelismo ci garantisce l'efficienza (non in senso computazionale, ma come coerenza, buona costruzione) del sistema formale: questo, infatti, \`e stato istruito bene a tal punto da saper riconoscere, a partire da un insieme finito di assiomi e regole, qualsiasi altra espressione con i numerali, e conseguentemente, sa agire su essa proprio come noi vogliamo. La distinzione rimane comunque marcata in quanto noi abbiamo istruito il robot, e possiamo riconoscere un senso a ciᅵ che elabora, aggiungendo l'interpretazione, mentre quest'ultimo resta solo un esecutore.

\vspace{.5cm}

\begin{defi}
Definiamo le relazioni d'ordine:
\vspace{.2cm}
\begin{itemize}
\item $t<z :=\ \exists\ w(w\neq 0\ \mbox{\emph{\&}}\ w+t=z)$ o, equivalentemente, $t<z :=\ \exists\ w(s(w)+t=z)$
\vspace{.2cm}
\item $t\leq z\ :=\ t<z\ \vee\ t=z$, cio\`e $t\leq z\ :=\ \exists\ w(w+t=z)$
\vspace{.2cm}
\item $t>z\ :=\ z<t$
\vspace{.2cm}
\item $t\geq z\ :=\ z\leq t$
\vspace{.2cm}
\item $t\not< z\ :=\ \neg(t<z).$
\end{itemize}
\end{defi}

\vspace{.5cm}
\begin{prop}
Per ogni termine $p,\ r,\ t$ valgono:
\vspace{.2cm}
\begin{enumerate}
	\item[(5.1)] $ \vdash t\not< t$ (antiriflessiva)
	\vspace{.2cm}
	\item[(5.2)] $t<p, p<r\vdash t<r$ (transitiva)
	\vspace{.2cm}
	\item[(5.3)] $t<p\vdash p\not< t$ (antisimmetrica)
	\vspace{.2cm}
	\item[(5.4)] $\vdash t<p\ \leftrightarrow\ t+r<p+r$
	\vspace{.2cm}
	\item[(5.5)] $\vdash t\leq t$
	\vspace{.2cm}
	\item[(5.6)] $t\leq p,p\leq r\vdash t\leq r$
	\vspace{.2cm}
	\item[(5.7)] $\vdash t\leq p\ \leftrightarrow\ t+r\leq p+r$
	\vspace{.2cm}
	\item[(5.8)] $t\leq p,p<r\vdash t<r$
	\vspace{.2cm}
	\item[(5.9)] $\vdash 0\leq t$
	\vspace{.2cm}
	\item[(5.10)] $\vdash 0<s(t)$
	\vspace{.2cm}
	\item[(5.11)] $\vdash t<r\ \leftrightarrow\ s(t)\leq r$
	\vspace{.2cm}
	\item[(5.12)] $\vdash t\leq r\ \leftrightarrow\ t<s(r)$
	\vspace{.2cm}
	\item[(5.13)] $\vdash t<s(t)$
	\vspace{.2cm}
	\item[(5.14)] $\vdash 0<\overline{1},\ \overline{1}<\overline{2},...$
	\vspace{.2cm}
	\item[(5.15)] $t\neq r\vdash t<r\ \vee\ r<t$
	\vspace{.2cm}
	\item[(5.16)] $\vdash t=r\ \vee\ t<r\ \vee\ r<t$
	\vspace{.2cm}
	\item[(5.17)] $\vdash t\leq r\ \vee\ r\leq t$
	\vspace{.2cm}
	\item[(5.18)] $\vdash t+r\geq t$
	\vspace{.2cm}
	\item[(5.19)] $r\neq 0\vdash t+r>t$
	\vspace{.2cm}
	\item[(5.20)] $r\neq 0\vdash t*r\geq t$
	\vspace{.2cm}
	\item[(5.21)] $\vdash r\neq 0\ \leftrightarrow\ r>0$
	\vspace{.2cm}
	\item[(5.22)] $r>0,t>0\vdash r*t>0$
	\vspace{.2cm}
	\item[(5.23)] $r\neq 0,t>\overline{1}\vdash t*r>r$
	\vspace{.2cm}
	\item[(5.24)] $r\neq 0\vdash t<p\ \leftrightarrow\ t*r<p*r$
	\vspace{.2cm}
	\item[(5.25)] $r\neq 0\vdash t\leq p\ \leftrightarrow\ t*r\leq p*r$
	\vspace{.2cm}
	\item[(5.26)] $\vdash t\not< 0$
	\vspace{.2cm}
	\item[(5.27)] $t\leq r\ \mbox{\emph{\&}}\ r\leq t\vdash t=r$
\end{enumerate}
\end{prop}

\vspace{.5cm}
\textsc{Dimostrazione}
\begin{enumerate}
	\item[(5.1)] [ $ \vdash t\not< t$ (antiriflessiva) ]:
	\vspace{.2cm}
	\\Basta applicare la definizione di $\not<$ e quanto precedentemente mostrato in (1.6) e (2.3).
	\vspace{.5cm}
	\item[(5.2)] [ $t<p, p<r\vdash t<r$ (transitiva) ]:
	\vspace{.2cm}
	\\Applicare la definizione di $<$, utilizzare i punti (1.8) e (1.5) e scegliere $w=v+z$ nell'$\exists_{right}$, dove $v$ e $z$ vengono scelte rispettivamente nell'$\exists_{left}$ in $t<p$ e $p<r$.
	\vspace{.5cm}
	\item[(5.3)] [$t<p\vdash p\not< t$ (antisimmetrica) ]:
	\vspace{.2cm}
	\\Proviamo prima (5A)

	{\scriptsize{$$\prooftree
	\vdash_{1.6}t=0+t\qquad \[v+w+t=t,t=0+t\vdash_{1.3}v+w+t=0+t\qquad v+w+t=0+t\vdash_{2.4}v+w=0\justifies v+w+t=t,t=0+t\vdash v+w=0\using{cut}\]\justifies v+w+t=t\vdash v+w=0\using{cut}
	\endprooftree$$	}}
\vspace{.5cm}
	(5B)
{\scriptsize{$$\prooftree
	v+w+t=t\vdash_{5A} v+w=0\qquad\[v+w=0\vdash_{3.4}v=0\mbox{\&} w=0\qquad\[\[v=0\vdash_{ax} v=0\justifies v=0,v\neq 0\vdash\bot\using{\neg_{left}}\]\justifies v=0\mbox{\&} w=0,v\neq 0\vdash\bot\using{\mbox{\&}_{left}}\]\justifies v+w=0,v\neq 0\vdash\bot\using{cut}\]\justifies v+w+t=t,v\neq 0\vdash\bot\using{cut}
	\endprooftree$$}}	\vspace{.2cm}
	e (5C)
	{\tiny{$$\prooftree
	p=w+t\vdash_{1.9}v+p=v+w+t\qquad\[v+p=v+w+t,v+p=t\vdash_{U1}v+w+t=t\qquad v+w+t=t,v\neq 0\vdash_{5B}\bot\justifies v\neq 0,v+p=v+w+t,v+p=t\vdash\bot\using{cut}\]\justifies p=w+t,v\neq 0,v+p=t\vdash\bot\using{cut}
	\endprooftree$$	}}
	\vspace{.2cm}
{\scriptsize{$$\prooftree
	\[\[w+t=p\vdash_{1.2}p=w+t\quad w\neq 0,p=w+t,v\neq 0,v+p=t\vdash_{5C}\bot\justifies w\neq 0,w+t=p,v\neq 0\mbox{\&} v+p=t\vdash\bot\using{cut+\mbox{\&}_{left}}\]\justifies\exists w(w\neq 0\mbox{\&}\ w+t=p),\exists v(v\neq 0\mbox{\&} v+p=t)\vdash\bot\using{\exists_{left}+\mbox{\&}_{left}}\]\justifies \exists w(w\neq 0\mbox{\&} w+t=p)\vdash\neg\exists v(v\neq 0\mbox{\&} v+p=t)\using{\neg_{right}}
	\endprooftree$$}}
	\vspace{.5cm}
	\item[(5.4)] [ $\vdash t<p\ \leftrightarrow\ t+r<p+r$ ]:
	\vspace{.2cm}
	\\Bisogna mostrare le due implicazioni \textsl{separatamente}, applicando la definizione di $<$ e scegliendo in entrambi i casi $w$ dell'$\exists_{right}$ uguale alla variabile $v$ ottenuta applicando l'$\exists_{left}$.
	\vspace{.5cm}
	\item[(5.5)] [ $\vdash t\leq t$ ]:
	\vspace{.2cm}
	\\Basta applicare la definizione di $\leq$ e scegliere, nell'$\vee_{right}$, $t=t$, dimostrata in precedenza nel punto(1.1).
	\vspace{.5cm}
	\item[(5.6)] [ $t\leq p,p\leq r\vdash t\leq r$ ]:
	\vspace{.2cm}
	\\Basta applicare la definizione di $\leq$, usando i punti (5.2) e (1.3).
	\vspace{.5cm}
	\item[(5.7)] [ $\vdash t\leq p\ \leftrightarrow\ t+r\leq p+r$ ]:
	\vspace{.2cm}
\begin{itemize}
	\item[($\leftarrow$)]
{\scriptsize{	$$\prooftree
	\[t+r<p+r\vdash_{5.4}t<p\justifies t+r<p+r\vdash t<p\vee t=p\using{\vee_{right}}\]\[t+r=p+r\vdash_{2.3}t=p\justifies t+r=p+r\vdash t<p\vee t=p\using{\vee_{right}}\]\justifies t+r<p+r\vee t+r=p+r\vdash t<p\vee t=p\using{\vee_{left}}
	\endprooftree$$}}
	\vspace{.2cm}
	\item[($\rightarrow$)] Proviamo prima (5D)
	{\scriptsize{$$\prooftree
	\[\[\[\[w\neq 0\vdash_{ax} w\neq 0\justifies w\neq 0\mbox{\&} w+t=p\vdash w\neq 0\using{\mbox{\&}_{left}}\]\quad\[w+t=p\vdash_{1.5}w+t+r=p+r\justifies w\neq 0\mbox{\&} w+t=p\vdash w+t+r=p+r\using{\mbox{\&}_{left}}\]\justifies w\neq 0\mbox{\&} w+t=p\vdash w\neq 0\mbox{\&} w+t+r=p+r\using{\mbox{\&}_{right}}\]\justifies w\neq 0\mbox{\&} w+t=p\vdash\exists v(v\neq 0\mbox{\&} v+t+r=p+r)\using{\exists_{right}}\]\justifies \exists w(w\neq 0\mbox{\&} w+t=p)\vdash\exists v(v\neq 0\mbox{\&} v+t+r=p+r)\using{\exists_{left}}\]\justifies\exists w(w\neq 0\mbox{\&} w+t=p)\vdash (t+r<p+r)\vee (t+r=p+r)\using{def.\leq }
	\endprooftree$$}}
	\vspace{.4cm}
	\\da cui segue:
	\vspace{.2cm}
	{\scriptsize{$$\prooftree
  \exists w(w\neq 0\mbox{\&} w+t=p)\vdash_{5D} (t+r<p+r)\vee (t+r=p+r)\qquad\[
t=p\vdash_{1.5}t+r=p+r\justifies t=p\vdash (t+r<p+r)\vee(t+r=p+r)\using{\vee_{right}}\]\justifies t<p\vee t=p\vdash (t+r<p+r)\vee (t+r=p+r)\using{\vee_{left}}
	\endprooftree$$}}

\vspace{.5cm}
\end{itemize}
	\item[(5.8)] [ $t\leq p,p<r\vdash t<r$ ]:
	\vspace{.2cm}
	\\Basta applicare le definizioni di $<$ e $\leq$ e ricordare il punto (5.2).
	\vspace{.5cm}
	\item[(5.9)] [ $\vdash 0\leq t$ ]:
	\vspace{.2cm}
	\\Applicando la definizione di $\leq$, e scegliere $w=t$ nell'$\exists_{right}$.
	\vspace{.5cm}
	\item[(5.10)] [ $\vdash 0<s(t)$ ]:
	\vspace{.2cm}
	\\Basta applicare la definizione di $<$, scegliendo $w=s(t)$ nell'$\exists_{right}$.
	\vspace{.5cm}
	\item[(5.11)] [ $\vdash t<r\ \leftrightarrow\ s(t)\leq r$ ]:
	\vspace{.2cm}
	\\Dimostrare i due versi \textsl{separatamente}, tenendo presente che in ($\rightarrow$) bisogna procedere per induzione su $r$.
	\vspace{.5cm}
	\item[(5.12)] [ $\vdash t\leq r\ \leftrightarrow\ t<s(r)$ ]:
	\vspace{.2cm}
	\\Basta applicare le definizioni di $<$ e $\leq$.
	\vspace{.5cm}
	\item[(5.13)] [ $\vdash t<s(t)$ ]:
	\vspace{.2cm}
	\\Basta applicare la definizione di $<$ e scegliere nell'$\exists_{right}$ $w=s(0)$.
	\vspace{.5cm}
	\item[(5.14)] [ $\vdash 0<\overline{1},\ \overline{1}<\overline{2},...$ ]:
	\vspace{.2cm}
	\\Considerare il punto precedente.
	\vspace{.5cm}
	\item[(5.15)] [ $t\neq r\vdash t<r\ \vee\ r<t$ ]:
	\vspace{.2cm}
	\\Per induzione su $r$.
	\vspace{.2cm}
\\Passo base:
\vspace{.2cm}
{\scriptsize{	$$\prooftree
	\[\[t\neq 0\vdash_{ax} t\neq 0\qquad\[\vdash_{A3}t+0=t\justifies t\neq 0\vdash t+0=t\using{ind}\]\justifies t\neq 0\vdash t\neq 0\mbox{\&} t+0=t\using{\mbox{\&}_{right}}\]\justifies t\neq 0\vdash\exists v(v\neq 0\mbox{\&} v+0=t)\using{\exists_{right}}\]\justifies t\neq 0\vdash t<0\vee 0<t\using{\vee_{right}}
	\endprooftree$$}}
	\vspace{.5cm}
	Per il passo induttivo mostriamo prima
	\vspace{.2cm}
	\\(5E)
 \vspace{.2cm}
	{\scriptsize{$$\prooftree
	\vdash_{A4}v+s(r)=s(v+r)\qquad s(v+r)=t\vdash_{ax} s(v+r)=t\justifies s(v+r)=t\vdash v+s(r)=t\using{tran}
	\endprooftree$$}}
	\vspace{.2cm}
	\\(5F)
	\vspace{.2cm}
	{\scriptsize{$$\prooftree
	\vdash_{1.7}s(v)+r=s(v+r)\qquad\[s(v)+r=s(v+r),s(v)+r=t\vdash_{U1}s(v+r)=t\qquad s(v+r)=t\vdash_{5E} v+s(r)=t\justifies
	s(v)+r=s(v+r),s(v)+r=t\vdash v+s(r)=t\using{cut}\]\justifies t\neq s(r),s(v)+r=t\vdash v+s(r)=t\using{cut+ind}
	\endprooftree$$}}
	\vspace{.2cm}
	\\(5G)
	\vspace{.2cm}
	{\tiny{$$\prooftree
	\[\[v=0\vdash_{1.5}v+r=0+r\qquad\vdash_{S3}0+r=r\justifies v=0\vdash v+r=r\using{tran}\]v+r=r\vdash_{U2} s(v+r)=s(r)\justifies v=0\vdash s(v+r)=s(r)\using{cut}\]\quad s(v+r)=t,s(v+r)=s(r)\vdash_{U1} t=s(r)\justifies v=0,s(v+r)=t\vdash t=s(r)\using{cut}
	\endprooftree$$}}
	\vspace{.2cm}
  (5H)
	\vspace{.2cm}
	{\scriptsize{$$\prooftree
	\[\[\vdash_{1.7}s(v)+r=s(v+r)\qquad\[s(v)+r=s(v+r),s(v)+r=t\vdash_{U1}s(v+r)=t\qquad v=0,s(v+r)=t\vdash_{5G} t=s(r)\justifies v=0,s(v)+r=s(v+r),s(v)+r=t\vdash t=s(r)\using{cut}\]\justifies v=0,s(v)+r=t\vdash t=s(r)\using{cut}\]\justifies s(v)+r=t,v=0,t\neq s(r)\vdash\bot\using{\neg_{left}}\]\justifies t\neq s(r),s(v)+r=t\vdash v\neq 0\using{\neg_{right}}
	\endprooftree$$}}
        \\
	\vspace{3cm}
	\\(5I)
	\vspace{.2cm}
	{\tiny{$$\prooftree
	\[\[\[\[\vdash_{S4}s(w)\neq 0\justifies w\neq 0\mbox{\&} w+t=r\vdash s(w)\neq 0\using{ind}\]\qquad\[\vdash_{1.7} s(w)+t=s(w+t)\qquad\[v\neq 0,w+t=r\vdash_{U2}s(w+t)=s(r)\justifies w\neq 0\mbox{\&} w+t=r\vdash s(w+t)=s(r)\using{\mbox{\&}_{left}}\]\justifies w\neq 0\mbox{\&} w+t=r\vdash s(w)+t=s(r)\using{tran}\]\justifies w\neq 0\mbox{\&} w+t=r\vdash s(w)\neq 0\mbox{\&} s(w)+t=s(r)\using{\mbox{\&}_{right}}\]\justifies w\neq 0\mbox{\&} w+t=r\vdash\exists v(v\neq
 0\mbox{\&} v+t=s(r))\using{\exists_{right}}\]\justifies\exists w(w\neq 0\mbox{\&} w+t=r)\vdash\exists v(v\neq 0\mbox{\&} v+t=s(r))\using{\exists_{left}}\]\justifies t\neq r,t<r,t\neq s(r)\vdash t<s(r)\using{ind+def. < }
	\endprooftree$$}}
 (5L)
 \vspace{.2cm}
{\scriptsize{$$\prooftree
\[\[\[\[t\neq s(r),s(v)+r=t\vdash_{5F}v+s(r)=t\qquad t\neq s(r),s(v)+r=t\vdash_{5H}v\neq 0\justifies t\neq s(r),s(v)+r=t\vdash v\neq 0\mbox{\&} v+s(r)=t\using{\mbox{\&}_{right}}\]\justifies t\neq s(r),s(v)+r=t\vdash\exists w(w\neq 0\mbox{\&} w+s(r)=t)\using{\exists_{right}}\]\justifies t\neq s(r),\exists v(s(v)+r=t)\vdash\exists w(w\neq 0\mbox{\&} w+s(r)=t)\using{\exists_{left}}\]\justifies t\neq r,r<t,t\neq s(r)\vdash s(r)<t\using{ind}\]\justifies t\neq r,r<t,t\neq s(r)\vdash t<s(r)\vee s(r)<t\using{\vee_{right}}
\endprooftree$$}}
\vspace{.5cm}
{\scriptsize{$$\prooftree
\[t\neq r,t<r,t\neq s(r)\vdash_{5I} t<s(r)\justifies t\neq r,t<r,t\neq s(r)\vdash t<s(r)\vee s(r)<t\using{\vee_{right}}\]\quad t\neq r,r<t,t\neq s(r)\vdash_{5L} t<s(r) \vee s(r)<t \justifies t\neq r \to t<r\vee r<t,t\neq s(r)\vdash t<s(r)\vee s(r)<t\using{\vee_{left}}
	\endprooftree$$}}
\vspace{.5cm}
	\item[(5.16)] [ $\vdash t=r\ \vee\ t<r\ \vee\ r<t$ ]:
	\vspace{.2cm}
	\\Per induzione su $t$.
	\vspace{.2cm}
	\\Passo base
	\vspace{.2cm}
		$$
  \prooftree \[\vdash_{5.9} 0\leq r \justifies \vdash 0=r \vee 0<r \using{def.\leq}\] \justifies \vdash 0=r \vee 0<r \vee r<0   \using{\vee_{right}}
	\endprooftree $$
	\vspace{.5cm}
	\\Passo induttivo
	\vspace{.2cm}
	$${\scriptsize{
\prooftree \[r\leq t \vdash_{5.12} r<s(t) \justifies r \leq t \vdash s(t)=r \vee s(t)<r \vee r<s(t)\using{\vee_{right}}\] \quad \[t<r\vdash_{5.11} s(t)\leq r \justifies t<r \vdash s(t)=r \vee s(t)<r \vee r<s(t)\using{\vee_{right}}\] \justifies t=r \vee t<r \vee r<t \vdash s(t)=r \vee s(t)<r \vee r<s(t) \using{\vee_{left}}
\endprooftree}
}$$	
\vspace{.5cm}
	\item[(5.17)] [ $\vdash t\leq r\ \vee\ r\leq t$ ]:
	\vspace{.2cm}
		$$\prooftree\vdash_{5.16} t=r \vee t<r \vee t>r \justifies  \vdash t \leq r \vee r \leq t \using{def. \leq + \vee_{right}}
	\endprooftree$$		
	\vspace{.5cm}
	\item[(5.18)] [ $\vdash t+r\geq t$ ]:
	\vspace{.2cm}
	\\Applicare la definizione di $\leq$ e procedere per induzione su $r$.
	\vspace{.5cm}
	\item[(5.19)] [ $r\neq 0\vdash t+r>t$  ]:
	\vspace{.2cm}
	\\Basta applicare la definizione di $>$, scegliendo $w=r$ nell'$\exists_{right}$.
	\vspace{.5cm}
	\item[(5.20)] [ $r\neq 0\vdash t*r\geq t$ ]:
	\vspace{.2cm}
	\\Applicare la definizione e procedere per induzione su $t$.
	\vspace{.5cm}
	\item[(5.21)] [ $\vdash r\neq 0\ \leftrightarrow\ r>0$ ]:
	\vspace{.2cm}
	\\Basta semplicemente applicare la definizione di $>$ e mostrare le due direzioni \textsl{separatamente}.
	\vspace{.5cm}
	\item[(5.22)] [ $r>0,t>0\vdash r*t>0$ ]:
	\vspace{.2cm}
	\\Applicare la definizione di $>$ e scegliere $w=r*t$ nell'$\exists_{right}$.
	\vspace{.5cm}
	\item[(5.23)] [ $r\neq 0,t>\overline{1}\vdash t*r>r$ ]:
	\vspace{.2cm}
	\\Applicare la definizione di $>$ e scegliere $w=v*r$ nell'$\exists_{right}$, ove $v$ \`e la variabile ottenuta nell'$\exists_{left}$.
	\vspace{.5cm}
	\item[(5.24)] [ $r\neq 0\vdash t<p\ \leftrightarrow\ t*r<p*r$ ]:
	\vspace{.2cm}
	\\Applicare la definizione di $<$ e mostrare le due direzioni \textsl{separatamente}; in $(\rightarrow)$ scegliere $w=u*r$ nell'$\exists_{right}$, dove $u$ \`e la variabile ottenuta nell'$\exists_{left}$.
	\vspace{.5cm}
	\item[(5.25)] [ $r\neq 0\vdash t\leq p\ \leftrightarrow\ t*r\leq p*r$ ]:
	\vspace{.2cm}
	\\Applicare la definizione di $\leq$, ricordando in $(\rightarrow)$ punti (1.11), e (5.24$\rightarrow$), e in $(\leftarrow)$ i punti (3.9) e (5.24$\leftarrow$).
	\vspace{.5cm}
	\item[(5.26)] [ $\vdash t\not< 0$ ]:
	\vspace{.2cm}
	\\Basta applicare la definizione di $\not<$ e il punto (3.4).
	\vspace{.5cm}
	\item[(5.27)] [ $t\leq r\ \mbox{\&}\ r\leq t\vdash t=r$ ]:
	\vspace{.2cm}
	\\Applicare la definizione di $\leq$.
\end{enumerate}
\hspace{\stretch{1}}$\Box$\\

\vspace{.5cm}
\newpage
\begin{prop}
Per ogni numero naturale $k$, e ogni $\varphi$ formula si hanno:
\vspace{.2cm}
\begin{enumerate}
	\item[(6.1.1)] $\vdash x=0\ \vee\ ...\ \vee\ x=\overline{k}\ \leftrightarrow\ x\leq \overline{k}$
	\vspace{.2cm}
	\item[(6.1.2)] $\vdash \varphi(0)\ \mbox{\emph{\&}}\ ...\ \mbox{\emph{\&}}\ \varphi(\overline{k})\ \leftrightarrow\ \forall x (x\leq \overline{k}\ \to\ \varphi(x))$
	\vspace{.2cm}
	\item[(6.2.1)] $\vdash x=0\ \vee\ ...\ \vee\ x=\overline{k-1}\ \leftrightarrow\ x<\overline{k}$ dove $k>0$
	\vspace{.2cm}
	\item[(6.2.2)] $\vdash \varphi(0)\ \mbox{\emph{\&}}\ ...\ \mbox{\emph{\&}}\ \varphi(\overline{k-1})\ \leftrightarrow\ \forall x (x<\overline{k}\ \to\ \varphi(x))$, dove $k>0$
	\vspace{.2cm}
	\item[(6.3)] $\vdash (\forall x (x<y\ \to\ \varphi(x))\ \mbox{\emph{\&}}\ \forall x (x\geq y\ \to\ \psi(x)))\ \to\ \forall x (\varphi(x)\ \vee\ \psi(x))$
\end{enumerate}
\end{prop}
\vspace{.6cm}
\textsc{Dimostrazione}
\vspace{.2cm}
\begin{enumerate}
  \item[(6.1.1)] Vediamo prima ($\rightarrow$)\\
  \vspace{.2cm}
	$$ \scriptsize{\prooftree
	\[\[\[\[x=0 \vdash_{1.5} x+\overline{k}=0+\overline{k}\qquad \vdash_{S3} 0+\overline{k}=\overline{k}\justifies x=0 \vdash x+\overline{k}= \overline{k}\using{tran}\]\justifies x=0 	\vdash \exists w (w+x)=\overline{k}\using{\exists_{right}}\] \justifies x=0 \vdash x \leq \overline{k} \using{def. \leq}\]\quad \cdots \quad \[ \cdots \justifies x= \overline{k} \vdash x \leq \overline{k}\] \justifies x=0 \vee x= \overline{1} \vee \cdots \vee x=\overline{k} \vdash x\leq \overline{k} \using{\vee_{left}}\] \justifies \vdash x=0 \vee x=\overline{1} \vee \cdots \vee x= \overline{k} \rightarrow x \leq \overline{k}\using{\rightarrow_{right}}
	\endprooftree}$$
  \vspace{.2cm}\\
  ($\leftarrow$) Per induzione su k, ricordando la (5.12)\\
\\ Passo base:
\vspace{.2cm}
$$
{\prooftree
\[x=0 \vdash_{ax} x=0 \justifies x=0 \vdash x=0\]  \quad \[x<0 \vdash \bot \quad \bot \vdash x=0 \justifies x<0 \vdash x=0 \using{cut}\] \justifies x \leq 0 \vdash x=0
\endprooftree}
$$
\vspace{.2cm}
Passo induttivo:\\ Dimostriamo prima (6A)
$$
{\prooftree
\vdash_{5.12} x < s(\overline{k}) \rightarrow x \leq \overline{k} \quad \[x<s(\overline{k}) \vdash_{ax} x<s(\overline{k})\quad x \leq \overline{k} \vdash_{ax} x \leq \overline{k} \justifies x<s(\overline{k}), x<s(\overline{k})\rightarrow x\leq \overline{k} \vdash x \leq \overline{k} \using{\rightarrow_{left}}\] \justifies x<s(\overline{k}) \vdash x\leq \overline{k} \using{cut}
\endprooftree}
$$
E (6B):\\
\vspace{.2cm}
$$
{\prooftree
x \leq \overline{k} \vdash_{ax} x \leq \overline{k} \qquad x=0\ \vee\ ...\ \vee\ x=\overline{k} \vdash_{ax} x=0\ \vee\ ...\ \vee\ x=\overline{k}
\justifies x \leq \overline{k}, x\leq\overline{k}\rightarrow x=0\ \vee\ ...\ \vee\ x=\overline{k} \vdash  x=0\ \vee\ ...\ \vee\ x=\overline{k} \using{\rightarrow_{left}}
\endprooftree}
$$\\
\vspace{.2cm}
Allora si ottiene:
$$
\scriptsize{
\prooftree
\[\[x<s(\overline{k})\vdash_{6A} x\leq\overline{k}\quad x\leq\overline{k}, hp. indutt.\vdash_{6B} x=0\ \vee\ ...\ \vee\ x=\overline{k} \justifies x<s(\overline{k}), hp. ind.\vdash x=0\ \vee\ ...\ \vee x=s(\overline{k}) \using{cut + \vee_{right}}\]  \quad x=s(\overline{k}) \vdash x=0\ \vee\ ...\ \vee x=s(\overline{k}) \justifies x\leq s(\overline{k}), hp. ind. \vdash x=0\ \vee\ ...\ \vee x=s(\overline{k}) \using{\vee_{left}}\] \justifies hp. induttiva\vdash x\leq s(\overline{k}) \rightarrow x=0\ \vee\ ...\ \vee x=s(\overline{k}) \using{\rightarrow{right}}
\endprooftree}
$$
\vspace{.5cm}
  I tre punti seguenti 6.1.2, 6.2.1, 6.2.2 derivano dal precedente.
  \vspace{.5cm}
  \item[(6.3)]
  \vspace{.2cm}
 {\scriptsize $$\prooftree
  \[\[\vdash_{6C}z<y\vee y\leq z\qquad \[\[\[z<y\vdash_{ax} z<y\qquad\varphi(z)\vdash_{ax}\varphi(z)\justifies z<y,z<y\rightarrow\varphi(z)\vdash\varphi(z)\using{\rightarrow_{left}}\]
\justifies z<y,z<y\rightarrow\varphi(z)\vdash\varphi(z)\vee\psi(z)\using{\vee{right}}\]\qquad\[\[y\leq z\vdash_{ax} y\leq z\qquad\psi(z)\vdash_{ax}\psi(z)\justifies y\leq z,y\leq z\rightarrow\psi(z)\vdash\psi(z)\using{\rightarrow_{left}}\]\justifies y\leq z,y\leq z\rightarrow\psi(z)\vdash\psi(z)\vee\psi(z)\using{\vee_{right}}\]\justifies z<y\vee y\leq z,z<y\rightarrow\varphi(z),z\geq y\rightarrow\psi(z)\vdash\varphi(z)\vee\psi(z)\using{\vee_{left}+ind}\]\justifies z<y\rightarrow\varphi(z),z\geq y\rightarrow\psi(z)\vdash\varphi(z)\vee\psi(z)\using{cut}\]\justifies\forall x(x<y\rightarrow\varphi(x))\mbox{\&}\forall x(x\geq y\rightarrow\psi(x))\vdash\varphi(z)\vee\psi(z)\using{\forall_{left}+\mbox{\&}_{left}}\]\justifies\forall x(x<y\rightarrow\varphi(x))\mbox{\&}\forall x(x\geq y\rightarrow\psi(x))\vdash\forall x(\varphi(x)\vee\psi(x))\using{\forall_{right}}
  \endprooftree$$}
  \vspace{.2cm}
  dove (6C) deriva direttamente da (5.16) e dalla definizione di $\leq$.
\end{enumerate}
\hspace{\stretch{1}} $\Box$\\

\vspace{.5cm}
La definizione di una funzione in HA viene fatta indirettamente ogniqualvolta si prenda un termine $t(x)$ (ovvero in cui sia presente la variabile $x$ libera), e vi si sostituisca dentro un numerale $\overline{n}$, ottenendo cos\`i $t(\overline{n})$; tuttavia, nel formalismo base di HA sono stati definiti segni per rappresentare solo l'addizione e la moltiplicazione, ma manca il termine $t$ per cui $t(\overline{n})=\overline{f(x)}$, quindi non si pu\`o controllare che il formalismo rappresenti proprio la funzione $f$ da noi voluta; solo nel caso dell'addizione e del prodotto, infatti, si pu\`o dimostrare (come abbiamo fatto nella proposizione (3.4)) che il risultato di tali operazioni sui numerali \`e proprio quello corretto. Per indicare altre funzioni diverse da somma e prodotto, dobbiamo ricorrere alla loro caratterizzazione tramite formule, mentre per la loro rappresentabilit\`a, bisogna attendere il prossimo capitolo.

\begin{defi}
Diciamo che $t$ divide $p$ se esiste $z$ che, moltiplicato per $t$, d\`a $p$. Ossia\ $t|p\ :=\ \exists\ z(p=t*z)$.
\end{defi}

\begin{prop} Per ogni termine $p$, $r$, $t$ valgono:
\vspace{.2cm}
\begin{enumerate}
	\item[(7.1)] $\vdash t|t$
	\vspace{.2cm}
	\item[(7.2)] $\vdash\overline{1}|t$
	\vspace{.2cm}
	\item[(7.3)] $\vdash t|0$
	\vspace{.2cm}
	\item[(7.4)] $t|p\ \mbox{\emph{\&}}\ p|r\vdash t|r$
	\vspace{.2cm}
	\item[(7.5)] $p\neq 0\ \mbox{\emph{\&}}\ t|p\vdash t\leq p$
	\vspace{.2cm}
	\item[(7.6)] $t|p\ \mbox{\emph{\&}}\ p|t\vdash p=t$
	\vspace{.2cm}
	\item[(7.7)] $t|p\vdash t|(r*p)$
	\vspace{.2cm}
	\item[(7.8)] $t|p\ \mbox{\emph{\&}}\ t|r\vdash t|(p+r)$
\end{enumerate}
\end{prop}

\vspace{.5cm}
\textsc{Dimostrazione}
\begin{enumerate}
  \item[(7.1)] Applicare la definizione e scegliere $z=s(0)$ nell'$\exists_{right}$.
  \vspace{.5cm}
  \item[(7.2)] Applicare la definizione e scegliere $z=t$ nell'\ $\exists_{right}$.
  \vspace{.5cm}
  \item[(7.3)] Applicare la definizione e scegliere $z=0$ nell'\ $\exists_{right}$.
  \vspace{.5cm}
  \item[(7.4)] Applicare la definizione e scegliere $z=v*w$ nell'\ $\exists_{right}$, ove $v$ e $w$ sono le variabili ottenute nelle $\exists_{left}$.
  \vspace{.5cm}
  \item[(7.5)] Per induzione su $z$, ove $z$ \`e la variabile ottenuta nell'\ $\exists_{left}$.
  \vspace{.5cm}
  \item[(7.6)] Applicare la definizione.
  \vspace{.5cm}
  \item[(7.7)] Applicare la definizione e scegliere $z=v*r$ nell'\ $\exists_{right}$, ove $v$ \`e la variabile ottenuta nell'\ $\exists_{left}$.
  \vspace{.5cm}
  \item[(7.8)] Applicare la definizione e scegliere $z=v+w$ nell'\ $\exists_{right}$, ove $v$ e $w$ sono le variabili ottenute nelle $\exists_{left}$.
\end{enumerate}
\hspace{\stretch{1}} $\Box$\\

\vspace{.5cm}
Dimostriamo ora l' esistenza e l' unicit\`a del resto e del quoto nell'algoritmo di divisione.

\begin{prop}
{\tiny{$$y\neq 0\vdash \exists q\ \exists r((x=y*q+r\ \mbox{\emph{\&}} r<y)\ \mbox{\emph{\&}} \forall u\ \forall v\ ((x=y*u+v\ \mbox{\emph{\&}} v<y)\ \to\ q=u\ \mbox{\emph{\&}} r=v)))$$}}
\end{prop}
\vspace{.5cm}
\textsc{Dimostrazione}
\\La dimostrazione sarebbe molto lunga con le derivazioni, per questo motivo procederemo diversamente, indicando solo come proseguire.\footnote{Comunque \textsl{esiste} una derivazione}\newline
Proviamo l'esistenza: sia $\varphi(x):=\ y\neq 0\vdash \exists q\ \exists r(x=y*q+r\ \mbox{\&}\ r<y)$. $\varphi(x)$ si dimostra facilmente per induzione su $x$. Per $x=0$ si utilizzano la (3.4) e la (3.5) per dimostrare che $q=0$ e $r=0$ vanno bene.\\ Nel passo induttivo, bisogna ragionare cos\`i: siano $q$, $r$ tali che $x=y*q+r$; se $r+1$ resta minore di $y$, allora $q$ e $r+1$ sono il quoto e il resto cercati per $s(x)$, invece se $r+1=y$, allora $q+1$ e $0$ sono il quoto e il resto cercati.\newline
Per provare l'unicit\`a, si assuma $y\neq 0$, $x=y*q+r \mbox{\&}\ r<y$ e $x=y*u+v\mbox{\&}\ v<y$. Dalla (5.16), si ha che, nel confrontare $q$ e $u$, ci sono solo tre possibilit\`a: $q=u\vee q<u\vee q>u$.  $q<u$ non si pu\`o presentare, perch\`e equivale a $\exists\ w(w\neq 0\ \mbox{\&}\ w+q=u)$; questo implica $y*q+r= y*(w+q)+v$, e quindi $r=y*w+v$, da cui $r\geq y$, poich\`e $y*w\geq y$, essendo $w$ non nullo, per la (5.20). \\Pertanto, abbiamo dimostrato che $q<u \vdash \bot $, quindi $\neg (q<u)$. Analogamente, neanche $q>u$ si presenta, perch\`e equivale a $\exists\ w(w\neq 0\ \mbox{\&}\ w+u=q)$, che implica $y*u+v= y*(w+u)+r$, cio\`e $v=y*w+r$, che implica $v\geq y$, per cui anche qui si arriva a $\neg (q>u)$. \\Resta quindi valida solo la possibilit\`a $q=u$. Per l'unicit\`a del resto della divisione, si sfrutta l'unicit\`a del quoziente appena dimostrata; si ha che $x=y*q+r=y*q+v$, e per i resti si hanno le tre possibilit\`a $r=v\vee r<v\vee r>v$. Se $r>v$ vuol dire che $\exists w(w\neq 0\ \mbox{\&}\ w+v=r)$, da cui $y*q+(w+v)=y*q+v$, che implica $w=0$; quindi anche qui abbiamo dimostrato $r>v \vdash \bot$. Lo stesso si fa con $r<v$, e l'unica possibilit\`a che resta \`e $r=v$, come voluto.
\hspace{\stretch{1}} $\Box$\\

\newpage Concludiamo ora facendo vedere come sia possibile in HA definire formalmente alcuni concetti fondamentali in matematica.

\begin{defi}
Definiamo le seguenti:
\vspace{.1cm}
\begin{itemize}
\item sottrazione adeguata($\dot{-}$)\footnote{$z$ \`e legato a $x$ e $y$ tramite $\dot{-}$ se soddisfa alla propriet\`a nella definizione. Questo \`e diverso dal dire che $z$ \`e il risultato della funzione $\dot{-}$ applicata a $x$ e $y$, nonostante la notazione matematica sia la stessa; \`e il problema a cui si \`e accennato prima di introdurre la divisione}: {\footnotesize{$$z=y\dot{-} x:=\ (x<y\ \mbox{\emph{\&}} x+z=y)\ \vee\ (y\leq x\ \mbox{\emph{\&}} z=0)$$}}
\vspace{.1cm}
\item $z$ \`e il resto della divisione di $y$ per $x$:{\footnotesize{ $$z=rm(x,y)\ :=\ (x\neq 0\ \ \exists u(y=u*x+z)\ \mbox{\emph{\&}} z<x)\ \vee\ (x=0\ \mbox{\emph{\&}} z=y)$$}}
\vspace{.1cm}
\item $x$ \`e un numero primo: {\footnotesize{$$Prime(x):=\ x>1\ \mbox{\emph{\&}} \forall yz\  (x=y*z\ \to\ y=x\ \vee\ y=1)$$}}
\end{itemize}
\end{defi}

\vspace{.2cm}
\begin{prop} Valgono:
\vspace{.2cm}
\begin{enumerate}
	\item[(11.1)] $\vdash Prime(x)\ \leftrightarrow\ x>1\ \mbox{\emph{\&}} \forall y\ (y|x\ \to\ y=1\ \vee\ y=x)$
	\vspace{.2cm}
	\item[(11.2)] $\vdash Prime(x)\ \leftrightarrow\ x>1\ \mbox{\emph{\&}}\ \forall yz\ (x|y*z\ \to\ x|y\ \vee\ x|z)$
\end{enumerate}
\end{prop}
\vspace{.5cm}
\textsc{Dimostrazione}
Basta applicare le definizioni.
\hspace{\stretch{1}} $\Box$\\


